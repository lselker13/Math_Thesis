\documentclass{beamer}
\usetheme{AnnArbor}
\usecolortheme{seahorse}
\input{commands_packages}
\graphicspath{{Figures/}}
\newcommand{\Col}{\text{Col}}
\usepackage{tabu, rotating, graphicx, color, multicol, empheq}
\usepackage[framemethod=TikZ]{mdframed}

\newcommand<>{\colorize}[4]{\only<#1>{#4}\only<#2>{\textcolor{#3}{#4}}}
\newcommand{\after}[2]{\colorize{1}{2-}{#1}{#2}}
\newcommand{\ago}{\after{green}{1}}
\newcommand{\go}{\textcolor{green}{1}}
\newcommand{\ono}{\textcolor{orange}{-1}}

\newcommand{\g}[1]{\textcolor{gray}{#1}}
\usepackage[euler-digits]{eulervm} 
\usetikzlibrary{decorations.pathreplacing,angles,quotes}

\begin{document}
\title{M\"obius Inversion}   
\author{Leo Selker \\ Advisor: Vin de Silva} 
\date{\today} %February 10, 2017

\frame{\titlepage} 

\section{Adding Up Integers}
\frame{\frametitle{Cumulative Sums}
We start with some sequences of numbers:

\begin{align*}
& f: 1, 1, 1, 1, \dots  && f: 2, 2, 2, 2, \dots  && f: 1, 2, 3, 4, \dots \\
& g: 1, 2, 3, 4, \dots  && g: 2, 4, 6, 8, \dots  && g: 1, 3, 6, 10, \dots
\end{align*}
\vspace{.3cm}\\

\begin{enumerate}[]
\item<2-> In each case, $g_n = f_1 + f_2 + \cdots + f_n$, a cumulative sum.\vspace{.2cm}
\item<3->\[g_n = \sum_{m \leq n} f_n\]
\end{enumerate}
}

\frame{\frametitle{Going the Other Way}
How do we get from $g$ back to $f$?\\

 

\begin{align*}
& f: 1, 1, \after{green}{1}, 1, \dots  && f: 2, 2, \after{green}{2}, 2, \dots  && f: 1, 2, \after{green}{3}, 4, \dots \\
& g: 1, \after{green}{2, 3}, 4, \dots  && g: 2, \after{green}{4, 6}, 8, \dots  && g: 1, \after{green}{3, 6}, 10, \dots
\end{align*}
\onslide<2->\[
f_n = g_n - g_{n-1}
\]
\only<3>{
\color{purple}
\begin{empheq}[box=\fbox]{align*} %TODO: Improve
g_n &= \textcolor{blue}{f_1 + ... + f_{n-1}} + f_n\\
- g_{n-1} &=\textcolor{blue}{f_ + ... + f_{n-1}}\\
g_n - g_{n-1} &= f_n.
\end{empheq}
}
\only<4>{
\begin{align*}
&\text{Let }\mu(m,n) = \begin{cases}
1 & m = n\\
-1 & m = n-1\\
0 & else.
\end{cases}
& \text{Then }f_n = \sum_{m \leq n}g_m\mu(m,n).
\end{align*}
}

}
\frame{\frametitle{Another Perspective}
We consider the first five values of each sequence as a vector:
\begin{align*}
& \mathbf{f} = [1, 1, 1, 1, 1]   && \mathbf{f} = [2, 2, 2, 2, 2]  && \mathbf{f} = [\after{green}{1, 2, 3, 4}, 5] \\
& \mathbf{g} = [1, 2, 3, 4, 5]  && \mathbf{g} = [2, 4, 6, 8, 10]  && \mathbf{g} = [1, 3, 6, \after{green}{10}, 15]
\end{align*}
And we encode the adding up process in a matrix:
\[\zeta = 
\begin{bmatrix}
1 & 1 & 1 & \after{green}{1} & 1\\
0 & 1 & 1 & \after{green}{1} & 1\\
0 & 0 & 1 & \after{green}{1} & 1\\
0 & 0 & 0 & \after{green}{1} & 1\\
0 & 0 & 0 & 0 & 1\\
\end{bmatrix}
\]
Then:
\[ \mathbf{g} = \mathbf{f} \zeta\]
}
\frame{\frametitle{Inverting $\zeta$ to get $\mu$}

\begin{align*}\zeta  = 
&\begin{bmatrix}
\begin{array}{rrrrr}
1 & 1 & 1 & 1 & 1\\
0 & 1 & 1 & 1 & 1\\
0 & 0 & 1 & 1 & 1\\
0 & 0 & 0 & 1 & 1\\
0 & 0 & 0 & 0 & 1\\
\end{array}
\end{bmatrix}\hspace{1cm}
\mu =\zeta^{-1} = 
&&\begin{bmatrix}
\begin{array}{rrrrr}
1 & -1 & 0 & 0 & 0\\
0 & 1 & -1 & 0 & 0\\
0 & 0 & 1 & -1 & 0\\
0 & 0 & 0 & 1 & -1\\
0 & 0 & 0 & 0 & 1\\
\end{array}
\end{bmatrix}\\
\vspace{.5cm}\\
& g_n = \sum_{m \leq n} f_n \textcolor{gray}{\zeta(m,n)}&& f_n  = \sum_{m \leq n}g_m\mu(m,n)\\
\end{align*}
}
\frame{\frametitle{Using $\mu$ to get $\mathbf{f}$}
We said that $\mathbf{g} = \mathbf{f}\zeta$, so $\mathbf{f} = \mathbf{g}\mu$.
An example:
\begin{align*}
& \mathbf{f} = [1, 2, 3, \after{green}{4}, 5]\\
& \mathbf{g} = [1, 3, \after{blue}{6}, \after{purple}{10}, 15] 
\end{align*}
\begin{align*}
\mathbf{f}= \mathbf{g}\mu = [1, 3, \after{blue}{6}, \after{purple}{10}, 15]\begin{bmatrix}
\begin{array}{rrrrr}
1 & -1 & 0 & 0 & 0\\
0 & 1 & -1 & 0 & 0\\
0 & 0 & 1 & \after{blue}{-1} & 0\\
0 & 0 & 0 & \after{purple}{1} & -1\\
0 & 0 & 0 & 0 & 1\\
\end{array}
\end{bmatrix}
\end{align*}
\onslide<2->

\[\textcolor{green}{4} = \textcolor{blue}{(-1)6} + \textcolor{purple}{(1)10}
\]
}
\frame{\frametitle{Analogy to Calculus}
What we did looks familiar.
\begin{align*}
\mathbf{g} = \mathbf{f}\zeta &\iff \mathbf{f} = \mathbf{g} \mu \\
\vspace{.5cm}\\
g(n) = \sum_{m \leq n}f(m) &\iff f(n) = \sum_{m \leq n}g(m)\mu(m,n)\\
\vspace{.8cm}\\
\only<2->{\color{blue}``g(x) = \int_1^x f(t)dt &\color{blue}\iff f(t) = \frac{d}{dx}g(t)"}
\end{align*}

}

\frame{\frametitle{Another Dimension}
We can represent functions on $m,n$ using tables of values:
\begin{align*}
f: \begin{tabu}{c|cccc}
& & & &n\\
\hline
& \ago  & \ago & \ago & \ago\\
& \ago & \ago & \ago& \ago\\
m & \ago & \ago & \ago & \ago
\end{tabu}\hspace{1cm}
g: 
\begin{tabu}{c|cccc}
& & & & n\\
\hline
& 1  & 2 & 3 & 4\\
& 2 & 4 & 6 &10\\
m & 3 & 6 & 9 & \after{green}{12} 
\end{tabu}
\end{align*}
We have
\[
g(m,n) = \sum_{(m',n')\leq(m,n)}f(m', n'),
\]
a generalized cumulative sum.
}
\frame{\frametitle{Another Dimension}
How do we get $f$ from $g$?
\begin{align*}
f: \begin{tabu}{c|cccc}
& & & &n\\
\hline
& 1  & 1 & 1 & 1\\
& 1 & 1 & 1& 1\\
m & 1 & 1 & 1 & \ago
\end{tabu}\hspace{1cm}
g: 
\begin{tabu}{c|cccc}
& & & & n\\
\hline
& 1  & 2 & 3 & 4\\
& 2 & 4 & \colorize{1-5}{6-}{blue}{6} & \colorize{1-4}{5-}{brown}{10}\\
m & 3 & 6 & \colorize{1-3}{4-}{orange}{9} & \colorize{1-2}{3-}{purple}{12}
\end{tabu}
\end{align*}

\begin{tikzpicture}[overlay]
\only<3->{\draw[purple,thick,rounded corners] (3.3,.5) rectangle (5.3,1.83);}
\only<4->{\draw[orange,thick,rounded corners] (3.3,.5) rectangle (4.8,1.83);}
\only<5->{\draw[brown,thick,rounded corners] (3.3,.9) rectangle (5.3,1.83);}
\only<6->{\draw[blue,thick,rounded corners] (3.3,.9) rectangle (4.8,1.83);}
\end{tikzpicture}
\[
\onslide<2-> \textcolor{green}{1} = 
\onslide<3->  \textcolor{purple}{12} 
\onslide<4-> - \textcolor{orange}{9}
\onslide<5-> - \textcolor{brown}{10}
\onslide<6-> + \textcolor{blue}{6}
\]
\onslide<7->
In general (finding $\mu$ would have told us this):
\[
\textcolor{green}{f(m,n)} = \textcolor{purple}{g(m,n)} - \textcolor{orange}{g(m, n-1)} - \textcolor{brown}{g(m-1, n)} + \textcolor{blue}{g(m-1, n-1)}
\]
}

\frame{\frametitle{Two Equivalent Statements}
$X, \leq$ a partially ordered set.

\begin{mdframed}
$f(x),g(x)$ functions, then
\[
g(x) = \sum_{y \leq x}f(y) \iff f(x) = \sum_{y \leq x}g(y)\color{purple}{\mu(y,x)}
\]
\end{mdframed}
\begin{mdframed}
$\bf{f}, \bf{g}$ vectors, $\zeta$ encodes $\leq$, $\mu = \zeta^{-1}$. Then
\[
\mathbf{g} = \mathbf{f}\zeta \iff \mathbf{f} = \mathbf{g}\mu
\]
\end{mdframed}
We use the matrix $\mu$ to find the function $\mu$, and to find $f$ given $g$.
}
\frame{\frametitle{What do we know about $\zeta$ and $\mu$?}
Some facts about $\zeta$ and $\mu$:
\begin{itemize}
\item $\zeta$ is upper triangular, with ones on the main diagonal. Therefore:
\item $\zeta$ is invertible, so we can write $\mu = \zeta^{-1}$, and
\item $\mu$ will have integer values.
\end{itemize}
Often, entries of $\mu$ will be -1, 0, 1. 

}
\section{Euler's Totient}
\frame{\frametitle{Integers and Divisibility}
We now replace $\leq$ with $|$ ($a|b$ if $a$ divides $b$ without remainder).  Some functions:
\begin{align*}
& f: 1, 1, 1, 1, 1, 1, 1, 1, \dots  && f: 0, 1, 0, 0, 0, 0,0, 0,  \dots  \\
& g: 1, 2, 2, 3, 2, 4, 2, 4, \dots  && g: 0, 1, 0, 1, 0, 1, 0, 1,\dots \\
\end{align*}
\onslide<2->
The first $g$ is the number of divisors. The second indicates the even numbers.

\onslide<3->
We will construct the matrices $\zeta$ and $\mu$ for the integers up to 10, and see what it tells us about how to go back from $g$ to $f$.
}
\frame{\frametitle{Constructing $\zeta$}
\[
\zeta = \bordermatrix{
 & \g{1} & \g{2} & \g{3} & \g{4} & \g{5} & \g{6} & \g{7} & \g{8} & \g{9} & \g{10} \cr
\g{1} & \go & \go & \go & \go & \go & \go & \go & \go & \go & \go  \cr
\g{2} & 0 & \go & 0 & \go & 0 & \go & 0 & \go & 0 & \go  \cr
\g{3} & 0 & 0 & \go & 0 & 0 & \go & 0 & 0 & \go & 0  \cr
\g{4} & 0 & 0 & 0 & \go & 0 & 0 & 0 & \go & 0 & 0  \cr
\g{5} & 0 & 0 & 0 & 0 & \go & 0 & 0 & 0 & 0 & \go  \cr
\g{6} & 0 & 0 & 0 & 0 & 0 & \go & 0 & 0 & 0 & 0  \cr
\g{7} & 0 & 0 & 0 & 0 & 0 & 0 & \go & 0 & 0 & 0  \cr
\g{8} & 0 & 0 & 0 & 0 & 0 & 0 & 0 & \go & 0 & 0  \cr
\g{9} & 0 & 0 & 0 & 0 & 0 & 0 & 0 & 0 & \go & 0  \cr
\g{10}& 0 & 0 & 0 & 0 & 0 & 0 & 0 & 0 & 0 & \go  \cr
}\]
}
\frame{\frametitle{Inverting $\zeta$ to get $\mu$}
\[
\zeta = \bordermatrix{
 & \g{1} & \g{2} & \g{3} & \g{4} & \g{5} & \g{6} & \g{7} & \g{8} & \g{9} & \g{10} \cr
\g{1} & \go & \ono & \ono & 0 & \ono & \go & \ono & 0 & 0 & \go  \cr
\g{2} & 0 & \go & 0 & \ono & 0 & \ono & 0 & 0 & 0 & \ono  \cr
\g{3} & 0 & 0 & \go & 0 & 0 & \ono & 0 & 0 & 0 & 0  \cr
\g{4} & 0 & 0 & 0 & \go & 0 & 0 & 0 & \ono & 0 & 0  \cr
\g{5} & 0 & 0 & 0 & 0 & \go & 0 & 0 & 0 & 0 & \ono  \cr
\g{6} & 0 & 0 & 0 & 0 & 0 & \go & 0 & 0 & 0 & 0  \cr
\g{7} & 0 & 0 & 0 & 0 & 0 & 0 & \go & 0 & 0 & 0  \cr
\g{8} & 0 & 0 & 0 & 0 & 0 & 0 & 0 & \go & 0 & 0  \cr
\g{9} & 0 & 0 & 0 & 0 & 0 & 0 & 0 & 0 & \go & 0  \cr
\g{10}& 0 & 0 & 0 & 0 & 0 & 0 & 0 & 0 & 0 & \go  \cr
}\]
}
\frame{\frametitle{What is $\mu$?}
It turns out that
\[
\mu(a,b) = \begin{cases}
0 & a \nmid b \text{ or } \frac ba \text{ is divisible by a square}\\
(-1)^{\text{\#prime factors of }\frac ba} & else
\end{cases}
\]
}
\frame{\frametitle{What is $\mu$?}
\[
\small
\zeta = \bordermatrix{
 & \g{1} & \g{2} & \g{3} & \g{4} & \g{5} & \g{6} & \g{7} & \g{8} & \g{9} & \g{10} \cr
\g{1} & \go & \ono & \ono & 0 & \ono & \textcolor{purple}{1} & \ono & 0 & 0 & \go  \cr
\g{2} & 0 & \go & 0 & \ono & 0 & \ono & 0 & 0 & 0 & \ono  \cr
\g{3} & 0 & 0 & \go & 0 & 0 & \ono & 0 & 0 & 0 & 0  \cr
\g{4} & 0 & 0 & 0 & \go & 0 & 0 & 0 & \ono & 0 & 0  \cr
\g{5} & 0 & 0 & 0 & 0 & \go & 0 & 0 & 0 & 0 & \ono  \cr
\g{6} & 0 & 0 & 0 & 0 & 0 & \go & 0 & 0 & 0 & 0  \cr
\g{7} & 0 & 0 & 0 & 0 & 0 & 0 & \go & 0 & 0 & 0  \cr
\g{8} & 0 & 0 & 0 & 0 & 0 & 0 & 0 & \go & 0 & 0  \cr
\g{9} & 0 & 0 & 0 & 0 & 0 & 0 & 0 & 0 & \go & 0  \cr
\g{10}& 0 & 0 & 0 & 0 & 0 & 0 & 0 & 0 & 0 & \go  \cr
}\]

\[\color{purple}\mu(1,6) = (-1)^2 = 1\]

}
\frame{\frametitle{Visualizing $\mu$}
\only<1>{
\begin{align*}
& f: 1, 1, 1, \textcolor{gray}{1, 1}, 1, \textcolor{gray}{1, 1}, \dots  \\
& g: 1, 2, 2, \textcolor{gray}{3,2}, 4, \textcolor{gray}{2,4}, \dots  
\end{align*}
}
\only<2->{
\[
\mu(a,b) = \begin{cases}
0 & a \nmid b \text{ or } \frac ba \text{ is divisible by a square}\\
(-1)^{\text{\#prime factors of }\frac ba} & else
\end{cases}
\]}
Values of $f,g$ on multiples of $2, 3$:
\begin{align*}
f: \begin{tabu}{c|cccc}
&2^0 &2^1 &2^2 &2^3\\
\hline
3^0& 1  & 1 & 1 & 1\\
3^1& 1 & 1 & 1& 1\\
3^2 & 1 & 1 & 1 & \ago
\end{tabu}\hspace{1cm}
g: 
\begin{tabu}{c|cccc}
&2^0 &2^1 &2^2 &2^3\\
\hline
3^0& 1  & 2 & 3 & 4\\
3^1& 2 & 4 & \colorize{1-5}{6-}{blue}{6} & \colorize{1-4}{5-}{brown}{10}\\
3^2 & 3 & 6 & \colorize{1-3}{4-}{orange}{9} & \colorize{1-2}{3-}{purple}{12}
\end{tabu}
\end{align*}

\begin{tikzpicture}[overlay]
\only<3->{\draw[purple,thick,rounded corners] (2.8,.5) rectangle (5.3,1.83);}
\only<4->{\draw[orange,thick,rounded corners] (2.8,.5) rectangle (4.8,1.83);}
\only<5->{\draw[brown,thick,rounded corners] (2.8,.9) rectangle (5.3,1.83);}
\only<6->{\draw[blue,thick,rounded corners] (2.8,.9) rectangle (4.8,1.83);}
\end{tikzpicture}
\[
\onslide<2-> \textcolor{green}{f_{72} = 1} = 
\onslide<3->  \textcolor{purple}{(-1)^012} 
\onslide<4-> + \textcolor{orange}{(-1)^19}
\onslide<5-> + \textcolor{brown}{(-1)^110}
\onslide<6-> + \textcolor{blue}{(-1)^26}
\]
}
\frame{\frametitle{More Prime Factors}
\begin{multicols}{2}
Values of $g$:
\only<1>{
\[
\includegraphics[scale = .1]{../Figures/divisibility_lattice.png}
\]
}
\only<2>{
\[
\includegraphics[scale = .08]{../Figures/divisibility_lattice_incexc.png}
\]
}
\onslide<2>
\begin{empheq}[box=\fbox]{align*}
\textcolor{green}{f(60)}& = \textcolor{green}{g(\frac{60}{1})} \\
&- \textcolor{blue}{g(\frac{60}{2}) - g(\frac{60}{3}) - g(\frac{60}{5})}\\
&+ \textcolor{orange}{g(\frac{60}{6}) + \frac({60}{10}) + \frac({60}{15})}\\
&- \textcolor{blue}{g(\frac{60}{30})}
\end{empheq}
\end{multicols}
}

\frame{\frametitle{Euler's Totient}
\begin{itemize}
\item Recall that Euler's Totient Function, or $\phi(n)$, gives us how many numbers less than or equal to $n$ are relatively prime to $n$.
\item Put another way: $\phi(n) = \#\{1 \leq m \leq n | \gcd(m,n) = 1\}$.
\item So $\phi(10) = 4: \textcolor{red}{1} 2 \textcolor{red}{3} 4 5 6 \textcolor{red}{7} 8 \textcolor{red}{9}$ theorem
\item Is there a way to write values of this function down without checking every number up to $n$?
\end{itemize}
}
\frame{\frametitle{Setup for M\"obius Inversion}
A fact we will start from:
\[ n = \sum_{d|n} \phi(d).
\]
Why? Consider $n=10$:
\[
\begin{tabu}{c|ccc}
d & \phi(d) & k \text{ counted by } \phi & \{\frac nd k\}\\
\hline
1 & 1 & \{1\} & \color{orange}\{10\}\\
2 & 1 & \{1\} &\color{green} \{5\}\\
5 & 4 & \{1,2,3,4\} &\color{blue} \{2,4,6,8\}\\
10 & 4 & \{1,3,7,9\} &\color{brown} \{1,3,7,9\}
\end{tabu}
\]
In the righthand column, we partitioned 10 numbers into four sets:
\[
\begin{tabu}{cccc}
\color{orange} \gcd(\frac nd k,n) = 10 &\color{green} \gcd(\frac nd k,n) = 5 & \color{blue} \gcd(\frac nd k,n) = 2 &\color{brown} \gcd(\frac nd k,n) = 1
\end{tabu}
\]
and added up the partitions' sizes. So we got 10.
\iffalse
\[
\color{brown}\textcolor{orange}{1} \textcolor{blue}{2} 3 \textcolor{blue}{4} \textcolor{green}{5} \textcolor{blue}{6} 7 \textcolor{blue}{8} 9 \textcolor{blue}{8}
\]
\fi
}

\frame{\frametitle{How to get $\phi(n)$?}
\begin{itemize}
\item We have \[ n = \sum_{d|n} \phi(d),
\]
a fact about values of $\phi$ added up. But we want individual values of $\phi$. So we use M\"obius inversion!
\item We follow the formula. Our poset is $\{1,\dots,10\}$, but using $|$ (divisibility) instead of the usual $\leq$ (see it in the sum).
\[ \boldsymbol{\phi} = [\phi(1),\dots,\phi(10)]\hspace{1cm} \mathbf{f} = [1,2,\dots,10]
\]
\item We need $\zeta$ such that \[\zeta_{ij} = 
\begin{cases}
1 & i|j\\
0 & else
\end{cases}
\]
\end{itemize}}

\frame{\frametitle{First $\zeta$}
\[
\zeta = \bordermatrix{
  & 1 & 2 & 3 & 4 & 5 & 6 & 7 & 8 & 9 & 10 \cr
1 & 1 & 1 & 1 & 1 & 1 & 1 & 1 & 1 & 1 & 1  \cr
2 & 0 & 1 & 0 & 1 & 0 & 1 & 0 & 1 & 0 & 1  \cr
3 & 0 & 0 & 1 & 0 & 0 & 1 & 0 & 0 & 1 & 0  \cr
4 & 0 & 0 & 0 & 1 & 0 & 0 & 0 & 1 & 0 & 0  \cr
5 & 0 & 0 & 0 & 0 & 1 & 0 & 0 & 0 & 0 & 1  \cr
6 & 0 & 0 & 0 & 0 & 0 & 1 & 0 & 0 & 0 & 0  \cr
7 & 0 & 0 & 0 & 0 & 0 & 0 & 1 & 0 & 0 & 0  \cr
8 & 0 & 0 & 0 & 0 & 0 & 0 & 0 & 1 & 0 & 0  \cr
9 & 0 & 0 & 0 & 0 & 0 & 0 & 0 & 0 & 1 & 0  \cr
10& 0 & 0 & 0 & 0 & 0 & 0 & 0 & 0 & 0 & 1  \cr
}\]
}
\frame{\frametitle{Then $\mu$}
\[
\mu = \zeta^{-1} = \bordermatrix{ 
  & 1 & 2 & 3 & 4 & 5 & 6 & 7 & 8 & 9 & 10 \cr
1 & 1 & -1& -1& 0 & -1& 1 & -1& 0 & 0 & 1  \cr
2 & 0 & 1 & 0 & -1& 0 &-1 & 0 & 0 & 0 & -1 \cr
3 & 0 & 0 & 1 & 0 & 0 & -1& 0 & 0 & -1& 0  \cr
4 & 0 & 0 & 0 & 1 & 0 & 0 & 0 & -1& 0 & 0  \cr
5 & 0 & 0 & 0 & 0 & 1 & 0 & 0 & 0 & 0 & -1 \cr
6 & 0 & 0 & 0 & 0 & 0 & 1 & 0 & 0 & 0 & 0  \cr
7 & 0 & 0 & 0 & 0 & 0 & 0 & 1 & 0 & 0 & 0  \cr
8 & 0 & 0 & 0 & 0 & 0 & 0 & 0 & 1 & 0 & 0  \cr
9 & 0 & 0 & 0 & 0 & 0 & 0 & 0 & 0 & 1 & 0  \cr
10& 0 & 0 & 0 & 0 & 0 & 0 & 0 & 0 & 0 & 1  \cr
}
\]

}
\frame{\frametitle{Putting it together}
\begin{itemize}
\item We apply M\"obius inversion: We know that $\mathbf{f} = \boldsymbol{\phi}\zeta$, so it must be that $\boldsymbol{\phi} = \mathbf{f}\mu$. Recall that another way to write this is that 
\[
\phi(n) = \sum_{m|n}m\mu(m,n) \hspace{1cm} \text{(Here $f(m) = m$ so we don't write it)}
\]
\item It turns out that \[
\mu(a,b) = \begin{cases}
0 & a \nmid b \text{ or } \frac ab \text{ is divisible by a square}\\
(-1)^{\text{\#prime factors of }\frac ab} & else
\end{cases}
\]
\item We can now write down values of $\phi(n)$. For example:
\begin{align*}
\phi(10)& = 1\mu(1,10) + 2\mu(2,10) + 5\mu(5,10) + 10\mu(10,10)\\
&= 1(1) + 2(-1) + 5(-1) + 10(1) = 4 \hspace{.5cm}(\textcolor{red}{1} 2 \textcolor{red}{3} 4 5 6 \textcolor{red}{7} 8 \textcolor{red}{9})
\end{align*}
\end{itemize}
}
\section{Clustering}
\frame{\frametitle{Other Applications}
\begin{itemize}
\item M\"obius inversion has many applications involving counting. Some I have been looking at:
\begin{itemize}
\item Inclusion-Exclusion principle
\item Counting orbits of groups
\end{itemize}
\item The reason I first started looking at M\"obius inversion is because of its somewhat surprising (to me) application in data clustering.
\end{itemize}
}
\frame{\frametitle{Hierarchical Clustering}
Hierarchical clustering is a general approach to data analysis
\begin{itemize}
\item Points represent data 
\item Idea: Capture structure at various scales
\end{itemize}
\vspace{1cm}
\[
\includegraphics[width = 0.5\paperwidth] {../Figures/multiscale_structure.png}
\]
}
\frame{\frametitle{A Strange Approach to Clustering}
\begin{itemize}
\item We want to count the number of clusters in a space. We'll do it in a weird way:
\item[] \begin{itemize}
\item Assign a metric to our space, i.e. a distance function $d$
\item Assign a \textbf{weighting} to the points, i.e. assign each point a real-valued "weight"
\item We want each cluster's weight to sum to close to 1
\item Points near many other points will have smaller values
\item Points which are separated will have larger values
\end{itemize}
\item Then, we'll sum the weights to get the \textbf{magnitude}, i.e. number of clusters, in the data set.
\item Note: There is no "correct" or "actual" number of clusters in a space. This will be scale-dependent.
\end{itemize}
}
\frame{\frametitle{Weighting and Magnitude}
\begin{itemize}
\item
A \textit{metric space} is a set of points $X$ together with a function $d:X \times X \rightarrow \Rr^{\geq 0}$ capturing a notion of distance.
\item A \textbf{weighting} on a finite metric space $X$ is a set of weights $\omega(x)$ in $\Rr$ such that, for all $x \in X$:
\[
\sum_{y \in X}\omega(y)e^{-d(x,y)} = 1
\]
\item Then the \textbf{magnitude} of $X$, denoted $|X|$, is defined by:
\[
|X| = \sum_{x \in X}\omega(x).
\]
\item This looks kind of like a setup for M\"obius inversion. But we will need $\zeta$ which is not upper triangular and just ones and zeros.
\end{itemize}
}
\frame{\frametitle{Writing Down Weightings}
We had
\[
\text{For all } x \in X, \sum_{y \in X}\omega(y)e^{-d(x,y)} = 1
\]
We write $X = \{x_1,...,x_n\}$. Then we 
\begin{itemize}
\item let $\zeta$ be the $n\times n$ matrix where $\zeta_{i,j} = e^{-d(x_i, x_j)}$, 
\item let $\boldsymbol{\omega} = [\omega(x_1), \dots, \omega(x_n)]$, and
\item let $\bf{e}$ be the $1 \times n$ vector of ones.
\end{itemize}
 So the above becomes the familiar
\[
\mathbf{e} = \boldsymbol{\omega} \zeta
\]
}
\frame{\frametitle{Writing Down Weightings}
\begin{itemize}
\item We want to let $\mu = \zeta^{-1}$. Now $\zeta$ may not be invertible. But if it is, let $\mu = \zeta^{-1}$, so that
\[
\mathbf{e} = \boldsymbol{\omega} \zeta \Rightarrow \boldsymbol{\omega} = \mathbf{e}\mu
\]
\item Since $\mathbf{e}$ is the all-ones vector, this means that the weights are sums of rows of $\mu$:
\[
\text{For all } x \in X,\omega(x) = \sum_{y \in X}\mu(x,y)
\]
Recall that the magnitude of $X$, or $|X|$, which was the quantity we wanted, was the sum of the weights. So $|X|$ is simply the sum of the entries of $\mu$:
\[
|X| = \sum_{x,y \in X}\mu(x,y)
\]
\end{itemize}
}
\section{}
\frame{\frametitle{Sources}
\tiny
\nocite{*}
\bibliographystyle{abbrv}
\bibliography{../Drafts/bibliography.bib}
}
\frame{
\huge
\begin{center}Thank you! \end{center}
}

\end{document}

