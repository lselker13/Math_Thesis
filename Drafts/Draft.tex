\documentclass[12pt]{pom_thesis}
\input{commands_packages}
\usetikzlibrary{decorations.pathreplacing,angles,quotes}
\author{Leo Selker}
\advisor{Vin de Silva}
\title{The Magnitude of a Metric Space: An Approach to Clustering}

\newcommand{\fix}{\text{fix }}
\newcommand{\cl}{\text{cl }}
\DeclareMathOperator{\cat}{cat}
\begin{document}

\maketitle

\begin{abstract}Clustering is an important application of topology. Here we explore a method of counting clusters at various scales. We build the necessary framework using the idea of a M\"obius inversion, and then we define the magnitude of a metric space\cite{Lein2}. We investigate behaviors of positive definite metric spaces, as well as how the magnitude acts under the product of spaces.
\end{abstract}

\pagenumbering{roman}
\tableofcontents

\newpage
\pagenumbering{arabic}
\begin{chapter}{Outline}
\section{Introduction}
We will be investigating the magnitude of metric spaces along with related/foundational ideas. We investigate M\"obius inversion, a way of calculating the magnitude of a metric space. We motivate and define magnitude of a metric space using the Euler characteristic of a category. We investigate positive definite metric spaces and tensor products. It's still slightly early to say what our "goal" is.
\section{M\"obius Inversion}
We will start with Mobius inversion on the positive integers, generalize to posets, and again generalize to categories\cite{Lein1}. At each step we will justify the way we choose to generalize.

Here we may also investigate the P\'olya Counting Formula/Burnside's Lemma. We're just starting on that, so we'll see where it goes.
\section{Euler Characteristic of a Category}
We define the Euler characteristic of a finite category as the sum of all values of its Mobius inversion\cite{Lein1}. We will justify this by showing that simplicial complexes behave well under this definition, and it coincides with the usual one in that case. We'll also investigate working the other way: Taking a poset and turning it into a simplicial complex of chains. Hopefully this will lead somewhere too.
\section{Magnitude, Properties, and Examples}
We will define the "induced category" of a metric space, and use it as a motivation for the definition of the \textbf{magnitude} of a finite metric space\cite{Lein2} (the magnitude is the Euler characteristic of the induces category). This definition captures the "effective number of points" in the space in some sense. As we scale the space smaller, the magnitude approaches 1, and as we scale it larger it approaches the number of points in the space. In this sense, we have a "backwards" scale factor, where to zoom in we increase the factor. 
In some ways magnitudes over scales approximate hierarchical clustering results. In fact, the magnitude of the tensor product of two finite spaces is the product of the spaces' magnitudes (we proved this, but it is likely known). We can also say more in the case of positive definite metric spaces\cite{Meck1}. We will investigate approximation theorems for "almost-product" spaces, and other situations which are well behaved.

Here we will also probably discuss results from code I am writing.
\section{Magnitude and Persistence}
Hierarchical clustering is a way of expressing structure at a variety of levels within data. We can keep track of that structure using a \textit{persistent set}\cite{Carl1}. We can define clusters however we want using some scale factor. Generally we use single linkage, i.e. we partition under the equivalence relation define by whether there's a path between points involving no jumps of too long a distance. 

We will attempt to connect the magnitude of a space with its persistent set. Both offer scaled counts of the number of clusters in some sense - we have some basic connections but hope to find more.
\end{chapter}
\begin{chapter}{Introduction to M\"obius Inversion}
\section{Initial Example}
M\"obius inversion is a general construct. The most basic definition of a M\"obius function is a function $\mu$ on the positive integers which satisfies the following for functions $f,g$ on the positive integers:

\begin{equation}\label{init_mob} %TODO: Unify numbering
g(n) = \sum_{d | n} f(d) \iff f(n) = \sum_{d | n} \mu\left(\frac dn\right) g(n)
\end{equation}


We will formally develop this using a more general construction, and then generalize to more even more abstract structures. We start with several definitions.
\section{Posets and Incidence Algebras}
\begin{defn}
A \emph{partially ordered set}, often called a \emph{poset}, is a set $X$ together with a binary relation $\leq$, such that, for all $a,b,c \in X$:
\begin{itemize}
\item $a \leq a$ (reflexive)
\item If $a \leq b$ and $b \leq a$ then $a = b$ (antisymmetric)
\item If $a \leq b$ and $b \leq c$ then $a \leq c$ (transitive).
\end{itemize}
\end{defn}
\begin{defn}
A poset $X$ is \emph{locally finite} if, for any $a,b \in X$, the set $\{c \in X : a \leq c \leq b\}$ is finite.
\end{defn}
\begin{defn}
The \emph{incidence algebra} on a locally finite poset $X$ is over the set of functions from $X \times X$ to $\Rr$ which map $(a,b)$ to zero if $a \nleq b$. So the elements of the incidence algebra are 
\[\{f:X \times X \rightarrow \Rr:a \nleq b \Rightarrow f(a,b) = 0\}
\]
Scalar multiplication and addition are defined pointwise.  Multiplication is denoted by ``*'', and is defined:
\[(f* g)(a,b) = \sum_{a \in X}f(a,c)g(c,b) = \sum_{a \leq c \leq b}f(a,c)g(c,b).\]
\end{defn}
\begin{rmk}
The requirement that $X$ be locally finite ensures that the above sum has a finite number of terms.
\end{rmk}
\begin{rmk}
We use $\Rr$ as the codomain of the functions in the incidence algebra. But this can be generalized to other commutative rings with identity.
\end{rmk}
I include the version with bounds for clarity, but they make no difference, because all other terms will go to zero anyway by the condition on elements. Considering the sum without th bounds, we see that this is identical to matrix multiplication. It will be useful to formalize this:
\begin{lemma}\label{mat_eq}
Let $X$ be a finite poset, with $|X|=n$. Choose a total ordering on $X$ such that the partial order is respected - in other words, label the elements of $X$ from $x_1$ to $x_n$, such that if $x_i \leq x_j, i \leq j$. Let $f,g$ be elements of the incidence algebra on $X$. Let $F$ be the $n\times n$ square matrix where $F_{ij}=f(x_i, x_j)$. Define $G$ likewise. Then:
\begin{enumerate}[a)]
\item $F$ and $G$ are upper triangular; and
\item For any $i,j, (f * g)(x_i, x_j)=FG_{ij}$.
\end{enumerate}
\end{lemma}
\begin{proof}
\begin{enumerate}[a)]
\item If $i>j$, by how we defined $F$, $x_i \nleq x_j$. So $F_{ij} = f(x_i, x_j) = 0$. 
\item
The result follows directly from the definition of matrix multiplication. Let $i,j\leq n$ be given. Remember that we can omit the bounds from the sum in the definition of incidence algebra multiplication. Then:
\begin{align*}
(f * g)(x_i, x_j) &= \sum_{y \in X} f(x_i,y)g(y,x_j)\\
&=FG_{ij}
\end{align*}
\end{enumerate}
\end{proof}


There are several important functions, present in any incidence algebra, which we will need. 
The \textbf{identity function} $\delta$ is defined by 
\[\delta(a,b) = \begin{cases} 1 & a = b \\ 0 & else. \end{cases}
\]
This is equivalent to the identity matrix, with 1's along the main diagonal.

We also define another special function, $\zeta$ (the ``indicator function''):
\[
\zeta(a,b) = \begin{cases} 1 & a \leq b \\ 0 & else. \end{cases}
\]

At this point, \ref{mat_eq} tells us several things about inverses. Since $\delta$ is equivalent to the identity matrix, inverses will coincide with the corresponding matrix inverses. This leads us to the following:
\begin{cor}\label{cor_inv}
Let $f,g$ be elements of the incidence algebra on $X$. Then:
\begin{enumerate}[a)]
\item $f*g = \delta \Rightarrow g*f = \delta$
\item If $\forall x \in X, f(x,x) \neq 0$, then $f$ is uniquely invertible - in other words, $\exists! f^{-1}$ such that $f*f^{-1} = \delta = f^{-1} * f$.
\end{enumerate}
\end{cor}
\begin{proof}
Part a) is true of all square matrices. Part b) is true of all upper triangular matrices, since the determinant of such matrices is the product of the main diagonal elements. We use \ref{mat_eq} to transfer the results to $f$ and $g$.
\end{proof}
We will refer to this unique inverse, if it exists, as $f^{-1}$. 

\section{Definition of M\"obius Inversion}
We are now ready to define the M\"obius function on partially ordered sets.
\begin{defn}\label{def_mob_pos}
Let $X$ be a finite poset. Let $\zeta$ be the indicator function defined above. Let $\mu = \zeta^{-1}$. Then $\mu$ is the \emph{M\"obius function} on $X$.
\end{defn}
\begin{rmk}
This inverse exists by \ref{cor_inv} part b. 
\end{rmk}
\begin{rmk}
Another statement of the definition is that $\mu$ is the unique function such that, for all $x,y \in X$,
\[
\sum_{z \in X} \mu(x,z)\zeta(z,y) = \delta(x,z) = \sum_{z \in X} \zeta(x,z)\mu(z,y)
\]
\end{rmk}
\section{Integers with Divisibility Ordering}
We are now able to justify the existence of, and find, the function $\mu$ from equation \ref{init_mob}. It is, of course, a type of M\"obius function. But the form in equation \ref{init_mob} uses functions of a single variable, while our construction uses functions of two variables. We must do some simple conversions to make the definitions line up.

Let $X$ be the poset consisting of the positive integers, using $|$, or divisibility, as the ordering relation. Let $\zeta$, and therefore $\mu = \zeta^{-1}$, be defined as usual in the associated incidence algebra.
%TODO: Finish this section (include intuition/direct proof)
\end{chapter}
% % %
\begin{defn}
A category $\Aa$ has \emph{Mobius Inversion} $\mu$ if $\mu$ satisfies
\[
\sum_b \mu(a,b)\zeta(b,c) = \delta(a,c) = \sum_b \zeta(a,b)\mu(b,c),
\]
with each equality implying the other by finite dimensionality (this is the same argument as showing that one-sided square matrix inverses are two-sided).
\end{defn}
\begin{chapter}{Groups and Orbits} 
We want to apply M\"obius inversion to groups. So far, the most general construct to which we have applied M\"obius inversion is the partially ordered set (poset). Given a group $G$, 
Suppose that a group $G$ acts on a set $\Omega$. 
\end{chapter}

\begin{chapter}{Metric Space Magnitude}
\section{Motivation: Clustering}
The \emph{magnitude} of a metric space is a notion introduced in \cite{Lein2}. %TODO: Finish introduction
\section{Important Definitions} \label{mag_defs}
A metric space is a space with a notion of distance between pairs of points. 
\begin{defn}
A \textit{metric space} $(M,d)$ is a set $M$ together with a function  $d:M \times M \rightarrow \Rr$ (the \emph{metric}), such that the following are satisfied for all $a,b,c \in M$:
\begin{itemize}
\item $d(a,b) \geq 0$, with equality $\iff a = b$;
\item $d(a,b) = d(b,a)$; and
\item $d(a,b) + d(b,c) \geq d(a,c)$.
\end{itemize}
We sometimes write just $M$ if the metric is unambiguous.
\end{defn}
A category is an algebraic structure which generalizes ``objects" and ``maps" (or ``morphisms"), with composition. There are several equivalent definitions, all somewhat technical, but the construct itself is quite intuitive once you get used to it.  
\begin{defn}
A \textit{category} $C$ consists of:
\begin{itemize}
\item A collection $C_0$ of \emph{objects};
\item For each $a,b \in C_0$, a collection $C(a,b)$ of \emph{morphisms from $x$ to $y$};
\item For each $a,b,c \in C_0$, a \emph{composition operator}: A function $\circ:C(a,b) \times C(b,c) \rightarrow C(a,c)$; and
\item For each $a \in C_0$, an \emph{identity morphism} $1_a \in C(a,a)$,
\end{itemize}
such that the following axioms are satisfied:
\begin{itemize}
\item Composition is associative: If $a,b,c,d \in C_0$ and $f\in C(a,b), g \in C(b,c), h \in C(c,d)$, then $f \circ (g \circ h) = (f \circ g) \circ h$, and
\item Composition respects identities: If $a,b \in C_0, f \in C(a,b)$, then $1_b \circ f = f = f \circ 1_a$.
\end{itemize}
\end{defn}
This is a definition of what is usually called a category, but here we will refer to it as an \emph{ordinary} category. This is because it will also be useful to mention a generalization of categories, called \emph{enriched categories}. The idea is that we replace the sets of morphisms with objects from an ordinary category of a certain type, and adapt the definition of composition accordingly. 

It's not necessary to fully understand how generalized metric spaces are categories - every space we consider here will be a bona fide metric space. Furthermore, the definitions and constructions we explore will be easy to apply directly to metric spaces. I point out that this construction is a type of category to justify dealing with them together with other categories in the next section. We will define the Euler characteristic of a category, and we will need to apply the definition to metric spaces by way of generalized metric spaces. This will make much more sense if we keep in mind that generalized metric spaces are categories.

The full definition of enriched categories is complicated, technical, and not necessary here. For a full discussion of enriched categories, see \cite{Kelly1}. We will instead directly define a particular enriched category object called a \emph{generalized metric space}.
\begin{defn}
A \emph{generalized metric space} $C$ consists of:
\begin{itemize}
\item A collection $M$ of points (or objects); and
\item For each $a,b \in M$, a non-negative real number $d(a,b)$, such
\end{itemize}
such that, for all $a,b \in M$,
\[C(a,a) = 0,\hspace{1cm} C(a,b) + C(b,c) \geq C(a,c).
\]
\end{defn}
Note that this is essentially the same as a metric space, but without the restriction that distinct points have nonzero distance, and without the requirement of symmetry. Relaxing these constraints allows us to cleanly express this as an enriched category. In keeping with the intuitive description of enriched categories which I gave earlier, this is a category where sets of morphisms are replaced by non-negative real numbers. The two axioms are enforced by details in the definition of an enriched category which I swept under the rug.

For those who are interested in the details, what we are doing is letting $\mathcal{V}$ be the monoidal category whose set of objects is $\Rr^{\geq 0}$, with a single morphism from $a$ to $b$ if and only if $a \leq b$, and with addition (+) as the monoid operator. Then a generalized metric space is a category enriched with $\mathcal{V}$. 

We will soon be interested in a notion of how close together things are - and it will need to measure closeness rather than distance, in the sense that it will increase for objects which are closer together. The reason for this will become clear soon. We will call this binary function $\zeta$, and define it on a category $C$ as follows: For $a,b \in C_0$, if $C$ is an ordinary category, then $\zeta(a,b) = |C(a,b)|$, or the size of the set of morphisms from $a$ to $b$. If $C$ is a generalized metric space, then $\zeta(a,b) = e^{-C(a,b)}$. 

The definition for ordinary categories is fairly natural. The version for generalized metric spaces is less intuitive, but the definitions are surprisingly consistent. Consider three objects $a,b,c$ in an ordinary category. The numbers above the arrows indicate the number of morphisms between adjacent objects:

\[
\begin{tikzpicture}[auto, node distance = 3cm, main node/.style={dot}]

\node[label = below:{a}, circle, draw, fill=black,
                        inner sep=0pt, minimum width=4pt](1) at (0,0) {};
\node[label = below:{b}, circle, draw, fill=black,
                        inner sep=0pt, minimum width=4pt](2) at (3,0) {};
\node[label = below:{c}, circle, draw, fill=black,
                        inner sep=0pt, minimum width=4pt](3) at (6,0) {};                        

\draw[-latex] (1) -- (2) node[midway, above = 3pt] {\textit{k}};
\draw[-latex] (2) -- (3) node[midway, above = 3pt] {\textit{l}};

\end{tikzpicture}\]
Then, as a result of the composition axioms, there must be a morphism $g \circ f$ from $a$ to $c$ for each pair of morphisms $f \in C(a,b), g \in C(b,c)$. This means that there are $kl$ morphisms from $a$ to $c$ which are implied by the morphisms shown above. So $\zeta(a,c) \geq \zeta(a,b)  \zeta(b,c)$. 

Now, consider the same situation in a generalized metric space. We know from one of the axioms that
\[
C(a,c) \leq C(a,b) + C(b,c).
\]
The inverse exponential function is monotonically decreasing. So we simply apply it to both sides and flip the inequality to get the same result, that $\zeta(a,c) \geq \zeta(a,b) \zeta(b,c)$.

There is another way of looking at essentially the same property, which might make it slightly more clear why this works out. The reason the addition (+) operator appears in the second axiom in the generalized metric space definition is because addition is what is called the \textit{monoidal product} in the enriched category. The equivalent in an ordinary category is the tensor product, $\otimes$. In terms of number of morphisms, $\otimes$ behaves like multiplication. So it makes sense that we exponentiate in the generalized metric space case. Adding exponents becomes multiplication.
\section{M\"obius Inversion on a Category}
To define M\"obius inversion on a partially ordered set, we used an "indicator function" which was - not coincidentally - notated as $\zeta$. We will henceforth refer to this function as $\zeta'$, reserving $\zeta$ for the function we just defined on categories. Recall that for $a,b$ in a poset $X$, $\zeta'(a,b)$ was defined to be 1 if $a \leq b$, and 0 otherwise. We would like to generalize the construction of M\"obius inversion. The first step is to observe the following: Given a poset $X$, we can construct an "equivalent" category by letting $C_0=X$, and adding a single morphism from elements $a,b \in C_0$ if and only if $a \leq b \in X$. It is easily checked that this is in fact a category, and that for each triplet in $C_0$ there is only one possible composition operator (this is why we don't bother to explicitly define them). 

The key fact here is that for $a,b \in C_0 = X, \zeta(a,b) = \zeta'(a,b)$. This motivates the following definition:
\begin{defn}\label{cat_mob} %TODO: Check uniqueness
Let $C$ be a category. Let $\zeta$ be as defined above. Then the \emph{M\"obius function} on $C$, if it exists, is a function $\mu$ which, for all $x,y \in C$, satisfies
\[
\sum_{z \in X} \mu(x,z)\zeta(z,y) = \delta(x,z) = \sum_{z \in X} \zeta(x,z)\mu(z,y).
\]
If such a $\mu$ exists, $C$ is said to \emph{have M\"obius inversion}.
\end{defn}
\section{Euler Characteristic of a Category}
\subsection{Definition}
We are now ready to construct central definition of this section: the Euler characteristic of a category. The entire construction can be found in \cite{Lein1}, \cite{Lein2}, \cite{Lein4}. We will immediately connect this to M\"obius inversion. This will lead to the initial definition of the Euler characteristic of a category, which in turn will directly lead to the definition of the magnitude of a metric space.

Throughout this section, objects declared as categories are implicitly allowed to be either ordinary categories or generalized metric spaces. The definition of $\zeta$ from section \ref{mag_defs} will be assumed.

\begin{defn}
A \emph{weighting} on a category $\Aa$ is a function $w_a$ on $\ob(\Aa)$ such that for all $a \in \Aa$,
\[
\sum_b \zeta(a,b)\omega_b = 1.
\]
\end{defn}
\begin{lemma}
\label{mobius_is_weighting}
If $\Aa$ has M\"obius inversion, then $\omega_a = \sum_b \mu(a,b)$ is a weighting on $\Aa$.
\end{lemma}
\begin{proof} For all $a \in \Aa$,
\begin{align*}
\sum_b \zeta(a,b)\omega_b &= \sum_b\left( \zeta(a,b)\sum_c \mu(b,c)\right)\\
&=\sum_b\sum_c\zeta(a,b)\mu(b,c)\\
&=\sum_c\left(\sum_b\zeta(a,b)\mu(b,c)\right)\\
&=\sum_c \delta(a,c)\\
&=1.
\end{align*}
\end{proof}

The following definition is due to \cite{Lein1}.
\begin{defn}
Let $\omega$ be a weighting on a finite category $A$. Then the \emph{Euler characteristic} of $A$, denoted (as usual) as $\chi(A)$, is defined by
\[
\chi(A) = \sum_{a \in A} \omega_a
\]
\end{defn}
Note that even if a category admits several weightings, they all lead to the same Euler characteristic by a simple sum rearrangement argument. Together with \ref{mobius_is_weighting}, this implies the following:
\begin{cor}
If a finite category $A$ has a M\"obius inversion $\mu$, then
\[\chi(A) = \sum_{a \subseteq A} \sum_{b \subseteq A} \mu(a, b).
\]
\end{cor}
This is important. It provides a way in which we can write down weightings on metric spaces, since the similarity matrix $\zeta$ is invertible in many cases. We will investigate exactly which cases when we discuss positive semidefinite matrices.
\subsection{Simplicial Complexes}
This definition needs some intuitive justification. We will use a new construction, the simplicial complex, to show that calling this value an Euler characteristic is reasonable. 

\begin{defn}
Let $\sigma$ be a simplicial complex. Then the \emph{category generated by $\sigma$}, or $\cat(\sigma)$, is the category whose objects are the cells of $\sigma$. If $\gamma$ and $\omega$ are cells of $\sigma$, then $|(\gamma, \omega)| = 1$ if $\gamma \subseteq \omega$ and 0 else.
\end{defn}
\begin{lemma}
\label{mu_lemma}
Suppose $\alpha \subseteq \beta$ are cells of a simplicial complex. Then $\mu(\cat(\alpha), \cat(\beta)) =(-1)^{|\beta - \alpha|} = (-1)^{|\alpha|} (-1)^{|\beta|}$.
\end{lemma}
\begin{proof}
We use induction. Base case: $\sigma = \beta$, \checkmark \\
For any simplex $\sigma$, let $\sigma'$ be $\sigma \cup \emptyset$. Suppose that $|\beta - \alpha| = k+1$. We know that \begin{align*}
\mu(\alpha, \beta) &=  - \sum_{\alpha \subseteq \sigma \subset \beta} \mu(\alpha, \sigma)\\
&= - \sum_{\gamma \subset (\beta - \alpha)'} \mu(\alpha, \gamma \cup \alpha) 
\end{align*}
Note that the above unions are always disjoint. So, by the inductive hypothesis, we can rewrite it:
\begin{align*}
\mu(\alpha, \beta) &= - \sum_{\gamma \subset (\beta - \alpha)'} (-1)^{|\gamma \cup \alpha| - |\alpha|} \\ %& \text{inductive hypothesis}\\
&= - \sum_{\gamma \subset (\beta - \alpha)'} (-1)^{|\gamma|}\\ % &\text{by disjointness}\\
&= - \sum_{\gamma \subseteq (\beta - \alpha)} (-1)^{|\gamma|} + (-1)^{|\beta - \alpha|} - (-1)^0 \\ %&\text{Removing the empty simplex and including $\gamma = \beta - \alpha$}\\
&= \chi(\beta - \alpha) + (-1)^{|\beta - \alpha|} - (-1)^0\\
&= 1-1+ (-1)^{|\beta - \alpha|}\\
&= (-1)^{|\beta - \alpha|}
\end{align*}


\end{proof}
\begin{thm}
\label{consistentEuler}
Let $\sigma$ be a simplicial complex. Then 
\[
\sum_{\gamma \subseteq \sigma} \sum_{\alpha \subseteq \gamma} \mu(\alpha, \gamma) = \chi(\sigma).
\]

\end{thm}
\begin{proof}

\begin{align*}
\sum_{\gamma \in \sigma} \sum_{\alpha \subseteq \gamma} \mu(\alpha, \gamma) &= \sum_{\gamma \in \sigma} \sum_{\emptyset \subset \alpha \subseteq \gamma} \mu(\alpha, \gamma)\\
&= \sum_{\gamma \in \sigma} \mu(\emptyset, \gamma)\\
&= \sum_{\gamma \in \sigma} (-1)^{|\gamma|} &\text{by \ref{mu_lemma}}\\
&= \chi(\sigma)
\end{align*}
\end{proof}
\section{Metric Space Magnitude}

\end{chapter}
\bibliographystyle{abbrv}
\bibliography{bibliography.bib}


\end{document}

