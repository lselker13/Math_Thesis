\documentclass[12pt]{pom_thesis}
\usepackage[T1]{fontenc}
\input{commands_packages}
\usetikzlibrary{decorations.pathreplacing,angles,quotes}
\usepackage{mathrsfs}
\usepackage[euler-digits]{eulervm} 
\usepackage[font={small}]{caption}
\usepackage{float}
\author{Leo Selker}
\advisor{Vin de Silva}
\title{M\"obius Inversion: From Posets to Categories}
\newcommand{\kron}{\odot}
\DeclareMathOperator{\fix}{Fix}
\DeclareMathOperator{\cat}{cat}
\DeclareMathOperator{\ps}{\mathscr{P}}
\DeclareMathOperator{\sub}{Sub}
\DeclareMathOperator{\stab}{Stab}
\DeclareMathOperator{\orb}{\mathcal{O}}
\DeclareMathOperator{\fixx}{fix}
\DeclareMathOperator{\obj}{ob}
\DeclareMathOperator{\im}{Im}
\newcommand{\catname}[1]{\textbf{\textsc{#1}}}

%\theoremstyle{plain}{
%\newtheorem{thm}{Theorem}[chapter]
%\newtheorem{lemma}[thm]{Lemma}
%\newtheorem{prop}[thm]{Proposition}
%\newtheorem{cor}[thm]{Corollary}
%\newtheorem{conj}[thm]{Conjecture}
%}

%\theoremstyle{definition}{
%\newtheorem{defn}[thm]{Definition}
%\newtheorem{examp}[thm]{Example}
%\newtheorem{rmk}[thm]{Remark}
%\newtheorem{fact}[thm]{Fact}
%}
\begin{document}

\maketitle

\begin{abstract}  M\"obius inversion is a general approach for reversing certain kinds of sums. We start by defining M\"obius inversion in its usual form, over a partially ordered set. We explore traditional applications in the integer divisibility lattice, including Euler's Totient. We then use the lattice of subgroups to count orbits under a group action. In the last two sections, we explore Tom Leinster's generalization of M\"obius inversion to categories and enriched categories, and investigate a data clustering algorithm.
\end{abstract}

\pagenumbering{roman}
\tableofcontents

\newpage
\pagenumbering{arabic}
\begin{chapter}{Introduction to M\"obius Inversion}\label{chap_intro}
\section{Introduction and Motivation}
%TODO: Expand introduction with more explicit examples
We motivate M\"obius inversion using several examples.
\begin{examp}
Suppose that we have the integer sequence $g$ of ones:
\[
g : 1, 1, 1, \cdots
\]
We can let $f$ be the sequence of cumulative sums of $g$. So $f_k = g_1 + \cdots + g_k$. We can write down the values:
\[
f: 1,2,3,\cdots
\]
In general, summing up values is easy. Finding values of $f$ from $g$ was not difficult. 


If we knew $f$ and wanted to find values of $g$, we would subtract adjacent values of $f$. So we would use the fact that $g_k = f_k - f_{k-1}$. It seems clear that, when we are talking about integer sequences, if for all integers $k$,
\[
\forall k f_k = \sum_{i = 1}^k g_i,
\]
then it is also true that
\[\\
g_k = f_k = f_{k-1}.
\]
\end{examp}

This relationship is a result of the simple structure of the positive integers. However, it is possible to sum up values over subsets of more complicated structures, such that this relationship isn't so obvious. For example, we could add up values of $g$ over a number's divisors, instead of over all numbers less than or equal to it.
\begin{examp}
Let $h, g$ be functions on the positive integers defined as follows:
\[
h_k = \sum_{l|k}g_l.
\] 
It is now much less clear how to obtain $g$ from $h$. In fact, we will see how to reverse this kind of sum soon.
\end{examp}

The goal of M\"obius inversion is to generalize the process of reversing sums over discrete, ordered structures called \emph{posets}. If sums like the above are a discrete analogue of integration (cumulatively summing over a region), then M\"obius inversion is discrete differentiation. In the same way that developing the theory of the derivative is useful for solving differential equations where functions are defined in terms of their integral, M\"obius inversion helps us extract values which are defined in terms of their cumulative sums. In addition to building the necessary machinery for M\"obius inversion, we will consider several applications of it, including a somewhat surprising one in data clustering.
\section{Initial Example and Partially Ordered Sets}
M\"obius inversion is a general construct. The classical definition of a M\"obius function is a function $\mu$ on the positive integers which satisfies the following for functions $f,g$ on the positive integers:
\begin{equation}\label{init_mob} 
f(n) = \sum_{d | n} g(d) \iff g(n) = \sum_{d | n}  f(d)\mu\left(\frac dn\right)
\end{equation}

Our first goal is to prove the existence of, and find, this function $\mu$. We will do this using by developing M\"obius inversion in general and then applying it. We start with several definitions.

\begin{defn}
A \emph{partially ordered set}, often called a \emph{poset}, is a set $X$ together with a binary relation $\leq$, such that, for all $a,b,c \in X$:
\begin{itemize}
\item $a \leq a$ (reflexive)
\item If $a \leq b$ and $b \leq a$ then $a = b$ (antisymmetric)
\item If $a \leq b$ and $b \leq c$ then $a \leq c$ (transitive).
\end{itemize}
\end{defn}
Note that, for completeness, a poset should be labeled with both its elements and its relations - e.g. $(X, \leq)$. But often we just write $X$ when the relation is implied.
\begin{rmk}
Given $x_1, x_2$ in a poset $X$, we write $[x_1, x_2]$ to denote the set 
\[
\{y \in X: x_1 \leq y \leq x_2\}
\]
Replacing one of the square brackets with a parenthesis means we are omitting that element. So
\begin{align*}
[x_1, x_2) &= \{y \in X: x_1 \leq y < x_2\},\\
(x_1, x_2] &= \{y \in X: x_1 < y \leq x_2\}.
\end{align*}
Note that if $x_1 \nleq x_2, [x_1, x_2] = [x_1, x_2) = (x_1, x_2]= \emptyset.$
\end{rmk}
\begin{defn}
A poset $X$ is \emph{lower finite} if, for any $a \in x$, the set $\{ c \in X: c \leq a\}$ is finite.
\end{defn}
This condition will be necessary for many of the following facts and theorems, to ensure that certain sums have a finite number of nonzero terms. Note that in a lower finite poset, $[x_1, x_2]$ is finite for any $x_1, x_2$.

Next, we define an important function on lower finite posets: $\zeta$ (the ``indicator function''), from $X \times X$ to $\Rr$:
\[
\zeta(a,b) = \begin{cases} 1 & a \leq b \\ 0 & else. \end{cases}
\]

M\"obius inversion is concerned with adding up values of functions. In particular, given a lower finite poset $X$ and a function $g:X \rightarrow \Rr$, we want to consider a function $f:X \rightarrow \Rr$ defined by

\[
f(a) = \sum_{b \leq a}g(b)
\]

Note that, because $X$ is lower finite, this sum has a finite number of terms, so is always finite-valued. Using the ``indicator'' function $\zeta$ from above, we can rewrite the definition of $f$ to remove the need for the bound on the sum:
\begin{equation}\label{sum_form_intro}
f(a) = \sum_{b \in X}g(b)\zeta(b,a)
\end{equation}
Each value of $f$, therefore, takes on precisely the structure of a dot product. We will use this structure to leverage linear algebra: 
\begin{itemize}
\item Totally order $X$ such that the partial order is respected;
\item Write $g$ as a row vector which contains the values $g$ takes on elements of $X$, with respect to the total ordering;
\item Write $\zeta$ as a matrix whose $i,j$ entry is equal to $\zeta(a_i, a_j)$, where $a_i, a_j$ are the $i$'th and $j$'th elements of $X$ under the total ordering.
\end{itemize}

Going forward, we will not explicitly define the vector/matrix forms of functions in this way. Instead, when we refer to functions on one or two elements of a partially ordered set, we will implicitly think of them as (row) vectors and as matrices. We will also implicitly assume a total ordering, and identify an element $a$ of $X$ with its index in the total ordering. So if we write $g(a)$, we are referring to the function $g$ applied to $a$ and also to the element of the corresponding vector containing that value - they are equivalent. We will also use this perspective for function multiplication and inversion. Going forward, $g\zeta$ will refer to the vector/matrix product of $g$ and $\zeta$ rather than function composition, and $\zeta^{-1}$ will refer to the matrix inverse of $\zeta$ rather than the inverse function under composition.

We now are able to define our desired function (or vector) $f$ more succinctly:
\[
f = g \zeta
\]

If the matrix multiplication is explicitly written out, what we get is identical to \eqref{sum_form_intro}.

The goal of M\"obius inversion is to obtain $g$ given $f$. If $\zeta$ were invertible, we would be able to do that. It turns out that $\zeta$ always is invertible. Because the total ordering respects the partial ordering, the matrix $\zeta$ is upper triangular and has ones on the diagonal. This implies the following straightforward but crucial fact:
\begin{lemma}\label{inv_ex}
Let $X$ be a lower finite poset. Fix a total order on $X$ which respects the partial order, and let $\zeta$ be the matrix where 
\[
\zeta(a,b) = \begin{cases}1 & a \leq b \\ 0 & else. \end{cases}
\]
Then $\zeta$ is invertible, and elements of $\zeta^{-1}$ are integers.
\end{lemma}
\begin{proof}
We have already observed that $\zeta$ is upper triangular and has ones on the main diagonal. This means that $\det(\zeta) = 1$, meaning that $\zeta$ is invertible. Together with the fact that elements of $\zeta$ are integers, this also implies that elements of $\zeta^{-1}$ are integers.
\end{proof}

This means that we can let $\mu=\zeta^{-1}$, which we now know exists, and write down the central relationship of M\"obius inversion, which holds for any lower finite poset $X$ and pair of functions $f,g$ on it:
\[
f = g \zeta \iff g = f \mu.
\]

We will develop this further in the next section.

\section{Definition of M\"obius Inversion}
We are now ready to define the M\"obius function on partially ordered sets.
\begin{defn}\label{def_mob_pos}
Let $X$ be a lower finite poset. Let $\zeta$ be the indicator function defined above. Let $\mu = \zeta^{-1}$. Then $\mu$ is the \emph{M\"obius function} on $X$.
\end{defn}
\begin{rmk}
Recall that $\zeta^{-1}$ is the matrix inverse rather than the functional inverse, and that it exists by Lemma \ref{inv_ex}.
\end{rmk}
\begin{rmk}\label{sum_form}
We can unpack the linear algebra to get a definition analagous to \eqref{sum_form_intro}: We can say that $\mu$ is the unique function from $X\times X$ to $\Rr$ such that, for all $x,y \in X$,
\[
\sum_{z \in X} \mu(x,z)\zeta(z,y) = \delta(x,y) = \sum_{z \in X} \zeta(x,z)\mu(z,y),
\]
or
\[
\sum_{z \leq y} \mu(x,z) = \delta(x,y) = \sum_{z \geq x} \mu(z,y).
\]
\end{rmk}

To complete the definition of M\"obius inversion, we turn to the standard use of this function. 
\begin{thm}[Single Variable M\"obius Inversion]\label{mob_inv}
Let $X$ be a lower finite poset. Let $f,g:X \rightarrow \Rr$ be given. Let $\zeta:X \times X \rightarrow \Rr$ be defined as usual, and let $\mu = \zeta^{-1}$. Then:
\begin{align*}
f = g \zeta & \iff g = f \mu.
\end{align*}
Equivalently,
\begin{align*}
\forall x \in X, f(x) = \sum_{y \in X} g(y)\zeta(y,x) &\iff \forall x \in X, g(x) = \sum_{y \in X} f(y)\mu(y,x).
\end{align*}
\end{thm}
\begin{proof}
Recall that by Lemma \ref{inv_ex}, $\mu = \zeta^{-1}$ exists. Given that, the result follows if we right-multiply both sides by $\mu$. The first statement becomes the second when the sums are made explicit.
\end{proof}

This result is quite straightforward based on how we defined $\zeta$ and $\mu$. In fact, it feels strange that a supposedly central fact is just the inversion of a single matrix operation. In fact, the simplicity of the inversion step is the whole point of M\"obius inversion. The tricky part is formulating a given problem such that Theorem \ref{mob_inv} can be applied to it. We will see many cases where values of $f$ from the theorem are readily attainable, but values of $g$ are what we really want. The M\"obius function allows us to obtain those values. This procedure - the use of the M\"obius function to reverse the process of summing terms as in theorem Theorem \ref{mob_inv} - is what is referred to as M\"obius inversion.

We now have two equivalent definitions of the M\"obius function, one using matrices (or linear operators) and vectors, and one directly considering elements and sums. We will continue to move back and forth between these two approaches throughout this investigation. This is because although the two definitions are completely equivalent, they each have advantages. Working with sums and elements helps us see the details of what is going on, and reminds us of the structure of our poset. It is also often the perspective from which motivating problems arise. On the other hand, using matrices and vectors greatly simplifies the equations we have to deal with, makes the manipulations themselves easier, and allows us to keep the bigger picture - the idea of inversion - in mind more easily. 
\section{Integer Lattices}
We now turn to a specific class of partially ordered sets, which will come up several times going forward.
\begin{defn}\label{lattice}
For any positive integer $n$, the \emph{integer lattice of dimension $n$}, which we denote $\Nn^n$, is the set of ordered $n$-tuples of non-negative integers. For any two elements $(a_1,\dots,a_n), (b_1,\dots,b_n) \in \Nn^n$, we say that $(a_1,\dots,a_n)\leq (b_1,\dots,b_n)$ if $a_i \leq b_i$ for each $i = 1,\cdots,n$.
\end{defn}
\begin{lemma}
$(\Nn^n, \leq)$ is a lower finite poset for any positive integer $n$. 
\end{lemma}
\begin{proof}
The poset axioms are easily checked. We also observe that for any $x \in (\Nn^n, \leq)$,
\[
\#\{y \in (\Nn^n, \leq):y \leq x\} = \prod_{i =1}^n x_i.
\]

This is a finite product of integers, so is finite.
\end{proof}

We turn to several examples.
\begin{examp}
Consider the $n$-dimensional unit cube in $\Rr^n$, placed such that one corner is at the origin and all vertices have non-negative coordinates. Let $S$ denote the set of vertices of the cube. Then we can think of the vertices of the cube as ordered $n$-tuples of zeros and ones. So $S \subseteq \Nn^{n}$. 

In this example, what is the interpretation of the poset relation? If $v_1, v_2 \subseteq S$, then $v_1 \leq v_2$ if $v_1$ has zeros in all of the coordinates where $v_2$ has zeros. So $v_1 \leq v_2$ if $v_1$ is part of a shortest path from $v_2$ to the origin along the edges of the cube. Equivalently, $v_1 \leq v_2$ if $v_1$ is the projection of $v_2$ onto some subset of the axes. 
\end{examp}
\begin{examp}
Each integer can be uniquely represented by its prime factorization. In particular, if we consider the set of integers whose factorizations only include the first $n$ primes, we can write those factorizations as ordered $n$-tuples, capturing each prime's multiplicity in the factorization. As a result, the integers whose factorizations only include the first $n$ primes form a copy of the $n$-dimensional integer lattice. 

In this case, if $\leq$ represents the poset relation from Definition \ref{lattice}, then $m \leq n$ if, for each prime $p$, $m$'s factorization contains no more copies of $p$ than $n$'s factorization. This is the same as saying that $m|n$. So the poset relation becomes divisibility. We will explore this example more in the next sections.
\end{examp}
It turns out that several important posets have this lattice structure. So the goal is to determine the M\"obius function on certain subsets of the integer lattice, and then transfer those results to other partially ordered sets. To do so, we first need a lemma which applies to the M\"obius function on any poset.
\begin{lemma}\label{recurse}
Let $X$ be a lower finite partially ordered set, and let $\zeta, \mu$ be defined as usual. Then for all $a, b \in X$,
\[
\mu(a,b) = 
\begin{cases}
0 & a \nleq b\\
\delta(a,b)-\sum_{c \in [a,b)}\mu(a,c) & else.
\end{cases}
\]
\end{lemma}
\begin{rmk}
Note that the piecewise form is actually not necessary, because if $a \nleq b$ then the sum will be empty. So it is correct - albeit less clear- to just write
\[
\mu(a,b) = \delta(a,b)-\sum_{c \in [a,b)}\mu(a,c).
\]
\end{rmk}
\begin{proof}
We fix $a,b$, and then rearrange terms starting from Remark \ref{sum_form}:
\begin{align*}
\sum_{c \in X}\mu(a,c) \zeta(c,b) &= \delta(a,b)\\
\sum_{c \in [a,b)} \mu(a,c) + \mu(a,b) &= \delta(a,b)\\
\mu(a,b) &= \delta(a,b) - \sum_{c \in [a,b)} \mu(a,c) 
\end{align*}
\end{proof}
We are now ready to write down $\mu$ on the integer lattice.
\begin{thm}\label{int_lattice}
Let $L$ be the $n$-dimensional integer lattice, with the usual poset relation. Let $\zeta, \mu$ be defined as usual on $L$. Let $x=(x_1,\dots,x_n),y = (y_1,\dots,y_n)$ be elements of $L$. Then:
\[
\mu(x,y) =
\begin{cases}
0 & x\nleq y\\
0 & \exists i \in \Nn: y_i-x_i \geq 2\\
(-1)^{\sum_{i=1}^ny_i-x_i} & \text{otherwise}
\end{cases}
\]
\end{thm}

\begin{proof}
If $x \nleq y$, then by Lemma \ref{recurse}, $\mu(x,y) = 0$. This is the first case.

We now proceed by induction on $\sum_{i=1}^ny_i-x_i$. If $x \leq y$ and $\sum_{i=1}^ny_i-x_i = 0$, then $x = y$. So by Lemma \ref{recurse}, $\mu(x,y) = 1$. This is the base case.

Suppose that we have Lemma \ref{int_lattice} for any $x \leq y$ where $\sum_{i=1}^ny_i-x_i \leq k-1$. Let $x \leq y$ be fixed such that $\sum_{i=1}^ny_i-x_i =k$. Suppose that for some $j \in \Nn, y_j-x_j \geq 2$. Let $y' = (y_1,\dots,y_j-1,\dots,y_n)$. Note that $x \leq y' < y$. So we can decompose the sum from Lemma $\ref{recurse}$. Let
\begin{align*}
\alpha &= \sum_{x \leq z \leq y'} \mu(x,z)\\
\beta &= \sum_{y' < z < y}\mu(x,z).\\
\end{align*}
Then
\begin{align*}
\mu(x,y) &= -\sum_{x \leq z \leq y}\mu(x,z)\\
&= -\sum_{x \leq z \leq y'} \mu(x,z) - \sum_{y' < z < y}\mu(x,z)\\
&= -(\alpha + \beta).
\end{align*}
We note that
\[\alpha = \sum_{z \in L} \mu(x,z)\zeta(z,y').
\]
So by Remark \ref{sum_form}, $\alpha=0$. 

Recall that for all $i \neq j, y'_i = y_i$. This means that for any $z \in L$ where $z>y'$ and $y > z, z_j = y_j$. So $z_j-x_j \geq 2$. By the inductive hypothesis, this implies that $\mu(z,x)=0$. So all the terms of $\beta$ are zero, so $\beta=0$. This implies that \[\mu(z,y)=-(\alpha+\beta)=0.\]

Finally, suppose that for all $i = 1,\dots,n, y_i-x_i \leq 1$. Because $\sum_{i=1}^ny_i-x_i =k$, there must be $k$ indices where $x$ and $y$ differ by one, while at the other $n-k$ indices they are equal. So for each $j = 0,\dots,k-1$, there are $k \choose j$ elements $z$ of $X$ where $x \leq z < y$ and $\sum_{i = 1}^n y_i-z_i = j$. For each of these, by the inductive hypothesis, $\mu(x,z) = (-1)^j$. So, using Remark \ref{sum_form}, we can write
\begin{align*}
\mu(x,y)& = -\sum_{j = 1}^k {k \choose j} (-1)^j\\
&= (-1)^k.
\end{align*}
\end{proof}
This result will be powerful for determining values of $\mu$ on various posets whose structure is similar to an integer lattice. 
\begin{thm}\label{mapping}
Let $(X, \leq_X),(Y, \leq_Y)$ be lower finite posets. Let $x_1 \leq x_2 \in S$ be given. Suppose that there exists a bijective map
\[
\phi: [x_1, x_2] \leftrightarrow [\phi(x_1), \phi(x_2)]
\]
such that for all $z_1, z_2 \in [x_1, x_2]$,
\[ 
z_1 \leq_X z_2 \iff \phi(z_1) \leq_Y \phi(z_2).
\] 
Let $\mu_X, \mu_Y$ denote the M\"obius functions on $X$ and $Y$ respectively. Then:
\[
\mu_X(x_1, x_2) = \mu_Y(\phi(x_1), \phi(x_2)).
\]
\end{thm}
This theorem tells us that if we can find a portion of a familiar poset which has identical structure to a portion of an unknown poset, we can transfer what we know about values of $\mu$ on that portion from one to the other.
\begin{proof}
We induct on $\#[x_1, x_2]$. Because $x_1 \leq x_2$, if $[x_1, x_2] = 1$, then $x_1 = x_2$. So by Lemma \ref{recurse}, $\mu_X(x_1, x_2) = 1$. Because $\phi(x_1) = \phi(x_2), \mu_Y(\phi(x_1), \phi(x)2) = 1$ also.

Suppose the theorem is true if $\#[x_1, x_2] \leq k-1$ for some integer $k$. Fix $x_1, x_2$ such that $\#[x_1, x_2] = k$. Then by Lemma \ref{recurse},
\[
\mu_Y(\phi(x_1), \phi(x_2)) = -\sum_{y \in [\phi(x_1), \phi(x_2)]}\mu_Y(\phi(x_1), y)
\]
Because $\phi$ is an isomorphism which preserves $\leq$, we can rewrite this sum:
\[
\mu_Y(\phi(x_1), \phi(x_2)) = -\sum_{x \in [x_1, x_2]}\mu_Y(\phi(x_1), \phi(x))
\]
We can restrict $\phi$ to $[\phi(x_1), \phi(x)]$ to satisfy the hypotheses for all intermediate elements $x$. So we can apply the inductive hypothesis to all the terms of this sum:
\begin{align*}
\mu_Y(\phi(x_1), \phi(x_2)) &= -\sum_{x \in [x_1, x_2]}\mu_X(x_1, x)\\
&= \mu_X(x_1, x_2)
\end{align*}
\end{proof}
The following corollary follows immediately if we let $\phi$ be the inclusion map:
\begin{cor}\label{subset}
Let $X$ be a poset. Let $I$ be an interval in $X$. In other words, let $I = [z_1, z_2]$ for some $z_1, z_2 \in X$. Then
\[
\mu_I = \mu_X|_I.
\]
\end{cor}
In the next section, we explore a poset where this fact will be extremely useful. 

\section{Integers with Divisibility Ordering}
We are now able to justify the existence of, and find, the function $\mu$ from \eqref{init_mob}, with little work.

We will consider the poset $(\Zz^{+}, |)$. This is the set of positive integers ordered by divisibility (recall that $d|n$ if $d$ divides $n$ without remainder). Note that this poset is lower finite: For any number $n$, the set $\{q \in \Zz^{+}: q|n\} \subseteq \{q \in \Zz^{+}: q \leq n\}$, and the latter set contains $n$ elements, which is finite (note that $d|n$ means that $d$ divides $n$ without remainder). This means we are able to use M\"obius inversion. Rather than compute the M\"obius function directly, we can use Theorems \ref{int_lattice} and \ref{mapping} to reuse what we already know.

\begin{thm}\label{mu_div}
Let $\zeta, \mu$ be defined as usual on the poset $(\Zz^+, |)$. Then for all $d,n \in \Zz^+$,
\[
\mu(d,n) =
\begin{cases}
0 & \nmid n \\
0 & \exists p: p^2|\frac nd\\
(-1)^{\#\{p|\frac nd : p \text{ prime}\}} & otherwise
\end{cases}
\]
\end{thm}

\begin{proof}
Let $d,n \in \Zz^+$ be given. We set up the isomorphism required for Theorem \ref{mapping}: Index the primes $p_1, p_2, \dots$. So $p_1=2, p_2=3$, and so on. Let $k$ be the index of the largest prime which divides either $d$ or $n$. Then let $\phi : [d,n] \rightarrow \Nn^k$ be defined by
\[
 p_1^{x_1}p_2^{x_2}\cdots p_k^{x_k} \mapsto (x_1, \cdots, x_k).
\]
We claim that $\phi$ is a bijection from $[d,n]$ to $\im(\phi)$.

Because prime factorizations are unique, this map is well-defined and injective. Let $m_1, m_2 \in [d,n]$ be given. Then $m_1|m_2$ if and only if the multiplicity of each prime $p_i$ in the factorization of $m_1$ is equal to or smaller than $p_i$'s multiplicity in the factorization of $m_2$. This is equivalent to checking whether or not $\phi(m_1)_i \leq \phi(m_2)_i$, which is the definition of $\leq$ on $\Nn^k$. So this map preserves $\leq$. This implies that $\im(\phi) \subseteq [\phi(d), \phi(n)]$, which allows us to check the final condition, surjectivity. 

Let $x = (x_1,...,x_k) \in [\phi(d), \phi(n)]$ be given. Then $\phi(p_1^{x_1}\cdots p_k^{x_k}) = x$. By above, because $\phi$ preserves $\leq$, $p_1^{x_1}\cdots p_k^{x_k} \in [d,n]$. So $x \in \im(\phi)$. 

So $\phi:[d,n] \leftrightarrow [\phi(d), \phi(n)]$ is a bijection.

This allows us to use Theorem \ref{mapping}, and the result immediately follows.
\end{proof}

This definition of $\mu$ depends only on $\frac nd$, which means that we can abuse notation to write $\mu(\frac nd)$ instead of $\mu(d,n)$. This allows us to use Theorem \ref{mob_inv} to arrive at \eqref{init_mob}:
\[
f(n) = \sum_{d | n} g(d) \iff g(n) = \sum_{d | n}  f(d)\mu\left(\frac nd\right).
\]
\section{Euler's Totient}
Euler's Totient is a famous function in number theory, which counts the how many numbers are relatively prime to a particular value. We denote the function by $\varphi : \Zz^+ \rightarrow \Zz^+$, and its definition is:
\[
\varphi(n) = \#\{1 \leq m \leq n : (m,n) = 1\},
\]
where $(m,n)$ denotes the greatest common divisor of $m,n$.

We wonder how we can write down the value of $\varphi(n)$ given a particular integer $n$. One approach would be to go through all the values from 1 to $n$, find each one's greatest common divisor with $n$, and then count how many are equal to 1. However, for large values of $n$, this could be computationally expensive. We will use the function $\mu$ from the previous part to arrive at a different, likely more efficient, approach.

We start with an important fact:
\begin{thm}[Gauss]\label{gauss}
Let $n \in \Zz^+$ be given. Then 
\[
n = \sum_{d|n}\varphi(d).
\]
\end{thm}
There are a variety of ways of proving this fact. Here we include a concise proof using group theory.
\begin{proof}
For any integer $d$, $\varphi(d)$ counts the generators of $C_d$. Each element of $C_n$ generates exactly one cyclic subgroup, so counting the total number of generators of cyclic subgroups of $C_n$ is equivalent to counting the elements of $C_n$. Also, $C_n$ has exactly one cyclic subgroup for each divisor of $n$ (and no others). So the sum indeed counts the number of elements of $C_n$, of which there are $n$.
\end{proof}
This sum gives us a value we have access to ($n$) in terms of a sum of values we want (values of $\varphi$). This is a perfect situation to apply M\"obius inversion, and undo that sum. We apply \eqref{init_mob}, which tells us
\[
\varphi(n) = \sum_{d|n} d\mu\left(\frac nd\right).
\]
Recall that Theorem \ref{mu_div} tells us how to write down values of $\mu$. Depending on what information we have access to, this might be much easier than just counting the number of values relatively prime to $n$. This is a straightforward example of how M\"obius inversion's ability to reverse a sum can be useful.
\end{chapter}


\begin{chapter}{Groups and Orbits} \label{chap_groups}
\section{Background}
For this chapter, it is necessary to be familiar with a simple structure called a \emph{group}. In this section, we will define it and briefly establish the machinery we need without going into too much detail, omitting proofs. More detail on elementary group theory can be found in an abstract algebra textbook. Knowledge of groups is not needed for any other chapters in this investigation.

\begin{defn}
A \emph{group} is a set $G$ together with a group operation (denoted with the empty operator) $G \times G \rightarrow G$. This operation must satisfy three requirements.
\begin{itemize}
\item For all $a,b,c \in G, (ab)c = a(bc)$
\item There exists an element $e \in G$, such that $ea=a=ae$ for all $a \in G$
\item For each $a \in G$, there exists an $a^{-1} \in G$ such that $aa^{-1} = e = a^{-1}a$.
\end{itemize} 
\end{defn}
\begin{defn}
Let $G$ be a group. Then $H \subseteq G$ is a \emph{subgroup} of $G$ if $H$ is itself a group, with the same group operation as $G$. If $H$ is a subgroup of $G$, we write $H \leq G$.
\end{defn}
\begin{defn}
Let $G$ be a group and let $\Omega$ be a set. Then an \emph{action} of $G$ on $\Omega$ is a map $\cdot:G \times \Omega \rightarrow \Omega$ which satisfies the following properties:
\begin{itemize}
\item $e \cdot x = x$ for all $x \in \Omega$
\item $g \cdot (h \cdot x) = (gh) \cdot x$ for all $g,h \in G, x \in \Omega$.
\end{itemize} 
\end{defn}
\begin{defn}
Let a group $G$ act on a set $\Omega$. Let $x \in \Omega$ be given. Then the \emph{orbit of $x$ under $G$}, denoted by $\orb_G(x)$ or $Gx$, is defined by
\[
\orb_G(x) = \{ y \in G : \exists g \in G \text{ where } g\cdot x = y\}
\]
or, equivalently,
\[
\orb_G(x) = \{g\cdot x : g \in G\}.
\]
\end{defn}

\begin{defn}
Let a group $G$ act on a set $\Omega$. Let $x \in \Omega$ be given. Then the \emph{stabilizer of $x$ under $G$}, denoted by $\stab_G(x)$, is defined by
\[
\stab_G(x) = \{g \in G:g \cdot x = x\}.
\]
\end{defn}
\begin{thm}
Let a group $G$ act on a set $\Omega$. Then for every $x \in \Omega, \stab_G(x)$ is a subgroup of $G$.
\end{thm}
\begin{proof}
This is easily checked.
\end{proof}
\begin{thm}
Let a group $G$ act on a set $\Omega$. Then the orbits of $\Omega$ under $G$ partition $\Omega$.
\end{thm}
\begin{proof}
Omitted, but straightforward.
\end{proof}
\begin{thm}[Fundamental Counting Principle]\label{fcp}
Let a \textbf{finite} group $G$ act on a set $\Omega$. Let $x \in \Omega$ be given. Then
\[
|\stab_G(x)||\orb_G(x)| = |G|.
\]
\end{thm}
\begin{proof}
Omitted - requires technology we don't have here.
\end{proof}
\begin{defn}
Let a group $G$ act on a set $\Omega$. Then $\Omega/G$ denotes the set of orbits in $\Omega$ under $G$.
\end{defn}
\begin{defn}
Let a group $G$ act on a set $\Omega$. Let $A \leq G$. Then the \emph{fixed set} of $A$, or $\fix_A(\Omega)$, is equal to
\[
\{x \in \Omega : \forall g \in A, g \cdot x = x\}.
\]
We can also speak of the fixed set of a group element $g \in G$. In that case, we say
\[
\fix_g(\Omega) = \{x \in \Omega : g \cdot x = x \}.
\]
\end{defn}

\section{M\"obius Inversion on the Poset of Subgroups}
We want to apply M\"obius inversion to finite groups. This means we will need a poset. We will use the \emph{poset of subgroups} (or \emph{lattice of subgroups}):
\begin{defn}
Let $G$ be a finite group. Then let the \textit{poset of subgroups} of $G$, or $\sub(G)$, be the poset consisting of subgroups of $G$ \emph{reverse} ordered by inclusion. In other words, we say that $A \leq B$ if $A \supseteq B$.
\end{defn}
\begin{rmk}
Because $G$ is finite, $\sub(G)$ is finite (so lower finite).
\end{rmk}
\begin{rmk}
We are ordering the poset this way for reasons that will soon become clear. But to reduce confusion, we will continue to use inclusion notation to define $\zeta$, and so $\mu$, rather than using $\leq$. Also note that I will be using $\subseteq$ rather than $\leq$ to denote subgroups, also to reduce confusion with the poset relation.
\end{rmk}
\begin{examp}
Consider the group $S_3$, the symmetric group of order 6. Then we can draw its lattice of subgroups:

\begin{figure}[h]
\[
\includegraphics{../Figures/s3_lattice.pdf}
\]
\caption{The lattice of subgroups of $S_3$.}
\end{figure}
So, for example, $\langle (1,2) \rangle \leq \langle e \rangle$, because $\langle (1,2) \rangle \supseteq \langle e \rangle$. 
\end{examp}
\begin{rmk}
We are ordering the poset this way for reasons that will soon become clear. But to reduce confusion, we will continue to use inclusion notation to define $\zeta$, and so $\mu$, rather than using $\leq$. I will be using $\subseteq$ rather than $\leq$ to denote subgroups, also to reduce confusion with the poset relation.
\end{rmk}
This puts us in the position to apply M\"obius inversion. We first define the usual functions on $\sub(G) \times \sub(G)$:
\begin{itemize}
\item Given a group $G$, let $\zeta:\sub(G) \times \sub(G) \rightarrow \Rr$ be defined by
\[
\zeta(a,b) = \begin{cases}
1 & A \supseteq B\\
0 & else
\end{cases}
\]
\item Let $\mu = \zeta^{-1}$, which exists by Lemma \ref{inv_ex}.
\end{itemize}
Recall that we manipulate $\zeta, \mu$ as matrices.

Now, Theorem \ref{mob_inv} tells us that, for any row vectors $f,g$ of length $|\sub{G}|$, 
\[
f = g\zeta \iff g = f\mu.
\]
Now, of course, the challenge is to determine how to formulate the problem such that inverting $\zeta$ will be useful.
\section{Counting Orbits}
Throughout this section, let $G$ be a finite group acting on a set $\Omega$.

The goal is to count the orbits in $\Omega$ under the action by $G$ - in other words, to find $|\Omega/G|$. We will do that by finding a weighted sum of elements of $\Omega$. If we were to simply count the elements of $\Omega$, we would almost certainly overestimate the number of orbits. We would prefer it if each orbit could ``send a representative.'' If that were possible, counting those representatives would be the same as counting the number of orbits. But there is no clear way of accomplishing this -- how would we choose one element exactly from each orbit? It seems like this would require prior knowledge of the orbit structure of $\Omega$ under $G$, which is exactly what we are seeking.

So we have to change our perspective. Suppose that every element of $\Omega$ were able to provide information telling us the size of its orbit. So suppose that given $x \in \Omega$, we had access to $|\orb_G(x)|$. 

Now suppose that we are counting orbits by checking elements of $\Omega$ one at a time. Each time we see an element $x \in \Omega$, we ask it for the size of its orbit, and then add $\frac 1 {|\orb_G(x)|}$ to a running total. Let $O$ be a single orbit of $\Omega$ under $G$ where $|O|=n$. Then we will see $n$ elements of $\Omega$ which belong to $O$, and for each of them, we will add $\frac 1n$ to our running total. This means we add exactly 1 for $O$. The same is true of all the orbits in $\Omega$. So our final total will be the size of $G/X$.

To formalize this, we have the following relationship:

\begin{equation}\label{eq_counting}
|G/X| = \sum_{x \in \Omega}\frac 1 {|\orb_G(x)|}.
\end{equation}

This is the central idea: We want to group elements of $\Omega$ by the size of their orbits. We will use the \emph{Fundamental Counting Principle}, Theorem \ref{fcp}, together with M\"obius inversion, to do this. Although we already justified \eqref{eq_counting}, we will prove it formally as well.

We start by defining some functions/vectors, each with entries corresponding with subgroups of $G$, which we will be using. Let:
\begin{align*}
f(A) &= \#\{x \in \Sigma: \stab(x) \supseteq A\} = |\fix(A)|\\
g(A) &= \#\{x \in \Sigma: \stab(x) = A\}\\
\sigma(A) &= |A|
\end{align*}

We will operate with the assumption that we are able to find $\sigma$, which is true if we know the structure of $\sub(G)$. We will also assume that we are able to carry out the group action, thereby determining which elements of $\Omega$ are fixed by given subgroups of $G$. This gives us access to $f$. Then there are two steps we will follow:
\begin{description}
\item[Step 1:] Determine how to use $g$ and $\sigma$ to find $|\orb(\Omega, G)|$; and
\item[Step 2:] Find $g$ from $f$ using M\"obius inversion.
\end{description}

Suppose we are handed $g$. In order to see how to find $|\orb(\Omega, G)|$, we recall Theorem \ref{fcp}:

\[|G| = |G(x)||\stab(x, G)|.\]

Because $G$ is finite, this implies that \[|G(x)| = \frac{|G|}{|\stab(x, G)|}.\]

Starting from a trivial fact, we have:
\begin{align*}
|\orb(\Omega, G)| &= \sum_{O \in \orb(\Omega, G)}1\\
&= \sum_{O \in \orb(\Omega, G)}\left(\sum_{x \in O}\frac{1}{|O|}\right)\\
&= \sum_{x \in \Omega}\frac{1}{|G(x)|}
\end{align*}

Substituting, we now have the relationship:
\[
|\orb(\Omega, G)| = \frac{1}{|G|}\sum_{x \in \Omega}|\stab(x, G)|
\]
Each element of $\Omega$ has exactly one stabilizer in $G$. This means that we can partition the above sum as follows:
\[
|\orb(\Omega, G)| = \frac{1}{|G|}\sum_{A \in \sub(G)}\left(\sum_{x \in \Omega: \stab(x) = A}|A|\right)
\]
Using the definitions of $g$ and $\sigma$, we can rewrite to get the relationship we need:
\begin{align*}
|\orb(\Omega, G)| &= \frac{1}{|G|}\sum_{A \in \sub(G)}g(A)|A|\\
&= \frac {1}{|G|}g \cdot \sigma
\end{align*}

On to Step 2. It is clear that $f_A = \sum_{B \supseteq A} g_A$. Another way of putting that is that $f = g\zeta$. As usual, this implies that $g = f\mu$. So M\"obius inversion makes this step trivial. Putting it all together:

\[|\orb(\Omega, G)| = \frac {1}{|G|}f\mu \cdot \sigma.\]

Recall that $\sigma$ and $\mu$ do not depend on the details of the group action, or on $\Omega$ at all. They only depend on $G$. So it is useful to rearrange this expression to isolate $f$, the only part which does depend on the action. This gives us the central result of this chapter:
\begin{thm}\label{orbit_count}
Let $G$ act on a set $\Omega$. Let $f, \mu, \sigma$ be defined as above. Then
\[
|\orb(\Omega, G)| = \frac {1}{|G|}(\mu \cdot \sigma)^Tf^T
\]
\end{thm}

By combining M\"obius inversion with basic group theory, we have a recipe for turning something readily obtainable (the function $f$) into something less obvious (the number of orbits of an action).

The reason Theorem \ref{orbit_count} is powerful is that we can do almost all of the work knowing only the structure of the group, before we know anything about the set or the action. If we are handed a group $G$, we can define the vector $\lambda$ by
\begin{equation}\label{eq:lambda}
\lambda = \frac {1}{|G|}(\mu \cdot \sigma)^T.
\end{equation}
Then, for any action of $G$ onto any set $\omega$, all we need to do is compute $f$, and then Theorem \ref{orbit_count} tells us that
\[
|\Omega/G| = \lambda f^T.
\]
In the next examples, we will walk through all the steps involved in counting the orbits of some simple actions.

\section{Examples}
\begin{examp}\label{ex:2x2}
Let $G = C_4 = \{e, \alpha, \alpha^2, \alpha^3\}$, the cyclic group of order 4.  
\begin{figure}[h]
\[
\includegraphics{../Figures/c4_lattice.pdf}
\]
\caption{The lattice of subgroups of $C_4$}
\label{fig:c4_lattice}
\end{figure}We first compute $\lambda$ using the formula from \eqref{eq:lambda}. We start by finding the M\"obius function. $C_4$ has three subgroups: All of $C_4$, the subgroup generated by $\alpha^2$, and the identity. So $\zeta$ will be a $3 \times 3$ matrix. Because each subgroup contains the next, there is only one admissible total ordering we can use to construct our vectors:
\[
C_4 < \langle \alpha^2 \rangle < \langle e \rangle.
\]
We can go through all the pairs of subgroups to determine that
\[
\zeta = \begin{bmatrix} 1 & 1 & 1 \\ 0 & 1 & 1 \\ 0 & 0 & 1 \end{bmatrix}.
\]
For example, because the first subgroup, $C_4$ contains the second, $\langle \alpha^2 \rangle$, the 1, 2 entry of $\zeta$ is equal to 1.

This implies that
\[
\mu = \zeta^{-1} = \begin{bmatrix} 1 & -1 & 0\\ 0 & 1 & -1 \\ 0 & 0 & 1 \end{bmatrix}.
\]
Next, we find $\sigma$. Recall that $\sigma$ is the vector of subgroup sizes. So 
\[
\sigma = [4, 2, 1].
\]
Finally, note that $|C_4|=4$. So
\begin{align*}
\lambda &= \frac 14 \mu \sigma^T\\
&= \frac 14 
\begin{bmatrix} 1 & -1 & 0\\ 0 & 1 & -1 \\ 0 & 0 & 1 \end{bmatrix}
\begin{bmatrix} 4 \\ 2 \\ 1 \end{bmatrix}\\
&= \frac 14 \begin{bmatrix} 2 \\ 1 \\ 1\end{bmatrix}
\end{align*}
This $\lambda$ is independent of any set $G$ might act on. 

So let's try a particular action. Suppose we have a $2 \times 2$ chessboard, and we want to place some number of rooks on the board such that none of them attack one another. If one configuration can be obtained from the other by rotating the board (ignoring light and dark squares), then we wish to count them as the same configuration. The goal is to count how many distinct configurations there are, which cannot be obtained from each other by rotation.

We can first solve this problem by inspection, or by "brute force," by just enumerating the possibilities:
\begin{description}
\item[Zero rooks:] There is one option with zero rooks: The empty board.
\[
\includegraphics{../Figures/zero_rooks.pdf}
\]
\item[One rook:] We can put one rook on any of the four squares. However, we can rotate the board to obtain any of these from the others. So there is just one distinct configuration. \\
\[
\makebox[\textwidth][c]{\includegraphics{../Figures/one_rook.pdf}}
\]
\item[Two rooks:] We can put rooks on opposite corners. There are two such pairs, but we can rotate the board and obtain either pair from the other. So, again, there is only one distinct possibility.
\[
\makebox[\textwidth][c]{\includegraphics{../Figures/two_rooks.pdf}}
\]
\end{description}
So there are a total of three distinct arrangements of rooks. Now we can try to solve this problem using $\lambda$.

Let $\Omega$ be the set of possible rook arrangements without regard for rotational equivalence. So putting a rook in the upper left hand corner and putting one in the lower left hand corner would be considered to be different arrangements. Now let $C_4$ act on $\Omega$ by rotating the board. So $\alpha$ rotates the board $90^\circ$ counterclockwise, $\alpha^2$ rotates it twice, etc. Then each distinct arrangement of rooks corresponds exactly with an orbit of $C_4$'s action on $\Omega$.

Luckily, we know how to count orbits. All we have to do is find $f$. We go through the subgroups of $C_4$.
\begin{itemize}
\item $C_4$ only fixes the empty board. Any other arrangement can be changed by a rotation. So $f(1) = 1$.
\item The subgroup $\langle \alpha^2 \rangle$ contains the identity element as well as the $180^\circ$ rotation. This subgroup fixes both arrangements with two rooks, as well as the empty board. So $f(2) = 3$.
\item The subgroup $\langle e \rangle$ fixes all the arrangements. There are 2 ways to place two rooks, 4 ways to place one rook, and 1 way to place zero rooks. So $f(2) = 7$.
\end{itemize}
Then all we have to do is compute $\lambda f^T$.
\begin{align*}
\lambda f^T &= \frac 14 [2, 1, 1] \begin{bmatrix} 1 \\ 3 \\ 7\end{bmatrix} \\
&= \frac 14 12\\
&=3
\end{align*}
So we indeed find the correct answer. 
\end{examp}

In this example, it's not clear that using $\lambda$ actually saved work - in fact, it seemed to take longer. But if we were to increase $n$, we could still determine the number of distinct arrangements relatively easily, because $\lambda$ wouldn't change. Furthermore, in many situations (including this one) it is much easier to determine how many \emph{total} arrangements are fixed by a subgroup that it is to find the number of \emph{distinct} arrangements. As an example of that, we can look at another example, which is more difficult to solve by hand. 
\begin{examp}
Suppose we have a $4 \times 4$ board, and we would like to color each square one of $k$ colors. We would like to consider colorings identical if one can be obtained from the other by rotating the board. The goal is to count the number of distinct colorings.

As in Example \ref{ex:2x2}, we let $C_4$ act on the total set of colorings, and count the orbits. This problem is different from the previous one, but because the group which is acting is the same, we can re-use $\lambda$ from the last example. This is one reason this method is powerful. All we have to do is find $f$ by finding the fixed set of each subgroup of $C_4$.

\begin{description}
\item[$C_4$:] We want the number of colorings which are fixed under any rotation. This means that all four quadrants must be symmetrically colored. So we color one quadrant, and then use symmetry to determine how to color the rest. There are four squares in each quadrant, so this subgroup fixes $k^4$ colorings.
\item[$\langle \alpha^2 \rangle$:] We use a similar approach: We color half the board, and then determine the coloring on the other half using symmetry. This means we are coloring 8 squares however we want. So there are $k^8$ colorings.
\item[$\langle e \rangle$:] The identity subgroup always fixes all elements. We have 16 squares, so the fixed set has size $k^{16}$.
\end{description}
So the number of distinct colorings is
\begin{align*}
\lambda f &= \frac 14 [2,1,1]\begin{bmatrix} k^4 \\ k^8 \\ k^{16}\end{bmatrix}\\
&= \frac 14 (2k^4 + k^8 + k^{16}).
\end{align*}
\end{examp}
It's not clear how we would have solved this problem by brute force, and we were able to solve it very quickly, reusing $\lambda$ to save even more work.
\end{chapter}

\begin{chapter}{Inclusion-Exclusion}
\section{Introduction}
We start by introducing the desired end result, the \emph{inclusion-exclusion principle}:
\begin{thm}[Inclusion-Exclusion]\label{iep}
Let $S$ be a finite set. Let $A_i \subseteq S$ for each $i \in I$ where $I$ is finite. Let $K \subseteq I$ be given. Then
\[
\left|\bigcup_{i \in K}A_i\right| = \sum_{\emptyset \neq J \subseteq K}(-1)^{|J| + 1}\left|\bigcap_{i \in J}A_j\right|
.\]
\end{thm}
This is a well-known fact. It allows us to find the size of unions of overlapping sets if we have information about intersections of those sets. There are many ways of proving the Inclusion-Exclusion Principle. %TODO: Explain why proving it with Mobius inversion is worthwhile

\section{Proof using M\"obius Inversion} 
We will need a poset.

\begin{defn}
For any set $I$, the \emph{power set} of $I$, denoted by $\ps(I)$, is the set of subsets of $I$. 
\end{defn}
\begin{rmk}\label{power_poset}
Let $I$ be a finite set. Then $(\ps(I), \supseteq)$ is a lower finite partially ordered set.
\end{rmk}
Note that the poset from Remark \ref{power_poset} is ordered by reverse inclusion rather than simple inclusion. The poset axioms are easily checked.

Throughout this section, we will have the following setup: Let $S$ be a finite set. Let $I$ be a finite index set, and for all $i \in I$, let $A_i \subseteq S$. We will be manipulating functions on $(\ps(I), \supseteq)$, so this will be our poset. 

There are two sets of subsets here, and it can be hard to keep them straight. $S$ is a set of elements, and for each $i$, $A_i$ is a subset of $S$. On the other hand, $I$ is our index set, and our partially ordered set consists of subsets of $I$. We will perform M\"obius inversion on subsets of $I$ in order to determine sizes of unions of subsets of $S$.

The goal of the inclusion-exclusion principle is, given a subset $K$ of $I$, to find the size of the union of the sets $A_i$ for $i \in K$. The difficulty arises from counting elements multiple times. For example, if we knew that the $A_i$ were disjoint, we could just consider
\[
\sum_{i \in K}\left|A_i\right|.
\]
But, because intersections of the $A_i$ might be nonempty, this sum might count some elements multiple times. We must approach the problem more carefully. First, we define the following indicator function $\chi: s \rightarrow \ps(I)$ by:
\[
\chi(x) = \{i \in I : x \in A_i\}.
\]
So $\chi$ gives us the indices of the subsets of $S$ which contain $x$.  
Next, we define some functions defined on our poset, $\ps(I)$.
Let $J \subseteq I$ be given. Define $f,g,\hat{f}, \hat{g}:\ps(I) \rightarrow \Nn$ by

\begin{align*}
f(J) &= \left| \bigcap_{i \in J} A_i\right| &= |\{x \in S: \chi(x) \supseteq J\}|\\
g(J) &= \left| \bigcap_{i \in J} A_i \cap \bigcap_{i \notin J} A_i^c\right| &= |\{x \in S: \chi(x) = J\}|\\
\hat{f}(J) &= \left| \bigcap_{i \in J} A_i^c\right| &= |\{x \in S: \chi(x)^c \supseteq J\}|\\
\hat{g}(J) &= \left|\bigcap_{i \in J} A_i^c \cap \bigcap_{i \notin J} A_i \right| &= |\{x \in S: \chi(x)^c = J\}|
\end{align*}

We can go through these functions to get a better handle on what they mean:
\begin{itemize}
\item The function $f$ counts the elements contained in $A_i$ for $i \in J$;
\item The function $g$ counts the elements contained \emph{only} in the $A_i$ for $i \in J$;
\item The function $\hat{f}$ counts the elements contained in none of the $A_i$ for $i \in J$; and
\item The function $\hat{g}$ counts the elements contained in none of the $A_i$ for $i \in J$, but contained in all the other $A_i$.
\end{itemize}
Finally, we recall that our goal is to find the union of the $A_i$ for $i$ in some given $K$. To that end, we observe that
\[
\left| \bigcup_{i \in J=K} A_i \right| = |S| - \hat{f}(K).
\]

In order to find $\hat{f}$, we suppose that we have access to \emph{intersections} (but not unions) of the $A_i$ (it is apparent that this is required to use Theorem \ref{iep}). This means that we have access to $f$. 

We can observe certain relationships, which will start to point the way for using M\"obius inversion:
\begin{align*}
f(J) &= \sum_{K \supseteq J} g(K)\\
\hat{f}(J) &= \sum_{K \supseteq J} \hat{g}(K)\\
g(J) &= \hat{g}(J^c)
\end{align*}
Using these relationships, we can:
\begin{enumerate}
\item Use $f$ to find $g$ by using M\"obius inversion to invert the sum;
\item Translate from $g$ to $\hat{g}$; and
\item Use $\hat{g}$ to determine $\hat{f}$.
\end{enumerate}

Let $\zeta$ be defined as usual on $(\ps(I), \supseteq)$. So let $\zeta:\ps(I) \times \ps(I) \rightarrow \Rr$ be defined by
\[
\zeta(J,K) = \begin{cases} 1 & J \supseteq K \\ 0 & \text{otherwise}. \end{cases}
\]

Also, we define the \emph{complement} function/matrix $c$ as follows: %TODO: Change variable name?
\[
c(J,K) = \begin{cases} 1 & J = K^c \\ 0 & \text{otherwise}. \end{cases}
\]
Note that by definition, $c(J,K) = c(K,J)$. So $c$ is symmetric. We also observe that $c^2 = I$, so $c$ is self-inverse. 

Using $\zeta$ and $c$, we can rewrite the relationships between the functions:
\begin{align*}
f &= g\zeta\\
\hat{f} &= \hat{g}\zeta\\
g &= \hat{g}c\\
\hat{g} &= gc
\end{align*}
The complement matrix is readily accessible and self-inverse, so we are able to move easily between $g$ and $\hat{g}$. It is also straightforward to determine $\hat{f}$ from $\hat{g}$ using $\zeta$. The only remaining step is to determine $g$ using $f$. As usual, we use M\"obius inversion to reverse multiplication by $\zeta$.

Let $\mu = \zeta^{-1}$ (as usual, we mean the matrix inverse). Then we can add the following important fact, using Theorem \ref{mob_inv}:
\[
g = f\mu.
\]
Putting it all together, we can write $\hat{f}$ in terms of $f$ step-by-step:
\begin{align*}
\hat{f} &= \hat{g}\zeta\\
&= gc\zeta\\
&= f\mu c\zeta.
\end{align*}
In a sense, we are done. We have determined how to find unions of sets of $A_i$ using $\hat{f}$, and we have written $\hat{f}$ in terms of values we have access to. But with a bit more work, we can show that what we have derived is really the Inclusion-Exclusion Principle. The first step is to interpret values of $\mu$. We will again use Theorems \ref{int_lattice} and \ref{mapping}. 

\begin{thm}
Let $\mu$ be defined as above. Then for all $J, K \in \ps(I)$,
\[
\mu(J,K) = 
\begin{cases}
0 & J \not\supseteq K\\
(-1)^{|K|-|J|} & \text{otherwise}.
\end{cases}
\]
\end{thm}
The idea is that $\ps(I)$ can be viewed as a subset of the $|I|$-dimensional integer lattice, where all the coordinates are either ones or zeros, with a one corresponding to an element being present in a subset and a zero corresponding with that element being absent. This is the idea behind the bijection we will set up.
\begin{proof}
Label the elements of $I$, so that $I = \{\iota_1, \dots, \iota_{|I|}\}$. Let $\Nn^{|I|}$ be the integer lattice of dimension $|I|$. Let $X$ be the subset of $\Nn^{|I|}$ consisting of the points with all coordinates either zero or one. Let $\phi:X \rightarrow \ps(I)$ be defined as follows:
\[
\phi(x_1, \dots, x_{|I|}) = \{\iota_w : x_w = 0\}
\]
There are a variety of things we must check to use Theorem \ref{mapping}.
\begin{itemize}
\item $X = [(0,\dots,0), (1,\dots,1)] \subset \Nn^{|I|}$, so $X$ is an interval, as required.
\item Let $J \subseteq I$ be given. Then we can construct an element of $X$ which maps to $J$ by setting the appropriate elements to zeros and ones. So $\phi$ is surjective on $\ps(I)$. 
\item Distinct elements of $X$ map to distinct subsets of $I$. So $\phi$ is injective. 
\item It is straightforward to check that $\phi(x) \leq_{\ps(I)} \phi(y) \iff x \leq_{\Nn^{|I|}} y$.
\end{itemize}
This means that, if $\mu_{\Nn^{|I|}}$ is the M\"obius function on $\Nn^{|I|}$, \begin{align*}
\mu_{\ps(I)}(J,K) &= \mu_{\Nn^{|I|}}(\phi(J), \phi(K)) \\
&= \begin{cases}
0 & J \not\supseteq K\\
(-1)^{|K|-|J|} & \text{otherwise}
\end{cases}
\end{align*}
\end{proof}
This allows us to determine values of $\mu$. Given that we know $f, c$, and $\zeta$, we are now ready to determine $\hat{f}$.

Recall that our goal is to determine 
\[
\left| \bigcup_{i \in J} A_i \right|,
\]
and that earlier we observed that
\begin{align*}
\left| \bigcup_{i \in J} A_i \right| &= |S| - \hat{f}(J)\\
&= |S| - (f\mu c\zeta)(J).
\end{align*}
We can walk through the matrix algebra. We consider the $J,K$ entry of $\mu c \zeta$. First we take the dot product of row $J$ of $\mu$ with column $K$ of $c$. Recall that column $K$ of $c$ is zero everywhere except in the row corresponding to $K^c$. So $(\mu c)(J,K) = \mu(J, K^c)$. 

Now we must consider the dot product of row $J$ of $(\mu c)$ and column $K$ of $\zeta$. Recall that $\zeta(L,K)$ is equal to one if and only if $L \supseteq K$, and zero otherwise. So, putting it all together:
\begin{align*}
(\mu c \zeta)(J,K) &= \sum_{L \supseteq K} \mu(J, L^c)\\
&= \sum_{L \subseteq K^c} \mu(J, L) &\text{re-indexing the sum}\\
\end{align*}
Finally, we would like to multiply this on the left by $f$ to obtain $\hat{f} = f\mu c\zeta$. Expanding the dot product and rearranging, we have
\begin{align*}
f\mu c\zeta(K) &= \sum_{J \in \ps(I)} f(J) \sum_{L \subseteq K^c} \mu(J,L)\\
&= \sum_{J \supseteq L} \sum_{L \subseteq K^c} (-1)^{|J|-|L|} f(J)\\
&= \sum_{J \in \ps(I)}\left( \sum_{L \subseteq K^c \cap J}(-1)^{|J| - |L|} \right) f(J)\\
&= \sum_{J \in \ps(I)}\left( \sum_{L \subseteq K^c \cap J}\mu(\emptyset, L)\right) (-1)^{|J|}f(J)
\end{align*}
We now apply Remark \ref{sum_form} to simplify the inner sum and insert the definition of $f$:
\begin{align*}
\hat{f}(K) &= f\mu c\zeta(K) \\
&=  \sum_{J \in \ps(I)}(\delta(\emptyset, K^c \cap J)(-1)^{|J|} )f(J)\\
&= \sum_{J \subseteq K}(-1)^{|J|}f(J)\\
&= \sum_{J \subseteq K}(-1)^{|J|}\left| \bigcap_{i \in J} A_i\right|\\
\end{align*}
Recall that 
\[
\left| \bigcup_{i \in K} A_i \right| = |S| - \hat{f}(K).
\]
So we can arrive at Theorem \ref{iep}:
\begin{align*}
\left| \bigcup_{i \in K} A_i \right| &= |S| - \sum_{J \subseteq K}(-1)^{|J|}\left| \bigcap_{i \in J} A_i\right|\\
&= \sum_{\emptyset \neq J \subseteq K}(-1)^{|J| + 1}\left|\bigcap_{i \in J}A_j\right|.
\end{align*}
This is precisely the Inclusion-Exclusion principle.

It makes sense that we would use M\"obius inversion to prove the Inclusion-Exclusion principle. The difficult aspect of counting how many elements are in the union of the sets is avoiding double-counting. %TODO: Finish this

\end{chapter}

\begin{chapter}{Euler Characteristic of a Category}\label{chap_euler}
\section{Introduction to Categories} \label{cat_defs}

A category is an algebraic structure which generalizes ``objects'' and ``maps'' (or ``morphisms''), with composition. There are several equivalent definitions, all somewhat technical, but the construct itself is quite intuitive once you get used to it.  
\begin{defn}[\cite{Kelly1}]
A \textit{category} $C$ consists of:
\begin{itemize}
\item A collection $C_0$ of \emph{objects};
\item For each $a,b \in C_0$, a set $C(a,b)$ of \emph{morphisms from $x$ to $y$}, sometimes called the \emph{hom-set} of $(a,b)$;
\item For each $a,b,c \in C_0$, a \emph{composition operator}: A function $\circ_{abc} :C(a,b) \times C(b,c) \rightarrow C(a,c)$ (note that we usually omit the subscript, because they can be inferred); and
\item For each $a \in C_0$, an \emph{identity morphism} $1_a \in C(a,a)$,
\end{itemize}
such that the following axioms are satisfied:
\begin{itemize}
\item Composition is associative: If $a,b,c,d \in C_0$ and $f\in C(a,b), g \in C(b,c), h \in C(c,d)$, then $f \circ (g \circ h) = (f \circ g) \circ h$, and
\item Composition respects identities: If $a,b \in C_0, f \in C(a,b)$, then $1_b \circ f = f = f \circ 1_a$.
\end{itemize}
\end{defn}
Note that we sometimes use function notation to write morphisms. So if $f \in C(a,b)$, we might right $f:a \rightarrow b$. We also often write $x \in C$ rather than $x \in C_0$ for an object $x$ of $C$, although technically $C$ is the entire category while $C_0$ is the set of objects.

Categories are extremely general constructions. Note that we did not even insist that the collection of objects in a category is a set, which allows us consider ``larger'' categories such as the category of sets. We consider that example here:
\begin{examp}
If $f:X \rightarrow Y, g:Y\rightarrow Z$ are functions, then $g \circ f$ exists, and function composition is associative. So we can let \catname{set} be the category of all sets, with a morphism between sets $X,Y$ for each map $X \rightarrow Y$. 
\end{examp}

We present several examples.
\begin{examp}
Let $G$ be a group. Then we can represent $G$ as a category $C$ as follows: 
\begin{itemize}
\item Let $C_0 = \{*\}$, a single object
\item Let $C(*, *) = G$. In other words, let the set of morphisms from $*$ to $*$ be equal to the set of group elements. 
\item Use the group operation for multiplication. In other words, if $g,h \in C(*,*)$, then, because $g,h \in G$, we can write $gh$. So let $g \circ h = gh$. 
\end{itemize}
It is straightforward to check that $C$ is indeed a category. 
\end{examp}



\section{M\"obius Inversion on a Category}
To define M\"obius inversion on a partially ordered set, we used an ``indicator function'' which was notated as $\zeta$. We will henceforth refer to this function as $\zeta'$, reserving $\zeta$ for a function we are about to define on categories. Recall that for $a,b$ in a poset $X$, $\zeta'(a,b)$ was defined to be 1 if $a \leq b$, and 0 otherwise. We would like to define something similar on categories. It turns out that categories are generalizations of partially ordered sets - any poset can be represented using a category.

Given a poset $X$, we can construct a category by letting $C_0=X$, and adding a single morphism from elements $a,b \in C_0$ if and only if $a \leq b \in X$. We formalize this:

\begin{defn}\label{cat_poset}
Let $(X, \leq)$ be a poset. Let $C$ be the category whose objects are elements of $X$, and where for any pair of elements $a,b \in X$, the set $C(a,b)$ contains one morphism if $a \leq b$, and zero otherwise. Then $C$ is the \emph{poset category of $X$}, which we will refer to as $\cat(X)$.
\end{defn}
Then $\cat(X)$ is a category for any poset $X$. The identity requirement follows from reflexivity of posets, and because pairs of elements have at most one morphism between them, we do not need to specify the composition operators.

To do M\"obius inversion, we need a function/matrix $\zeta$. We would like the $\zeta$ arising from $\cat(X)$ to be the same as the one arising from $X$. Motivated by that intuition, we define $\zeta$ as follows. Note that it is not clear what a ``lower finite'' category would look like, so we restrict ourselves to finite categories.

\begin{defn}
Let $C$ be a finite category. Then define the \emph{similarity matrix} or \emph{similarity function} $\zeta : C \times C \rightarrow \Rr$ by 
\[
\zeta(a,b) = |C(a,b)|.
\]
\end{defn}
So $\zeta$ simply counts the number of morphism from $a$ to $b$. In $\cat(X)$, there is one morphism if $a \leq b \in X$ and zero otherwise. So $\zeta(a,b) = \zeta'(a,b)$ (note that here $a,b \in C$ and also $a,b \in X$).

What is unclear at this point is what the purpose is of doing M\"obius inversion on a category. It is not clear what kind of sum we are attempting to reverse - unlike the poset case, we don't know why we would want to reverse the operation of multiplying by $\zeta$. It turns out that inverting $\zeta$ (if we are able to) will allow us to calculate what is called the \emph{Euler--Leinster characteristic} of a category, which is the analog of an Euler characteristic on other structures. We will explore this over the next few sections. First, we finish the definition of the M\"obius function on categories.

\begin{defn}[\cite{Lein1}]\label{cat_mob} 
Let $C$ be a finite category. Let $\zeta$ be the similarity matrix defined above. Then the \emph{M\"obius function} $\mu$ on $C$, if it exists, is defined by 
\[
\mu = \zeta^{-1}.
\]
Equivalently, $\mu$ is the function which, for all $x,y \in C$, satisfies
\[
\sum_{z \in X} \mu(x,z)\zeta(z,y) = \delta(x,z) = \sum_{z \in X} \zeta(x,z)\mu(z,y).
\]
If such a $\mu$ exists, $C$ is said to \emph{have M\"obius inversion} or to \emph{admit M\"obius inversion}.
\end{defn}
Note that $\mu$ may not exist. Unlike in the poset case, we have no way of ensuring that $\zeta$ is upper triangular or invertible. For example:
\begin{examp}\label{2_pt}
Let $C$ be the category with two objects $a$ and $b$, with exactly one morphism going each direction between $a$ and $b$, and with no other morphisms except the required identity morphisms. 
\begin{figure}[h]
\[
\includegraphics{../Figures/cat_2.pdf}
\]
\caption{Diagram of symmetric category with 2 points and 4 morphisms.}
\end{figure}
Note that whenever we draw a category, we omit the identity morphisms. They are implied, and serve only to clutter the diagram. We also omit implied compositions of morphisms. In this case, because the identity morphisms are the only elements of $C(a,a)$ and $C(b,b)$, it must be that the two drawn morphisms compose to the identity morphisms.


We observe $\zeta$ defined on $C$ is
\[\begin{bmatrix}1 & 1 \\ 1 & 1 \end{bmatrix}.\]
This is not an invertible matrix. So this simple category does not admit M\"obius inversion.
\end{examp} 

\section{Euler--Leinster Characteristic of a Category}
We are now ready to construct central definition of this section: the Euler--Leinster characteristic (or just Euler characteristic) of a category. The entire construction can be found in \cite{Lein1, Lein2, Lein4}, and is due to Tom Leinster.

\begin{defn}\label{def_weighting}
A \emph{weighting} on a category $\Aa$ is a function $\omega$ on $\obj(\Aa)$ such that for all $a \in \Aa$,
\[
\sum_{b \in C}|C(a,b)|\omega(b) = 1,
\]
or, equivalently, if
\[
\sum_{b \in C}\zeta(a,b)\omega(b) = 1.
\]
\end{defn}

One of the first questions we ask when we see this definition is how we can write find valid weightings. Luckily, this definition looks familiar. We have a value we know (1) expressed as the sum of function we are seeking ($\omega$). So we can use M\"obius inversion! The situation is slightly more complicated, because we don't know for sure that $\zeta$ is invertible. We formalize that:

\begin{lemma}
\label{mobius_is_weighting}
If $C$ has M\"obius inversion, then the weighting $\omega_\mu$ defined by $\omega_\mu(a) = \sum_b \mu(a,b)$ is a weighting on $C$.
\end{lemma}
\begin{proof} For all $a \in C$,
\begin{align*}
\sum_b \zeta(a,b)\omega_\mu(b) &= \sum_b\left( \zeta(a,b)\sum_c \mu(b,c)\right)\\
&=\sum_b\sum_c\zeta(a,b)\mu(b,c)\\
&=\sum_c\left(\sum_b\zeta(a,b)\mu(b,c)\right)\\
&=\sum_c \delta(a,c)\\
&=1.
\end{align*}
\end{proof}

Note that even if $C$ does not admit M\"obius inversion, it might admit a weighting. For example, the category in Example \ref{2_pt} admits the weighting $\omega(a) = \omega(b) = 0.5$, even though it does not admit M\"obius inversion.

This is important. It provides a way in which we can write down weightings on metric spaces, since the similarity matrix $\zeta$ is invertible in many cases. We will investigate exactly which cases when we discuss positive semidefinite matrices.

\begin{defn}[Leinster, \cite{Lein1}]\label{cat_euler}
Let $\omega$ be a weighting on a finite category $A$. Then the \emph{Euler--Leinster characteristic}, or just the \emph{Euler characteristic}, of $A$, denoted by $\chi(A)$, is defined by
\[
\chi(A) = \sum_{a \in A} \omega(a)
\]
\end{defn}
Note that even if a category admits several weightings, they all lead to the same Euler characteristic by a simple sum rearrangement argument. Together with Lemma \ref{mobius_is_weighting}, this implies the following:
\begin{cor}\label{euler_mu}
If a finite category $A$ has a M\"obius inversion $\mu$, then
\[\chi(A) = \sum_{a \subseteq A} \sum_{b \subseteq A} \mu(a, b).
\]
\end{cor}
%TODO: Think of category w/ no euler char
\section{Simplicial Complexes}
Calling a new construction an Euler characteristic needs some intuitive justification. We will introduce a structure called a \textit{simplicial complex} to show that calling this value an Euler characteristic is reasonable. 

The most elementary form of an Euler characteristic is the version defined on polyhedrons: Vertices minus edges plus faces. As a result, it is common to think of the Euler characteristic as a value defined on geometric objects. A simplicial complex is a generalization of a polyhedron into higher dimensions. We use this construction because it has a natural notion of an Euler characteristic, which is consistent with the common one, and yet also is easily connected with categories. We start with the definition:

\begin{defn}
Let $V$ be a set. A \emph{simplicial complex} $\sigma$ over $V$ is a set of finite, non-empty subsets of $V$, called \emph{simplices} or \emph{cells}, such that, if $\alpha \in \sigma$ and $\emptyset \neq \beta \subset \alpha$, then $\beta \in \sigma$.
\end{defn}

Put another way, a simplicial complex over $V$  is a set of subsets of $V$ which is closed under taking subsets, with the exception of the empty subset. 

If a set in a simplicial complex $\sigma$ has $k$ elements, we call it a \emph{$k-1$-simplex} or a \emph{$k-1$-cell}. We think of 0-cells as vertices, 1-cells as edges, 2-cells as faces, etc. This will become much clearer with some examples.
\begin{examp}
Let $V=\{a,b\}, \sigma_2 = \{\{a\},\{b\}, \{a,b\}\}$. Then $\sigma_2$ is a simplicial complex over $V$. 
\begin{figure}[h]
\[
\begin{tikzpicture}[auto, node distance = 3cm, main node/.style={dot}]

\node[label = above:{$\{a\}$}, circle, draw, fill=black,
                        inner sep=0pt, minimum width=4pt](1) at (0,0) {};
\node[label = above:{$\{b\}$}, circle, draw, fill=black,
                        inner sep=0pt, minimum width=4pt](2) at (3,0) {};

\draw[-latex] (1) -- (2) node[midway, above = 3pt] {$\{a,b\}$};

\end{tikzpicture}\]
\caption{The simplicial complex $\sigma_2$.}
\end{figure}
Because $\{a\} \subseteq \{a,b\}$, the vertex $\{a\}$ is an endpoint of the line $\{a,b\}$. Because $\sigma_2$ is closed under taking subsets, as all simplicial complexes must be, the line has both its endpoints.
\end{examp}
Going forward, when writing down simplicial complexes, we will not explicitly write down the set $V$ - it will implicitly be the set of all elements used in the simplicies. We will also omit the braces around the subsets to reduce clutter. So we would write $\sigma_2 = \{a,b,ab\}$.

We saw in the previous example why it is that lines always have both their endpoints in the geometric representation of a simplicial complex. For the same reason, faces have all three edges. But the converse is not necessarily true - the outline of a triangle, for example, might exists without the triangle being filled.
\begin{examp}
Let $\sigma = \{a, b, c, d, ab, ac, bc, bd, cd, abc\}$. Then the cell $bcd$ is not in $\sigma$, even though all of its sides and vertices are.
\begin{figure}[h]
\[
\includegraphics[scale = 1.3]{../Figures/simp_1.pdf}
\]
\caption{$\sigma$ is an example of how a triangle can be outlined but not filled.}
\end{figure}
\end{examp}
As these examples illustrate, a $k$-simplex, or a simplex with $k+1$ elements, is considered to be a $k$-dimensional feature.

The goal of this construction was to justify our definition of the Euler characteristic of a category. To do that, we need to define the Euler characteristic of a simplicial complex.
\begin{defn}
Let $\sigma$ be a simplicial complex. Then the \emph{Euler characteristic} of $\sigma$, denoted by $\chi(\sigma)$, is equal to
\[
\sum_{\tau \in \sigma}(-1)^{|\tau|}.
\]
\end{defn}
Note that this is a distinct (although related) definition from Definition \ref{cat_euler}, yet we use $\chi$ for both. Which we mean will be determined by what kind of object we are applying $\chi$ to. 

This definition is consistent with the definition of the Euler characteristic of a polyhedron: If the largest cells of a simplicial complex are 2-cells, or two-dimensional faces, then the definitions are identical. 

It turns out that there is a straightforward way of converting a simplicial complex into a category. We will find that the Euler--Leinster characteristic of the resulting category will be the same as the Euler characteristic of the simplicial complex. 
\begin{defn}
Let $\sigma$ be a simplicial complex. Then the \emph{category generated by $\sigma$}, or $\cat(\sigma)$, is the category whose objects are the cells of $\sigma$. If $\alpha$ and $\beta$ are cells of $\sigma$, then $|(\alpha, \beta)| = 1$ if $\alpha \subseteq \beta$ and 0 else.
\end{defn}
Note that a simplicial complex can be converted into a poset, where the cells are ordered by inclusion. Recall that we also defined $\cat(X)$ for a poset $X$ in Definition \ref{cat_poset}. If $\sigma$ is a simplicial complex, and $X$ is the corresponding poset, then $\cat(\sigma) = \cat(X)$.

The notation for $\cat(\sigma)$ is a little tricky. If $\sigma$ is a simplicial complex, and $\alpha \in \sigma$ is a simplex, then we write $\cat(\alpha)$ to denote the object in $\cat(\sigma)$ corresponding with the simplex $\alpha \in \sigma$. We consider an example.
\begin{examp}
Let $\sigma = \{a, b, ab\}$. Then $\cat(\sigma)$ has three objects and five morphisms, including the three identity morphisms.
\begin{figure}[h]
\[
\includegraphics{../Figures/simp_cat.pdf}
\]
\caption{$\sigma$ and $\cat(\sigma)$. Recall that we omit identity morphisms when drawing a category.}
\end{figure}
\end{examp}

To prove that this operation preserves the Euler characteristic, we start with two lemmas about $\mu$, the M\"obius function on a category. First, recall that categories do not necessarily admit M\"obius inversion. So $\mu$, a priori, might not exist on a given category. However, the following lemma will rule this possibility out in the case of the category generated by a simplicial complex.
\begin{lemma}\label{mu_exists}
Let $\sigma$ be a simplicial complex. Then $\cat(\sigma)$ admits M\"obius inversion.
\end{lemma}
\begin{proof}
Totally order the simplices of $\sigma$ respecting the partial order of number of elements (so if $\alpha, \beta \in \sigma$ and $|\alpha| < |\beta|$, then $\alpha < \beta$ in the total order). Write down the matrix $\zeta$ with the rows and columns in that order. Then $\zeta$ will be upper triangular, so it will be invertible. So $\mu = \zeta^{-1}$ exists.
\end{proof}
Now that we know $\mu$ exists, we can inductively determine its value, similarly to how we have done so in previous sections with M\"obius functions on partially ordered sets.
\begin{lemma}
\label{mu_lemma}
Suppose $\sigma$ is a simplicial complex, and $\alpha \subseteq \beta$ are simplices in $\sigma$. Let $\mu$ denote the M\"obius function on $\cat(\sigma)$. Then $\mu(\cat(\alpha), \cat(\beta)) =(-1)^{|\beta - \alpha|} = (-1)^{|\alpha|} (-1)^{|\beta|}$ (note that $|\alpha|$ simply denotes the size of $\alpha$ as a set).
\end{lemma}
\begin{proof}
We use induction. Base case: $\alpha = \beta$, and $\mu(\alpha, \beta)=1$.

For the inductive step, suppose that if $|\beta-\alpha| \leq k$, the result is true.

Suppose that $|\beta - \alpha| = k+1$. We know by Definition \ref{cat_mob} that, because $\alpha \neq \beta$,
 \begin{align*}
\sum_{\alpha \subseteq \sigma \subseteq \beta} \mu(\alpha, \sigma)&= 0\\
\mu(\alpha, \beta) &=  - \sum_{\alpha \subseteq \sigma \subset \beta} \mu(\alpha, \sigma)\\
&= - \sum_{\gamma \subset (\beta - \alpha)} \mu(\alpha, \gamma \cup \alpha) 
\end{align*}
Note that the above unions are always disjoint. So, by the inductive hypothesis, we can rewrite it:
\begin{align*}
\mu(\alpha, \beta) &= - \sum_{\gamma \subset (\beta - \alpha)'} (-1)^{|\gamma \cup \alpha - \alpha|} \\ %& \text{inductive hypothesis}\\
&= - \sum_{\gamma \subset (\beta - \alpha)'} (-1)^{|\gamma|}\\ % &\text{by disjointness}\\
\end{align*}
We can add and subtract $(-1)^{|\beta - \alpha|}$ and apply a binomial identity to finish the proof:
\begin{align*}
\mu(\alpha, \beta) &= - \sum_{\gamma \subseteq (\beta - \alpha)} (-1)^{|\gamma|} + (-1)^{|\beta - \alpha|} \\ 
&= (-1)^{|\beta - \alpha|} -  \sum_{i = 0}^{|\beta - \alpha|}(-1)^i{|\beta- \alpha| \choose i}\\ 
&= (-1)^{|\beta - \alpha|} \\ 
\end{align*}
\end{proof}
We are now ready to prove that the two definitions of the Euler characteristic coincide.
\begin{thm}
\label{consistentEuler}
Let $\sigma$ be a simplicial complex. Then 
\[
\chi(\cat(\sigma)) = \chi(\sigma).
\]

\end{thm}
\begin{proof}
Recall that by Definition \ref{cat_euler}, 
\[
\chi(\cat(\sigma)) = \sum_{\gamma \in \sigma} \sum_{\alpha \in \sigma} \mu(\cat(\alpha), \cat(\gamma)).
\]
We know that if $\alpha \not \subseteq \gamma, \mu(\alpha, \gamma) = 0$. So
\[
\chi(\cat(\sigma)) = \sum_{\gamma \in \sigma} \sum_{\emptyset \neq \alpha \subseteq \gamma}\mu(\cat(\alpha), \cat(\gamma)).
\]
So we have
\begin{align*}
\chi(\cat(\sigma)) &= \sum_{\gamma \in \sigma} \sum_{\emptyset \neq \alpha \subseteq \gamma}\mu(\cat(\alpha), \cat(\gamma))\\
&= \sum_{\gamma \in \sigma} \sum_{\emptyset \neq \alpha \subseteq \gamma}(-1)^{|\gamma| - |\alpha|} &\text{by \ref{mu_lemma}}\\
&= \sum_{\gamma \in \sigma}(-1)^{|\gamma|} \sum_{i = 1}^{|\gamma|} (-1)^i{|\gamma| \choose i}\\
&= \sum_{\gamma \in \sigma} (-1)^{|\gamma|} \\
&= \chi(\sigma)
\end{align*}
\end{proof}
The fact that $\chi(\cat(\sigma)) = \chi(\sigma)$ is comforting. It justifies calling using $\chi$ in Definition \ref{cat_euler}, and calling this construction a type of Euler characteristic.

\section{Euler--Leinster Characteristic Examples}
%TODO: Groups? Posets are boring... categories which look like other things?

\end{chapter}
\begin{chapter}{Metric Space Magnitude}\label{chap_magnitude}
\section{Motivation: Hierarchical Clustering}
A common task in data analysis is clustering. Given a set of data points, it is often desirable to decide how many ``groups'' those data points fall into. Of course, there is usually no clear-cut definition of which points are in groups and which are not. 

The most straightforward way to cluster points is based on some notion of distance between them. If we assign a distance to each pair of points, then we can cluster points which are close together. But there is often no clear threshold for which points are and aren't close enough together to be in the same cluster. It all depends on the scale of structure we are looking for. One common technique for avoiding choosing a scale is what is called \emph{hierarchical clustering}. 

The idea of hierarchical clustering is that instead of choosing a scale, we capture the data's structure at a variety of scales.
\begin{figure}
\[
\includegraphics[width = 0.4\paperwidth] {../Figures/multiscale_structure.png}
\]

\caption{A graphical representation of hierarchical clustering.}
\label{fig:hier}
\end{figure}

We can see in Figure \ref{fig:hier} that as we ``zoom out'', the structure of this data changes. First the data has first eight clusters of one point each, then four clusters of two points, and finally two clusters of four points. If we kept going, we would end up with a single cluster containing all of the points.

There are many methods of hierarchal clustering. In the above example, explicit clusters are formed by grouping points which are near each other. Methods like this are common; see \cite{Carl1}. We will explore an alternate method of clustering, in which we do not explicitly group points, but instead estimates the number of clusters using a real, rather than integer, value. We will use a generalization of the Euler--Leinster characteristic from Chapter \ref{chap_euler}. We start by building some important definitions.

A metric space is a space with a notion of distance between pairs of points. 
\begin{defn}
A \textit{metric space} $(M,d)$ is a set $M$ together with a function  $d:M \times M \rightarrow \Rr$ (the \emph{metric}), such that the following are satisfied for all $a,b,c \in M$:
\begin{itemize}
\item $d(a,b) \geq 0$, with equality $\iff a = b$;
\item $d(a,b) = d(b,a)$; and
\item $d(a,b) + d(b,c) \geq d(a,c)$.
\end{itemize}
We sometimes write just $M$ if the metric is unambiguous.
\end{defn}

In almost all forms of hierarchical clustering, we represent data as a metric space. Here, we will actually define a slightly different structure which will be quite similar to a metric space, and use that instead. However, it's useful to keep in mind this more fundamental definition of a set with distance.

In the next section, we expand the Euler--Leinster characteristic to a generalization of categories called \emph{enriched categories}. We will connect this back to data and metric spaces in subsequent sections.
\section{Enriched Categories}
In Chapter \ref{chap_euler}, we defined a structure called a category. It turns out that this construction - which we will now call an \emph{ordinary} category - can be generalized further, to become an \emph{enriched} category. The idea is that we replace the sets of morphisms with objects from an ordinary category of a certain type, and adapt the definition of composition accordingly. 

In order to define the next construction, we need to define a notion of isomorphism between objects in a category.
\begin{defn}
Let $C$ be an ordinary category. Let $x,y \in C$ be two of $C$'s objects. Suppose that $f \in C(a,b), g \in C(b,a)$ are morphisms such that $g \circ f = 1_x, f \circ g = 1_y$. Then we say
\begin{itemize}
\item $f = g^{-1}, g = f^{-1}$
\item $f$ and $g$ are \emph{inverses} of each other
\item $f$ and $g$ are both \emph{isomorphisms}
\item $f \cong g$
\item $a$ and $b$ are \emph{isomorphic}.
\end{itemize}
\end{defn}

If two objects $a$ and $b$ are isomorphic, then for any morphism to or from either $a$ or $b$, we can use the isomorphisms to construct an ``equivalent'' map to or from the other object. As a result, $a$ and $b$ are equivalent in nearly every way. The next several definitions will replace requirements for - for example - associativity with requiring associativity up to isomorphism. This is common in category theory, and makes definitions more versatile, although it also makes them seem much more complicated at first. 

\begin{defn}\label{def_monoid}
A \emph{monoidal category} is an ordinary category $A$, together with
\begin{itemize}
\item A \emph{monoidal product} $\otimes : A \times A \rightarrow A$;
\item A \emph{unit object} $u \in A$;
\end{itemize}
such that
\begin{enumerate}[(1)]
\item For each ordered triple of objects $x,y,z \in A$, $(x \otimes y) \otimes z \cong x \otimes (y \otimes z)$
\item For each $x \in A$, $u \otimes x \cong x \cong x \otimes u$,
\item The function $\otimes$ is \emph{functorial}. This essentially means that if $x_1, x_2, y_1, y_2 \in M, f \in C(x_1, x_2), g \in C(y_1, y_2)$, then there exists a corresponding morphism in $C(x_1 \otimes y_1, x_2 \otimes y_2)$. We call that morphism $f \otimes g$, and require that it behave as we would expect (e.g. if $f,g$ are both isomorphisms, $f \otimes g$ also is).
\item A variety of commutation/coherence conditions are satisfied. Here, we did not name the isomorphisms required in points (1) and (2), but in a more thorough definition they would be named. It would then be required that the various isomorphisms commute appropriately, along with other technical details.
\end{enumerate} 
\end{defn}

There are a variety of commutation requirements which we are not considering in detail here. The property of being functorial is also more involved. For a full account, see \cite{Kelly1}.

We turn to several examples of monoidal categories.
\begin{examp}
The category $\catname{set}$ is the prototypical example of a monoidal category. In that case, the monoidal product $\otimes$ is the Cartesian product on sets (hence the notation $\otimes$ for the monoidal product), and $u = \{*\}$, some set with a single element.

Here, we can consider what the required isomorphisms look like. It's required that for any set $X \in \catname{set}, \{*\} \otimes X \cong X$. Elements of $\{*\} \otimes X$ are of the form $(*, x)$ for $x \in X$. Also recall that an isomorphism is an invertible morphism, and that in \catname{set}, morphisms are just maps. So, because $|X| = |\{*\} \otimes X|$, the two sets are isomorphic in \catname{set}. The associativity isomorphisms are similar.
\end{examp}
\begin{examp}\label{cat_real}
A monoidal category which will be very important here is $\Rr^{\geq 0}$, where there is exactly one morphism $f \in \Rr(a,b)$ if $a \geq b$. The monoidal product is addition, and the unit object is 0. The requirements are easily checked. We will refer to this category as $D$, because soon it will be used to represent distances.

In this case, there are no need for isomorphisms, because for any $a,b,c \in \Rr^{\geq 0}, (a + b) + c = a + (b + c)$, and $0 + a = a = a + 0$. A monoidal category like this, where all the associativity and identity isomorphisms are the identity, is called a \emph{strict} monoidal category. In fact, in $D$, no two distinct elements are isomorphic. 
\end{examp}
We are now ready to define an important generalization on ordinary categories. This definition is highly technical, and a full understanding of it is not necessary for this investigation. So we will not spend much time exploring examples or intuition beyond the specific applications we need.
\begin{defn}[\cite{Nog1}]\label{en_cat}
Let $M$ be a monoidal category. Then a \emph{category enriched by $M$} consists of
\begin{itemize}
\item A set $C_0$ of objects;
\item For each $a,b \in C_0$, an \emph{object of morphisms} $C(a,b) \in M$;
\item For each $a,b,c \in C_0$, a \emph{composition morphism} $\circ_{abc}:  (C(b,c) \otimes  C(a,b)) \rightarrow  C(a,c)$;
\item For each $a \in C_0$, an \emph{identity element morphism} $i_a : u \rightarrow C(a,a)$, where $u$ is the unit object in $M$,
\end{itemize}
such that all the $i_1, \circ_{abc}$ commute appropriately.
\end{defn}
As with ordinary categories, we usually will omit the subscripts on the composition morphisms. They will be implied by context.
\begin{rmk}
For a full discussion of enriched categories, including a complete description of the commutation requirements, see \cite{Kelly1}. 
\end{rmk}

\section{Norms and the Euler--Leinster Characteristic}
The following constructions can be found in \cite{Nog1, Lein4}.

We would like to generalize the weightings and Euler--Leinster characteristic from Chapter \ref{chap_euler} so we can use them on enriched categories. Recall that in Definition \ref{def_weighting}, we used the function $\zeta$ to require that for $\omega$ to be a weighting on a category $C$, for all $a \in C$,
\[
\sum_{b \in C}|C(a,b)|\omega(b) = 1.
\]

The problem is that given an enriched category over some monoidal category $M$, for a pair $a,b$, $C(a,b)$ is no longer a set but an element of $M$. So it is no longer possible to use the function $|\cdot|$ as defined on sets. Instead, we must define our own \emph{norm} function on $M$. We define this as follows:
\begin{defn}\label{def:norm}
Let $(M,\otimes)$ be a monoidal category. Then the function $| \cdot |: M \rightarrow \Rr$ is a \emph{norm} on $M$ if, for any $a,b\in M$, and if $u$ is the unit object of $M$, it satisfies
\[
|a \otimes b| = |a||b|,
\]
\[
|u| = 1.
\]
\end{defn}

The usual cardinality function on the monoidal category \catname{set} does satisfy these requirements - for any two finite sets $A,B, |A \otimes B| = |A||B|$, and $|\{*\}|=1$.

It's not clear why we require that the norm function be multiplicative. We will see a reason why in Section \ref{sec:prod}, where we will explore a notion of a product category. There, it will be necessary that norms behave in this way.

We can consider some examples of norms.
\begin{examp}\label{exp_norm}
Let $D$ be the monoidal category of non-negative real numbers defined in example Example \ref{cat_real}. Then, for all $t \in \Rr$, the function $|\cdot|:D \rightarrow \Rr$ defined by 
\[
|x| = e^{-tx}
\]
is a norm on $D$. We can see that it satisfies the requirements:
\[|0| = 1
\]
\[|x + y| = e^{-tx + ty} = |x||y|.
\]
We will explore this norm in detail shortly.
\end{examp}

Now that we know how to replace the norm function on sets, we can redefine the $\zeta$ function, M\"obius inversion, weightings, and the Euler--Leinster characteristic on enriched categories. We go through these definitions, and then we will explore their ramifications.
\begin{defn}
Let $C$ be a finite category enriched over a monoidal category $M$. Suppose that $|\cdot|$ is a norm on $M$. Then we define the similarity matrix $\zeta$ on $C$ by
\[
\zeta_{|\cdot|}(a,b) = |C(a,b)|.
\]
\end{defn}

\begin{defn}[\cite{Nog1}]
Let $C$ be a finite category enriched over a monoidal category $M$. Let $|\cdot|$ be a norm on $M$, and let $\zeta_{|\cdot|}$ be the similarity matrix defined above. Then the M\"obius function on $C$, denoted $\mu_{|\cdot|}$, is defined by 
\[
\mu_{|\cdot|} = \zeta_{|\cdot|},
\]
if $\zeta$ is invertible. If $\mu$ exists, we say that $C$ admits M\"obius inversion.
\end{defn}

\begin{defn}
Let $C$ be a finite category enriched over a monoidal category $M$, with a specificed norm $|\cdot|$. Then $\omega$ is a \emph{weighting} on $C$ if, for all $a \in C$, 
\[
\sum_{b \in C}|C(a,b)|\omega(b) = 1,
\]
or, equivalently, if
\[
\sum_{b \in C}\zeta_{|\cdot|}(a,b)\omega(b) = 1.
\]
\end{defn}

\begin{defn}[\cite{Nog1}]
Let $C$ be a finite category enriched over a monoidal category $M$, with a specified norm $|\cdot|$. Let $\omega$ be a weighting on $C$. Then the \emph{Euler--Leinster characteristic} of $C$, denoted $\chi_{|\cdot|}(C)$, is defined by
\[
\chi_{|\cdot|}(C) = \sum_{a \in c} \omega(a).
\]
\end{defn}
In the notation just introduced, we will sometimes omit the subscripts specifying the norm, if the norm is implied.
\begin{rmk}
When $C$ is an enriched category, $\chi(C)$ is sometimes called the \emph{magnitude} of $C$ (for example, in \cite{Lein4}). We reserve that terminology for a particular class of enriched categories which we will soon introduce, and continue to refer to the general construction as the Euler--Leinster characteristic or the Euler characteristic of $C$.
\end{rmk}
\begin{rmk}
As before, it is easy to show by rearranging terms that the value of $\chi_{|\cdot|}(C)$ is independent of the specific weighting chosen. However, it does depend on the norm we choose. This will become very important later.
\end{rmk}
%TODO: Examples of euler characteristics of non-metric space enriched categories

\section{Magnitude}
Recall that the motivation for this section was data clustering. So we would like to use the technology from the previous section to process data in the form of a metric space. We will use a particular enriched category for that purpose, called a \emph{generalized metric space}. This is a category enriched over a monoidal category whose objects are non-negative real numbers, allowing us to preserve a notion of distance.  

\begin{defn}[Lawvere, \cite{Lawv1}]\label{gen_met}
Let $D$ be the monoidal category of non-negative real numbers defined in Example \ref{cat_real}. Let $C$ be a category enriched over $D$. Then we call $C$ a \emph{generalized metric space}.
\end{defn}
There is a lot of technology buried in that definition, and it can be difficult to untangle definitions involving enriched categories. Luckily, because of the simplicity of $D$, we can simplify the definition of a generalized metric space to one which does not depend on the definition of an enriched category.
\begin{thm}\label{direct_def}
A generalized metric space $C$ can be defined equivalently by the following:
\begin{itemize}
\item A collection $C_0$ of points (or objects); and
\item For each ordered pair $a,b \in C_0$, a non-negative real number $C(a,b)$
\end{itemize}
such that, for all $a,b \in C_0$,
\[C(a,a) = 0,
\]
\[\hspace{1cm} C(a,b) + C(b,c) \geq C(a,c).
\]
\end{thm}
\begin{proof}
We will start with the definition from Definition \ref{gen_met}, and argue that it is equivalent to the above. As defined originally, a generalized metric space consists of a set of objects $C_0$, with a non-negative real number $C(a,b)$ between any ordered pair of elements $a,b \in C_0$. So what remains is to show that the composition requirements from Definition \ref{en_cat} are equivalent to the conditions in Definition  \ref{direct_def}. 

Let $C$ be a generalized metric space as defined in Definition \ref{gen_met}. Then Definition \ref{en_cat} requires that there be a morphism $i_a:0 \rightarrow C(a,a)$ for any $a \in C$. This exists if and only if $C(a,a) \leq 0$, which is equivalent to saying that $C(a,a)=0$. So the existence of $i_a$ for all $a$ is equivalent to the first requirement above.

It is also required that for any $a,b,c \in C$, there is a map $\circ_{abc}: C(b,c) + C(a,b) \rightarrow C(a,c)$. This is equivalent to saying that $C(b,c) + C(a,b) \geq C(a,c)$, which is the second requirement. So the definitions are indeed identical.
\end{proof}

This makes it clear why we call this construction a generalized metric space. It is a metric space, but without the restriction that distinct points have nonzero distance, and without the requirement of symmetry. Relaxing these constraints allows us to cleanly express this as an enriched category.  

Generalized metric spaces are, of course, more general than ordinary metric spaces. So we can always express a metric space as a generalized metric space. This allows us to find the Euler--Leinster characteristic of a finite set of data points in a metric space. But first we must decide on the norm we will use on $D$, our monoidal category, and to do that, we take a step back and consider why we are doing this.

We would like to estimate the number of clusters in a dataset. To justify the rather strange way in which we'll do that, we can turn to an analogy. Suppose each data point represented a spectator at a sports game, some of whom are sitting in groups. Each group would like to attain a particular level of excitement based on the energy levels of its members. But each group only needs to have the same total level of excitement, and the individual energy levels of the members of group add together to determine the group's excitement. The question is what level of energy each fan should have so that their group has the appropriate total level of excitement. 
\begin{figure}[h]
\[
\includegraphics[width=.5\textwidth]{../Figures/no_weights.pdf}
\]
\caption{Seven sports spectators in groups}
\label{fig:sports}
\end{figure}
In Figure \ref{fig:sports}, we have a group of two on the left, a group of one in the middle, and a group of three on the right. The goal is to assign each individual an energy level so that all three groups' excitement levels are equal to 1. This isn't hard to do - we give one example in Figure \ref{fig:sports_labeled}.
\begin{figure}[h]
\[
\includegraphics[width=.5\textwidth]{../Figures/initial_weights.pdf}
\]
\caption{A possible allocation of energy}
\label{fig:sports_labeled}
\end{figure}
Given these values, it's simple to count how many groups there are. Each group has total excitement equal to 1, so if we add the energy levels of each individual together, we will obtain the number of groups (in this case, 3).

Of course, this is circular. We had to decide ahead of time which individuals were grouped together so that we could find the values to assign them. So we use a modification of this method: Each individual must have an excitement level of 1, and the excitement level of an individual is the sum of their own energy level together with a weighted sum of the levels of those around them. The closer two individuals are to each other, the more each of their energy levels contributes to the other's excitement. 

Let $M$ be the set of spectators. Suppose that for each spectator $a \in M, \omega(a)$ is the energy level of $a$. Suppose that for any $a,b \in M$, $\zeta(a,b)$ is a measure of how close together those two people are. So if $a$ and $b$ are very close together, $\zeta(a,b)$ approaches 1, and as they move apart it approaches zero. Then we are asking that the following be true for each spectator $a \in M$:
\[
\sum_{b \in M}\zeta(a,b)\omega(b).
\]

This looks exactly like the requirement (from Definition \ref{def_weighting}) for $\omega$ to be a weighting on a category $M$! So adding up values of $\omega$ is the same as finding the Euler--Leinster characteristic of $M$, assuming we can properly encode the structure of the spectators into a category. This is the key motivation behind the use of the Euler characteristic to count clusters.

We now move from our analogy back into the realm of categories. We will model the situation using a generalized metric space. We need a norm $|\cdot|$ on $D$ such that $|x|$ will be close to 1 as $x$ gets small, and will approach zero as it gets large. 

Recall that in Example \ref{exp_norm}, we observed that the function $|x|=e^{-tx}$ is a norm on $D$. For any positive $t$, this is a decreasing function. 
\begin{figure}[h]
%TODO: Replace plot
\[
\includegraphics[width = 0.5\paperwidth]{../Figures/exp.pdf}
\]
\caption{A graph of $e^{-x}$ against $x$.}
\end{figure}
And the larger the value of $t$, the quicker the function decays. In fact, we will see that if we use this family of norms, we can view $t$ as a ``zooming'' parameter, which allows us to consider the data at different scales.

We have a special name for the Euler characteristic of a generalized metric space which results from this particular norm:
\begin{defn}[\cite{Lein4}]
Let $M$ be a generalized metric space. Let $|\cdot|_t:D \rightarrow \Rr$ be defined by
\[
|x|_t = e^{-tx}.
\]
If it exists, let $\chi_t(M)$ be the Euler characteristic resulting from $|\cdot|$. Then $\chi_t(M)$ is called the \emph{$t$-magnitude} of $M$, denoted $\magn_t(M)$. We will refer to the 1-magnitude as simply the magnitude of $M$, denoted $\magn(M)$.
\end{defn}

It makes sense that the magnitude of a metric space should capture the number of clusters it contains. We can consider a simple example to see how it behaves.
\begin{examp}\label{ex:2pt}
Consider the space $M_2$ of two points separated by distance $1$, shown in Figure \ref{fig:2pt}.
\begin{figure}[h]
                        \[
\begin{tikzpicture}[auto, node distance = 3cm, main node/.style={dot}]

\node[circle, draw, fill=black,
                        inner sep=0pt, minimum width=4pt](1) at (0,0) {};
\node[circle, draw, fill=black,
                        inner sep=0pt, minimum width=4pt](2) at (4,0) {};

\draw[decoration={brace,raise=7pt},decorate] (1) -- (2) node[midway, above = 10pt] {1};

\end{tikzpicture}\]
\caption{$M_2$, a simple generalized metric space with two points.}
\label{fig:2pt}
\end{figure}
If $t$ indeed behaves as a zooming parameter, then we would expect that for small values of $t$, $\chi_t(M_2)$ should have a single cluster, because we are not zoomed in. But for large values of $t$, the number of clusters should approach 2, as the two points fully separate from each other.

In this case, for $t \in (1,\infty),$ we can let the weighting of each point equal
\[\frac{1}{1+e^{-t}}.\]
It is straightforward to check that this is indeed a valid weighting. So for each $t$,
\[
 \chi_t(M_2) = \frac{2}{1+e^{-t}}.
\]

We are varying a parameter, $t$, in the exponent of the norm definition. So it is reasonable to expect that the behavior of the magnitude as $t$ varies will be logarithmic. If we plot $\chi_t(M_2)$ against $t$, we can see that it behaves just as we expect (see Figure \ref{fig:2pt_mag}).
\begin{figure}[H]
\[
\makebox[\textwidth][c]{\includegraphics[width = 0.9\paperwidth]{../Figures/2_point_.png}}
\]
\caption{A plot of $\chi_t(M_2)$ against $\log(t)$, where $t$ is the scale factor.}
\label{fig:2pt_mag}
\end{figure}
\end{examp}
\section{Product Spaces and Approximations}\label{sec:prod}
In Figure \ref{fig:3+4}, we see two sets of points, $M$ and $N$. $M$ has three points while $N$ has four.

\begin{figure}[h]
\[
\includegraphics{../Figures/4+3.pdf}
\]
\caption{Two sets of points.}
\label{fig:3+4}
\end{figure}
In Figure \ref{fig:3x4}, we see another set of points, $A$.
\begin{figure}[h]
\[
\includegraphics{../Figures/4x3.pdf}
\]
\caption{A set of points $A$: A copy of $N$ for each point in $M$.}
\label{fig:3x4}
\end{figure}
The set $A$ is what you would get if you replaced each point in $M$ with the set $N$. 

Suppose we choose a scale, and somehow count the number of clusters in $N$ and in $M$. If the scale is very small -- if we ``zoom in'' so that points must be very close together before they are considered to be in the same cluster -- then every point will be in its own cluster. So $N$ will have four clusters, $M$ three, and $A$ will have 12.

If we choose a middle scale, then it might be that the points in $N$ are clustered together, but $M$ remains discrete. Then $N$ has one cluster, $M$ has three, and $A$ has four, as each copy of $N$ forms a cluster.

Finally, if our scale is large enough that $M$ is grouped into a single cluster, all three sets of points will have a single cluster.

In all three cases, the number of clusters in $A$ is the product of the number of clusters in $M$ and in $N$. It appears that this remains the case as we change our scale. 

$A$ is an approximation of what is called a \emph{Cartesian product category}. In this section, we define the construction formally, and explore how the Euler--Leinster characteristic behaves when two categories are combined in this way. In particular, we would like for the magnitude of a product space to be the product of the magnitudes of the original spaces - this is how the number of clusters seems like it should behave when we look at examples like $A$ above.

Not all enriched categories can be used to construct product spaces. We first must define a restriction on the definition of a monoidal category.

\begin{defn}
Let $M$ be a monoidal category with monoidal product $\otimes$. Suppose that for all $a,b \in M, a \otimes b \cong b \otimes a$. Then $M$ is a \emph{symmetric monoidal category}.
\end{defn}

We add this restriction to allow us to prove the following lemma:

\begin{lemma}
Let $M$ be a symmetric monoidal category. Let $P,Q,R,S,T,U$ in $M$ be given. Suppose that we have morphisms
\begin{align*}
f&:(P \otimes Q) \rightarrow T\\
g&:(R \otimes S) \rightarrow U.
\end{align*}
Then, if $f \otimes g$ is the morphism required by the functoriality requirement of Definition \ref{def_monoid}, we have
\[
(f \otimes g):(P \otimes Q) \otimes (R \otimes S) \rightarrow T \otimes U.
\]
\end{lemma}
\begin{proof}\label{prod_morph}
Using the symmetry of $\otimes$, we can rearrange:
\[
(P \otimes Q) \otimes (R \otimes S) \cong (P \otimes R) \otimes (Q \otimes S).
\]
Then Definition \ref{def_monoid} tells us that the desired morphism exists.
\end{proof}
This extra requirement is necessary for the product of two enriched categories to exist. In particular, the composition maps do not necessarily exist if the monoidal category is not symmetric. We can now define that:

\begin{defn}

Let $A,B$ be categories both enriched over a symmetric monoidal category $M$. Fix a norm on $M$. Then the \emph{Cartesian product} (or simply the \emph{product}) of $A$ and $B$, denoted $A \times B$, is the category enriched over $M$ defined as follows:
\begin{itemize}
\item The set of objects, $(A \times B)_0$, is $\{(a,b):a \in A_0, b \in  B_0\}$
\item For $(a,b), (c,d) \in A_0, C((a,b), (c,d)) = C(a,b) \otimes C(c,d)$.
\item The unit object is $u_A \otimes u_B$, where $u_A, u_B$ are the units of $A,B$ respectively
\item The composition and identity morphisms are defined using Lemma \ref{prod_morph} (the details are not important for this discussion, but they are straightforward).
\end{itemize}
\begin{rmk}
The Cartesian product of two categories enriched over $M$ is indeed a category enriched over $M$. There are many requirements to check, and we do not go through all of them here.
\end{rmk}
\end{defn}

We consider some examples of simple product categories. Note that \catname{set} is a symmetric monoidal category, so ordinary categories are eligible to be used as factors of product categories. 
\begin{examp}
Let $A$ be the following category:
\begin{figure}[H]
\[
\includegraphics{../Figures/cat_2_directed.pdf}
\]
\caption{A category with two points and three morphisms. Recall that we omit the identity morphisms when we draw categories.}
\end{figure}
Then $A \times A$ looks like this:
\begin{figure}[H]
\[
\includegraphics{../Figures/cat_2_prod.pdf}
\]
\caption{$A \times A$.}
\end{figure}
To see why, recall that the monoidal product on \catname{set} is the direct product. So consider, for example, $(a,a)$ and $(a,b)$. There is one morphism from $a$ to $a$ and one from $a$ to $b$. So the direct product of the hom-sets also has one element, which is why there is one arrow from $(a,a)$ to $(a,b)$. On the other hand, because there is no morphism $b$ to $a$, there is also no morphism from $(a,b)$ to $(a,a)$. 
\end{examp}
\begin{examp}\label{met_prod_ex}
We will be focusing on products of generalized metric spaces. So we consider an example of that. Let $A, B$ each be generalized metric spaces with two points. Label the objects of $A$ as $a_1, a_2$, and the objects of $B$ as $b_1, b_2$.
\begin{figure}[h]
\[
\includegraphics{../Figures/gen_met_small.pdf}
\]
\caption{$A$ and $B$}
\end{figure}
Suppose that $A(a_1, a_2) = 1, B(b_1, b_2) = 2$. Recall that the monoidal product is $+$. So the product category $A \times B$ looks like this:
\begin{figure}[H]
\[
\includegraphics[width=.4\paperwidth]{../Figures/gen_met_small_prod}
\]
\caption{$A \times B$. It is impossible to draw this product space to scale because of the way distances behave under the Cartesian product.}
\end{figure}

We can observe the same behavior here as in the initial example. Suppose we used some clustering scheme where points within a threshold distance $t$ of each other were put into the same cluster. This is not how magnitude works, but it is easier to visualize. Then we have three cases:
\begin{description}
\item[$t < 1$:] $A$ has two clusters, $B$ has two clusters, and $A \times B$ has four clusters.
\item[$1 \leq t < 2$:] $A$ has one cluster, $B$ has two clusters, and $A \times B$ has two clusters.
\item[$t \geq 2$:] $A, B$, and $A \times B$ each have just one cluster (note that $(a_1, b_2)$ and $(a_2, b_1)$ are in the same cluster in $A \times B$ even though their distance is 3, because both are in the same cluster as $(a_1, b_1)$).
\end{description}
So in all cases, the number of clusters in $A \times B$ is the product of the numbers of clusters in $A$ and $B$. We will see that if we count clusters using the magnitude of the space, the same thing happens.
\end{examp}

Magnitude is obtained using $\zeta$, the similarity matrix. So we are interested in how the similarity matrix behaves under the Cartesian product. It turns out that the similarity matrix of a product category is obtained using the matrix \emph{Kronecker product}, not coincidentally (but confusingly) written using $\otimes$. Because this is the same notation used for the monoidal product, we will use non-standard notation for the Kronecker product: $\kron$.
\begin{defn}
Let $A$ be an $m \times n$ matrix and $B$ be a $q \times r$ matrix. Then the \emph{Kronecker product} of $A$ and $B$, denoted $A \kron B$, is the $mq \times nr$ matrix defined by the following block matrix:
\[
A \kron B = 
\begin{bmatrix}
A_{11}B & A_{12}B & \cdots & A_{1n}B \\
A_{21}B & A_{22}B & \cdots & A_{2n}B \\
\vdots & \vdots & \ddots & \vdots \\
A_{m1}B & A_{m2}B & \cdots & A_{mn}B
\end{bmatrix}.
\]
\end{defn}
\begin{examp}
Let 
\begin{align*}
A &= \begin{bmatrix}
1 & 2 \\
3 & 1
\end{bmatrix},\\
B &= \begin{bmatrix}
2  \\
1 
\end{bmatrix}.
\end{align*}
Then
\begin{align*}
A \otimes B &= 
\begin{bmatrix}
B & 2B\\
3B & B
\end{bmatrix}\\
&= \begin{bmatrix}
2 & 4 \\
1 & 2 \\
6 & 2\\
3 & 1
\end{bmatrix}.
\end{align*}
\end{examp}
\begin{rmk}\label{kron_prod}
It is straightforward to show that, for any appropriately sized matrices $A, B, C, D$,
\[
(A \otimes B) (C \otimes D) = AC \otimes BD.
\]
We will use that fact soon.
\end{rmk}
We show that this definition is useful by connecting it immediately to the Cartesian product of categories.

\begin{lemma}\label{prod_kron}
Let $A, B$ be categories both enriched over a symmetric monoidal category $M$. Fix some norm on $M$. Let $\zeta_A, \zeta_B$ be the similarity matrices of $A,B$ under the chosen norm. If $a_1,\dots,a_m$ and $b_1,\dots,b_n$ are the orders chosen for the objects of $A$ and $B$ when constructing $\zeta_A, \zeta_B$, then fix the order \[a_1b_1, a_1b_2, \dots, a_1b_n, a_2b_1, \dots, a_mb_n\] for $A \times B$. Then
\[
\zeta_{A \times B} = \zeta_A \kron \zeta_B.
\]
\end{lemma}
\begin{proof}
Let $a_1, a_2 \in A, b_1, b_2 \in B$ be given. Recall that norms are multiplicative (see Definition \ref{def:norm}). Then the entry of $\zeta_{A \times B}$ corresponding with $(a_1, b_1), (a_2, b_2)$ should contain 
\begin{align*}
|(A \times B)((a_1, b_1), (a_2, b_2))|& = |A(a_i, a_j) \otimes B(b_k, b_l)|\\
&= |A(a_i, a_j)| |B(b_k, b_l)|
\end{align*}
which is the value of the corresponding entry of $\zeta_A \kron \zeta_B$.
\end{proof}

Given this lemma, we can prove that the behavior we saw in Example \ref{met_prod_ex} occurs when we count clusters using the magnitude function. In fact, it turns out the for any category enriched over a symmetric monoidal category, the Euler--Leinster characteristic is multiplicative over the product. 

\begin{thm}
Let $A, B$ be categories both enriched over a monoidal category $M$ . Fix some norm on $M$. Then, if $\chi(A), \chi(B)$ both exist,
\[
\chi(A \times B) = \chi(A)\chi(B).
\]
\end{thm}
\begin{proof}
Let $\zeta_A, \zeta_B$ be the similarity matrices of $A,B$. Then, by Theorem \ref{prod_kron}, the similarity matrix of $A \times B$ is $\zeta_A \kron \zeta_B$. 

Because both Euler characteristics exist, there must exist a valid weighting on each category. Let $\omega_A, \omega_B$ be weightings on $A, B$ respectively. Let $1^{|A|}, 1^{|B|}$ be the ones vectors of lengths $|A|$ and $|B|$. Then, using the fact from Remark \ref{kron_prod}, we have
\begin{align*}
1^{|A|}&= \zeta_A\omega_A \\
1^{|B|}&= \zeta_B\omega_A \\
1^{|A||B|} &= \zeta_A\omega_A \kron \zeta_B\omega_B\\
&= (\zeta_A \kron \zeta_B)(\omega_A \kron \omega_B)
\end{align*}
So the vector $\omega_A \kron \omega_B$ is a weighting on $A \times B$. We have:
\begin{align*}
\chi(A \times B) &= \sum_{\alpha \in A \otimes B}(\omega_A \kron \omega_B)(\alpha) \\
&= \sum_{a \in A, b \in B} \omega_A(a)\omega_B(b) \\
&= \left(\sum_{a \in A}\omega_A(a)\right)\left(\sum_{b \in B}\omega_B(b)\right) \\
&= \chi(A)\chi(B).
\end{align*}
\end{proof}
\begin{examp}\label{ex:4pt}
In Example \ref{ex:2pt}, we saw the behavior of two points. While $t$ was small, the magnitude was close to 1, and as $t$ increased, the magnitude approached 2, the number of points. Now, we let $M$ be the following space of four points:
\begin{figure}[H]
\[
\begin{tikzpicture}[auto, node distance = 3cm, main node/.style={dot}]
\node[circle, draw, fill=black,
                        inner sep=0pt, minimum width=4pt](1) at (0,0) {};
\node[circle, draw, fill=black,
                        inner sep=0pt, minimum width=4pt](2) at (5,0) {};
\node[circle, draw, fill=black,
                        inner sep=0pt, minimum width=4pt](3) at (0,-.5) {};
\node[circle, draw, fill=black,
                        inner sep=0pt, minimum width=4pt](4) at (5,-.5) {};
\draw[decoration={brace,raise=7pt},decorate] (1) -- (2) node[midway, above = 10pt] {1};
\draw[decoration={brace, mirror, raise = 7pt},decorate] (1) -- (3) node[midway, left = 10pt] {$\frac{1}{1,000}$};
% \draw (2) -- (4);
% \draw (3) -- (4);
\end{tikzpicture}\]
\caption{$M$, a generalized metric space with four points. $M$ contains structure at two very different scales.}
\label{fig:4pt}
\end{figure}
$M$ is very nearly a product space of two 2-points spaces at different scales, one separated by a distance 1, and the other by $\frac 1 {1,000}$. If the magnitude behaves as we expect it to, we should see the magnitude of $M$ behaving very much like the magnitude of the products of those two 2-point spaces.
\begin{figure}[h]
\[
\begin{tikzpicture}[auto, node distance = 3cm, main node/.style={dot}]
\node[circle, draw, fill=black,
                        inner sep=0pt, minimum width=4pt](1) at (0,0) {};
\node[circle, draw, fill=black,
                        inner sep=0pt, minimum width=4pt](2) at (0,-.5) {};
\node[circle, draw, fill=black,
                        inner sep=0pt, minimum width=4pt](3) at (4, -.25) {};
\node[circle, draw, fill=black,
                        inner sep=0pt, minimum width=4pt](4) at (9,-.25) {};
\node[] at (2, -0.25) {\huge{$\times$}};
\draw[decoration={brace,mirror, raise=7pt},decorate] (1) -- (2) node[midway, left = 10pt] {$\frac{1}{1,000}$};
\draw[decoration={brace, raise = 7pt},decorate] (3) -- (4) node[midway, above = 10pt] {1};
% \draw (2) -- (4);
% \draw (3) -- (4);
\end{tikzpicture}\]
\caption{Two spaces whose product should be very similar to $M$.}
\label{fig:2_2pt}
\end{figure}

In fact, we observe exactly that behavior. Figure \ref{fig:prod_mag} shows the magnitudes of the three spaces as we vary $t$. At every value of $t$, $\chi_t(M)$ is almost exactly the product of the magnitudes of the other two spaces at that $t$ value.
\begin{figure}[H]
\[
\makebox[\textwidth][c]{\includegraphics[width = 0.9\paperwidth]{../Figures/prod_mag.png}}
\]
\caption{The magnitude of $M$ alongside the magnitudes of the factor spaces against $\log(t)$.}
\label{fig:prod_mag}
\end{figure}
\end{examp}

\section{Approximations and Error Bounding}
In Example \ref{ex:4pt}, $M$ was \emph{almost} a product space, so we expected its magnitude to be very nearly the product of the magnitudes of the two 2-point spaces. It would be useful to quantify how similar the magnitudes of two spaces are, given that we know something about how similar those spaces are. To do that, we need some simple tools from linear algebra involving vector and matrix norms. Note that these norms are not directly related to the norms we defined on monoidal categories - both are non-negative real-valued functions, but they operate on different structures.

There are many norms on matrices. Formally, a matrix norm is simply a function from matrices to the non-negative real numbers which satisfies certain properties.

\begin{defn}\label{def:spec}
Let $n$ be given. Let $|v|$ denote the Euclidean vector norm. Then the \emph{spectral norm} on real- (or complex-) valued $n \times n$ matrices, denoted $||\cdot ||_s$, is defined by
\[
||M||_s = \sup_{0 \neq x \in \Rr^n}\left\{ \frac{|Ax|}{|x|} \right\}.
\]
\end{defn}
\begin{rmk}
Definition \ref{def:spec} does not tell us how to find the spectral norm of a matrix. However, it is not difficult to do so. For a given $M$, we can find $||M||_s$ as follows:

Let $M^*$ be the adjoint, or conjugate transpose, of $M$. Let $\lambda_1$ be the largest eigenvalue of the matrix $MM^*$. Then $\lambda_1$ will be real-valued, and $||M||_s=\sqrt{\lambda_1}$. 

We will not prove or use this fact - it is simply here to assure readers that the spectral norm can be computed.
\end{rmk}
This norm is a common way of determining how ``large'' a matrix is. It will allow us to determine which matrices are ``small enough'' to satisfy some approximation theorems.

We also define a second matrix norm, called the $L_1$ norm.
\begin{defn}
Let $A$ be an $n \times n$ matrix. Then the \emph{$L_1$ norm} of $A$, denoted $||A||_2$, is the sum of the absolute values of entries of $A$. In other words,
\[
||A||_2 = \sum_{i = 1}^n\sum_{j = 1}^n |A_{ij}|.
\]
\end{defn}
\begin{rmk}
This is one of an infinite number of $L_p$ norms, one for each positive integer $p$. For any matrix $A$ and positive integer $p$, 
\[
||A||_p = \sqrt[p]{\sum_{i = 1}^n\sum_{j = 1}^n |A_{ij}|^p}.
\]
\end{rmk}





There are three simple properties of both norms which we will need.
\begin{lemma}\label{lem:norm_prop}
Let $A, B$ be $n \times n$ matrices. Let $||\cdot||$ be \emph{either} the spectral norm or the $L_1$ norm. Then
\begin{enumerate}
\item \[
||AB|| \leq ||A||\times ||B||,
\]

\item
\[
||A + B|| \leq ||A|| + ||B||.
\]
\item 
\[
||A|| = 0 \iff A = 0.
\]
\end{enumerate}
\end{lemma}
\begin{proof}
\begin{description}
\item[Spectral Norm:]
\begin{enumerate}
\item By the definition of the spectral norm, for any $v \in \Rr^n$,
\begin{align*}
|ABv| &\leq |A(||B||_s \times |v|)|\\
&\leq (||A||_s \times ||B||_s)|v|.
\end{align*}
It follows that 
 \[
||AB|| \leq ||A||_s\times ||B||.
\]
\item For any $v \in \Rr^n$,
\begin{align*}
|(A + B)v| &= |Av + Bv|\\
 &\leq |Av| + |Bv|\\
 &\leq ||A||_s + ||B||_s)v.
\end{align*}
\item It is clear that if $A=0$ then $||A||_s=0$. Now suppose that $A$ is not zero. Then the column space of $A$ has nonzero dimension, so there exists a $v$ such that $Av \neq 0$. So $||A||_s \geq \frac{|Av|}{|v|} > 0$.
\end{enumerate}
\item[$L_1$ Norm:]
\begin{enumerate}
\item We have
\begin{align*}
|AB_ij| &= \left| \sum_{k = 1}^nA_{ik}B{kj}\right|\\
&\leq \sum_{k = 1}^n|A_{ik}||B{kj}|. \text{ Therefore}\\
||AB||_1 &\leq \sum_{i = 1}^n\sum_{j=1}^n\sum_{k=1}^n|A_{ik}||B{kj}|\\
&\leq \sum_{i = 1}^n\sum_{j=1}^n\sum_{k=1}^n\sum_{l=1}^n|A_{ik}||B_{lj}|\\
&= ||A||_1||B||.
\end{align*}
\item For each $i,j$, $|(A + B)_{ij}| = |A_{ij} + B_{ij}| \leq |A_{ij}| + |B_{ij}|$. The result follows.
\item If $A=0$ then $||A||_1=0$. On the other hand, if any entry of $A$ is nonzero, then $||A||_1 > 0$ by definition.
\end{enumerate}
\end{description}
\end{proof}
Again, there are many other matrix norms. For our purposes, when a theorem uses an arbitrary matrix norm, we will assume that it satisfies Lemma \ref{lem:norm_prop}. 

Now that we have some basic technology for working with matrix norms, we would like to show some results about limits and series convergence of matrices. To do that, we need to establish some notation. 
\begin{defn}\label{def:mat_conv}
Let $A_i$ be an infinite sequence of $m \times n$ matrices, and let $||\cdot||$ be some matrix norm (the spectral norm, the $L_1$ norm, or another matrix norm). Suppose we have an $m \times n$ matrix $L$ such that 
\[
\lim_{i \rightarrow \infty} ||A_i - L|| = 0.
\]
Then we say that 
\[
\lim_{i \rightarrow \infty} A_i = L.
\]
\end{defn}
It turns out that if the matrices involved are finite-dimensional, the choice of norm does not matter - a sequence will either always converge or always diverge. So there is no need to specify the norm when stating a fact about matrix convergence.

This is an instance of the general definition of a convergent sequence in a metric space, where our chosen norm induces a metric. Furthermore, for this discussion, we will be concerned primarily with norms of matrices, so this definition of convergence makes sense for us to use.

\begin{lemma}\label{lem:lim0}
Let $E$ be a real-valued $n \times n$ matrix.  Suppose that for some matrix norm $||\cdot||$, $||E|| < 1$. Then:
\[
\lim_{i \rightarrow \infty}E^i = 0.
\]
\end{lemma}
\begin{proof}
Recall that we are supposing that any norm we use satisfies Lemma \ref{lem:norm_prop}.

Let $||E|| < 1$. Let $\vec{0}$ represent be the zero matrix. We now use Lemma \ref{lem:norm_prop}:
\begin{align*}
\lim_{i \rightarrow \infty}|| E^i - \vec{0}|| &\leq \lim_{i \rightarrow \infty}|| E^i|| - ||\vec{0}|| \\
&\leq \lim_{i \rightarrow \infty}|| E||^i\\
&= 0.
\end{align*}
Then, by Definition \ref{def:mat_conv},
\[
\lim_{i \rightarrow \infty} E^i = 0.
\]
\end{proof}

This allows us to prove a useful fact which allows us to approximate the inverses of certain matrices.
\begin{thm}\label{thm:approx}
Let $A$ be an invertible $n \times n$ matrix, let $||\cdot||$ be some matrix norm, and let $E$ be an $n \times n$ matrix such that $||E|| < \frac 1{||A^{-1}||}$. Then $(A-E)$ is invertible, and
\begin{align*}
(A - E)^{-1} &= A^{-1} + A^{-1}EA^{-1} + A^{-1}EA^{-1}EA^{-1} + \cdots\\
&= A^{-1} + A^{-1}\sum_{i = 1}^\infty (EA^{-1})^i.
\end{align*}
\end{thm}
\begin{proof}
We apply Lemma \ref{lem:lim0}
\begin{align*}
(A-E)  A^{-1}\sum_{i = 0}^\infty (EA^{-1})^i
&= \left(\sum_{i = 0}^\infty (EA^{-1})^i\right) - \left( \sum_{i = 0}^\infty (EA^{-1})^{i+1}\right)\\
&= I - \lim_{i \rightarrow \infty} (EA^{-1})^i\\
&= I
\end{align*}
\end{proof}
\begin{rmk}
Recall that, because we rely on Lemma \ref{lem:lim0}, the meaning of this theorem is governed by Definition \ref{def:mat_conv}. That is what the equals sign really means. 
\end{rmk}
This fact makes sense, in particular if we consider the spectral norm. $||A^{-1}||_s$ is the inverse of the smallest eigenvalue of $AA^*$, and it tells us how much $A$ is capable of reducing the euclidean norm of a vector it multiplies. So if that value is very large, $A$ is very nearly not invertible. In that case, it makes sense that adding a relatively small error matrix might change the inverse significantly. 

Next, we use Theorem \ref{thm:approx} to find a closed-form bound on how much a perturbation matrix changes the magnitude of a generalized metric space.

Let $A$ be a generalized metric space which admits M\"obius inversion. Suppose we know $\chi(A)$, the magnitude of $A$, and we would like to know how much adding $E$, a perturbation matrix, to $\zeta_A$, will change the magnitude. Suppose that $||EA^{-1}|| < 1$ for some matrix norm, so we can use Theorem \ref{thm:approx}.

The magnitude of $A$ is equal to the sum of the entries of $\zeta_A^{-1}$. So the $L_1$ norm of the difference between $\zeta_A$ and $\zeta_A + E$ will be an upper bound on how much adding $E$ changes the magnitude of $A$. We can repeatedly use Lemma \ref{lem:norm_prop} to isolate the error term:
\begin{align}
\left|\left|\zeta_A^{-1} - \zeta_A^{-1} - \zeta_A^{-1}\sum_{i = 1}^\infty (EA^{-1})^i\right|\right|_1 \nonumber 
&= \left|\left|\zeta_A^{-1}\sum_{i = 1}^\infty (EA^{-1})^i\right|\right|_1 \nonumber \\
&\leq ||\zeta_A^{-1}||_1\sum_{i = 1}^\infty ||(EA^{-1})^i||_1. \label{eq:series}
\end{align}

Suppose that $||EA^{-1}||_1 <1$. Then we can convert \eqref{eq:series} to a closed form. 
\begin{align*}
||\zeta_A^{-1}||_1\sum_{i = 1}^\infty ||EA^{-1}||_1^i &\leq
||\zeta_A^{-1}||_1 \sum_{i = 1}^\infty||(\zeta_A + E)^{-1}||_1^i\\
 &\leq ||\zeta_A^{-1}||\frac{||EA^{-1}||_1}{1-||EA^{-1}||_1}.
\end{align*}

However, as we mentioned earlier, all matrix norms are equivalent when it comes to convergence. So if $||EA^{-1}||_1 \geq 1$ but, for example, $||EA^{-1}||_s <1$, the above would no longer work, but we would still expect \eqref{eq:series} to converge. This might be useful, because the spectral norm is often smaller than the $L_1$ norm. To make this work, we will need some way of replacing the $L_2$ norm with a spectral norm. It turns out we can do that:
\begin{lemma}\label{lem:norm_switch}
Let $A$ be an $n \times n$ matrix. Then
\[
||A||_1 \leq n^2||A||_s.
\]
\end{lemma}
\begin{proof}
Let $m$ be the absolute value of the largest element of $A$. Suppose that for some $i,j, |A_{ij}| > ||A||_s$. Let $e_j$ be the column vector with a one in index $j$ and zeros everywhere else. Then $Ae_j$ will have $A_{ij}$ in its $i$'th index, so $|Ae_j| \geq |A_{ij}| >||A||_s$. Because $|e_j|=1$, this is a contradiction. So $m \leq ||A||_1$. $||A||_1 \leq n^2m$, so $||A||_1 \leq n^2||A||_s$.
\end{proof}

This means that we can rewrite \eqref{eq:series} in the following way:
\begin{align*}
||\zeta_A^{-1}||_1\sum_{i = 1}^\infty ||EA^{-1}||_1^i &\leq n^2||\zeta_A^{-1}||_1\sum_{i = 1}^\infty ||(EA^{-1})^i||_s^i\\
&\leq n^2||\zeta_A^{-1}||_1\sum_{i = 1}^\infty ||(EA^{-1})||_s^i\\
&\leq n^2||\zeta_A^{-1}||\frac{||EA^{-1}||_s}{1-||EA^{-1}||_s}.
\end{align*}

We have (possibly) reduced the value of the norm, at the price of a factor of $n^2$. We can exchange the $L_1$ norm for other matrix norms in a similar fashion, using different constants.

We demonstrate this approximation of magnitude through an example:
\begin{examp}
We will consider a situation quite similar to Example \ref{ex:4pt}. Let $M$ be the following generalized metric space:
\begin{figure}[H]
\[
\begin{tikzpicture}[auto, node distance = 3cm, main node/.style={dot}]
\node[circle, draw, fill=black,
                        inner sep=0pt, minimum width=4pt](1) at (0,0) {};
\node[circle, draw, fill=black,
                        inner sep=0pt, minimum width=4pt](2) at (5,0) {};
\node[circle, draw, fill=black,
                        inner sep=0pt, minimum width=4pt](3) at (0,-.5) {};
\node[circle, draw, fill=black,
                        inner sep=0pt, minimum width=4pt](4) at (5,-.5) {};
\draw[decoration={brace,raise=7pt},decorate] (1) -- (2) node[midway, above = 10pt] {$d$};
\draw[decoration={brace, mirror, raise = 7pt},decorate] (1) -- (3) node[midway, left = 10pt] {1};
% \draw (2) -- (4);
% \draw (3) -- (4);
\end{tikzpicture}\]
\caption{}
\label{fig:prod}
\end{figure}
And define $A,B$ as follows:

\begin{figure}[H]
\[
\begin{tikzpicture}[auto, node distance = 3cm, main node/.style={dot}]
\node[circle, draw, fill=black,
                        inner sep=0pt, minimum width=4pt](1) at (9,0) {};
\node[circle, draw, fill=black,
                        inner sep=0pt, minimum width=4pt](2) at (9,-.5) {};
\node[circle, draw, fill=black,
                        inner sep=0pt, minimum width=4pt](3) at (0, -.25) {};
\node[circle, draw, fill=black,
                        inner sep=0pt, minimum width=4pt](4) at (5,-.25) {};
\draw[decoration={brace,mirror, raise=7pt},decorate] (1) -- (2) node[midway, left = 10pt] {1};
\draw[decoration={brace, raise = 7pt},decorate] (3) -- (4) node[midway, above = 10pt] {d};
% \draw (2) -- (4);
% \draw (3) -- (4);
\end{tikzpicture}\]
\caption{Again, two spaces whose product should be very similar to $M$.}
\label{fig:factors}
\end{figure}
Suppose that we are trying to approximate $M$ using a product space of $A$ and $B$. We are interested in finding a bound on the error of our estimate.

Let $\zeta_A, \zeta_B, \zeta_{A \times B}, \zeta_M$ be the similarity matrices of $A, B, A \times B$, and $M$ respectively. Recall that we are using the norm $|x| = e^{-tx}$, where we will vary $t$. So the matrices are as follows:
\begin{align*}
\zeta_A &= 
\begin{bmatrix}
1 & e^{-td} \\
e^{-td} & 1
\end{bmatrix}\\
\zeta_B &= 
\begin{bmatrix}
1 & e^{-t} \\
e^{-t} & 1
\end{bmatrix}\\
\zeta_{A \times B} = \zeta_A \otimes \zeta_B &= 
\begin{bmatrix}
1 & e^{-t} & e^{- td} & e^{-t(1 + d)}\\
e^{-t} & 1 & e^{-t(1 + d)}& e^{- td}\\
e^{-td} & e^{-t(1 + d)} & 1 & e^{-t}\\
e^{-t(1 + d)} & e^{-td} & e^{-t} & 1
\end{bmatrix}\\
\zeta_M &= 
\begin{bmatrix}
1 & e^{-t} & e^{- td} & e^{-t\sqrt{1 + d^2}}\\
e^{-t} & 1 &  e^{-t\sqrt{1 + d^2}}& e^{- td}\\
e^{-td} &  e^{-t\sqrt{1 + d^2}} & 1 & e^{-t}\\
e^{-t\sqrt{1 + d^2}}& e^{-td} & e^{-t} & 1
\end{bmatrix}
\end{align*}
The similarity matrices of $A \times B$ and $M$ have most entries identical. We can quantify the difference between them. For any $d, t \in (0, \infty)$, let 
\begin{align*}
\ep_{dt} = e^{-t(1 + d)} - e^{-t\sqrt{1 + d^2}}\\
E_{dt} = \begin{bmatrix}
0 & 0 & 0 & \ep_{dt} \\
0 & 0 & \ep_{dt} & 0 \\
0 & \ep_{dt} & 0 & 0 \\
\ep_{dt} & 0 & 0 & 0
\end{bmatrix}
\end{align*}
Then for each $d,t$,
\[
\zeta_M = \zeta_{A \times B} - E_{dt}.
\]
Note that I am omitting the $d,t$ subscripts from the similarity matrices to avoid clutter, but they do depend on $d$ and $t$.

If we are interested in bounding the difference in magnitude of $M$ and $A \times B$, then we can apply Theorem \ref{thm:approx}. We fix a $d$ to determine the relative scales of the spaces. Then, for any $t \in (0, \infty):$
\[
\mu_M = \mu_{A \times B} + \mu_{A \times B}\sum_{i = 1}^\infty (E_{dt}\zeta_{A \times B}^{-1})^i.
\]

This formula gives us a way of bounding how different the magnitude of the Euclidean spaces is from the product space. We can numerically calculate values for $\chi_t(M), \chi_t(A \times B)$, and the error term, and plot them.

If $d = 0.1$, so the scales differ by a factor of ten, then we get the following as we vary $t$:
\begin{figure}[H]
\[
\makebox[\textwidth][c]{\includegraphics[width = 0.9\paperwidth]{../Figures/error_10.png}}
\]
\caption{Magnitudes and error bounds, $d = 0.1$, error calculated to third term, against $\log(t)$ where $t$ is the scale factor.}
\label{fig:approx10}
\end{figure}
We can see that the Euclidean space and the product space have very similar magnitude values, and the error bounds are quite tight. Note that the period where there are two clusters is much more subtle now, as the points merge together in faster succession. 

If we were trying to approximate $M$ with $A \times B$, then we would not have access to the orange line in Figure \ref{fig:approx10}. But in this case, we can see that we could determine the line with good accuracy using only $A \times B$ and the error bounds.

As we move the value of $d$ closer to 1, we expect both the error bounds and the actual error to increase. The product space and the Euclidean space will differ more as their scales become more similar. Recall that the only pairs of points which differ at all in distance between $A \times B$ and $M$ are points at opposite corners (see Figure \ref{fig:4pt}). When one dimension is extremely small, that diagonal distance is very close to the length of the longer side. On the other hand, in the product space the distance is the sum of the dimensions. When the small dimension is small enough, that, too, is essentially the distance of the longer side.

On the other hand, as $d$ approaches 1, the space becomes square. If $d=1$, then the distance of opposite points in $M$ will be $\sqrt{2}$, while in the product space the distance will be 2. This is quite a significant difference. In the next example, we look at exactly this case.

Here we see what happens when $d=1$, so our dimensions are identical. Both spaces move smoothly from having a single cluster to having four clusters, and there is more error. 
\begin{figure}[H]
\[
\makebox[\textwidth][c]{\includegraphics[width = 0.9\paperwidth]{../Figures/error_1.png}}
\]
\caption{Magnitudes and error bounds, $d = 1$, error calculated to third term, against $\log(t)$ where $t$ is the scale factor.}
\end{figure}
\end{examp}
However, even if the orange line were removed, we would still have a good idea of the cluster structure of $M$. We would know that $M$ starts with one cluster, and then at a scale factor of approximately $10^{-\frac12} \approx 3$, the points begin to move apart. They smoothly separate until four clusters form. Because of the error bounds, we know that given the behavior of $A \times B$, the magnitude of $M$ can't differ too much from that behavior.

\end{chapter}

\bibliographystyle{abbrv}
\bibliography{bibliography.bib}






















% % % Extra stuff/deleted portions
\iffalse


A more familiar way of counting the orbits of a group is the following:
\begin{thm}[Burnside's Lemma]\label{burnside}
Let $G$ be a finite group acting on a finite set $\Omega$. Then:
\[
|\orb(\Omega, G) = \frac{1}{G}\sum_{g \in G}\fix_g(\Omega).
\]
\end{thm}
Burnside's Lemma allows us to compute the number of orbits of a group action if we have access to the size of the fixed sets of each group element. On the other hand, Theorem \ref{orbit_count} requires that we know $f$, the size of the fixed set of each \emph{subgroup}. These are two different perspectives on the same question, and each might be more or less useful depending on the situation. 

We can replace the vectors $f,g$ with matrices to get the following generalization, which :
\begin{thm}[Two Variable M\"obius Inversion]\label{mob_inv_old}
Let $X$ be a lower finite poset.  Let $f,g:X \times X \rightarrow \Rr$ be given. Let $\zeta:X \times X \rightarrow \Rr$ be defined as usual, and let $\mu = \zeta^{-1}$. Then:
\begin{align*}
f = g \zeta &\iff g = f \mu;\\
f = \zeta g &\iff g = \mu f.
\end{align*}
Equivalently,
\begin{align*}
\forall x,y \in X, f(x,y) = \sum_{z \in X}g(x,z)\zeta(z,y) &\iff \forall x,y \in X, g(x,y) = \sum_{z \in X}f(x,z)\mu(z,y);\\
\forall x,y \in X, f(x,y) = \sum_{z \in X}\zeta(x,z)g(z,y) &\iff \forall x,y \in X, g(x,y) = \sum_{z \in X}\mu(x,z)f(z,y).
\end{align*}
\begin{proof}
Again, the result is trivial when we treat $f,g,\zeta,\mu$ as matrices.
\end{proof}
\end{thm}


\begin{defn}
The \emph{incidence algebra} on a lower finite poset $X$ is over the set of functions from $X \times X$ to $\Rr$ which map $(a,b)$ to zero if $a \nleq b$. So the elements of the incidence algebra are 
\[\{f:X \times X \rightarrow \Rr:a \nleq b \Rightarrow f(a,b) = 0\}
\]
Scalar multiplication and addition are defined pointwise.  Multiplication is denoted by ``*'', and is defined:
\[(f* g)(a,b) = \sum_{a \in X}f(a,c)g(c,b) \left( = \sum_{a \leq c \leq b}f(a,c)g(c,b)\right).\]
\end{defn}
\begin{rmk}
The requirement that $X$ be lower finite ensures that the above sum has a finite number of nonzero terms, so is finite-valued.
\end{rmk}
\begin{rmk}
We use $\Rr$ as the codomain of the functions in the incidence algebra. But this can be generalized to other commutative rings with identity.
\end{rmk}
I include the version with bounds for clarity, but they make no difference, because all other terms will go to zero anyway by the condition on elements. Considering the sum without the bounds, we see that this is identical to matrix multiplication. It will be useful to formalize this:
\begin{lemma}\label{mat_eq}
Let $X$ be a lower finite poset, with $|X|=n$. Choose a total ordering on $X$ such that the partial order is respected - in other words, label the elements with integers such that if $x_i \leq x_j, i \leq j$. Let $f,g$ be elements of the incidence algebra on $X$. Let $F$ be the (possibly infinite) square matrix where $F_{ij}=f(x_i, x_j)$. Define $G$ likewise. Then:
\begin{enumerate}[a)]
\item $F$ and $G$ are upper triangular; and
\item For any $i,j, (f * g)(x_i, x_j)=FG_{ij}$.
\end{enumerate}
\end{lemma}
\begin{proof}
\begin{enumerate}[a)]
\item If $i>j$, by how we defined $F$, $x_i \nleq x_j$. So $F_{ij} = f(x_i, x_j) = 0$. 
\item
The result follows directly from the definition of matrix multiplication. Let $i,j\leq n$ be given. Then:
\begin{align*}
(f * g)(x_i, x_j) &= \sum_{y \in X} f(x_i,y)g(y,x_j)\\
&=FG_{ij}
\end{align*}
\end{enumerate}
\end{proof}



The \textbf{identity function} $\delta$ is defined by 
\[\delta(a,b) = \begin{cases} 1 & a = b \\ 0 & else. \end{cases}
\]


The definition for ordinary categories is fairly natural. The version for generalized metric spaces is less intuitive, but the definitions are surprisingly consistent. Consider three objects $a,b,c$ in an ordinary category. The numbers above the arrows indicate the number of morphisms between adjacent objects:

\[
\begin{tikzpicture}[auto, node distance = 3cm, main node/.style={dot}]

\node[label = below:{a}, circle, draw, fill=black,
                        inner sep=0pt, minimum width=4pt](1) at (0,0) {};
\node[label = below:{b}, circle, draw, fill=black,
                        inner sep=0pt, minimum width=4pt](2) at (3,0) {};
\node[label = below:{c}, circle, draw, fill=black,
                        inner sep=0pt, minimum width=4pt](3) at (6,0) {};                        

\draw[-latex] (1) -- (2) node[midway, above = 3pt] {\textit{k}};
\draw[-latex] (2) -- (3) node[midway, above = 3pt] {\textit{l}};

\end{tikzpicture}\]
Then, as a result of the composition axioms, there must be a morphism $g \circ f$ from $a$ to $c$ for each pair of morphisms $f \in C(a,b), g \in C(b,c)$. This means that there are $kl$ morphisms from $a$ to $c$ which are implied by the morphisms shown above. So $\zeta(a,c) \geq \zeta(a,b)  \zeta(b,c)$. 

Now, consider the same situation in a generalized metric space. We know from one of the axioms that
\[
C(a,c) \leq C(a,b) + C(b,c).
\]
The inverse exponential function is monotonically decreasing. So we simply apply it to both sides and flip the inequality to get the same result, that $\zeta(a,c) \geq \zeta(a,b) \zeta(b,c)$.

There is another way of looking at essentially the same property. The reason the addition (+) operator appears in the second axiom in the generalized metric space definition is because addition is what is called the \textit{monoidal product} in the enriched category. The equivalent in an ordinary category is the tensor product, $\otimes$. In terms of number of morphisms, $\otimes$ behaves like multiplication. We want a fact about addition to give us a fact about multiplication. Exponentiating is the most natural way to translate multiplication into addition, so we can apply the axiom.
\fi
\end{document}

