\documentclass[12pt]{pom_thesis}

\author{Leo Selker}
\advisor{Vin de Silva}
\title{Absolutely Fascinating Thesis Title}
\iffalse
\theoremstyle{plain}{
\newtheorem{thm}{Theorem}[chapter]
\newtheorem{lemma}[thm]{Lemma}
\newtheorem{prop}[thm]{Proposition}
\newtheorem{cor}[thm]{Corollary}
\newtheorem{conj}[thm]{Conjecture}
}

\theoremstyle{definition}{
\newtheorem{defn}[thm]{Definition}
\newtheorem{examp}[thm]{Example}
\newtheorem{rmk}[thm]{Remark}
}\fi
\begin{document}

\maketitle

\begin{abstract}In this paper we don't really do much.  However, there are a lot of {\it real} theorems that still need to be proved. That is what you will probably do in your thesis.
\end{abstract}

\pagenumbering{roman}
\tableofcontents

\newpage
\pagenumbering{arabic}

% EXAMPLE STUFF
% this is how you begin a chapter:
% the label is so that you can refer to it later
\begin{chapter}{Boring Title for the First Chapter}
\label{Intro}

Let us do some math:
% to start a new line in math mode
% either write your equation inside $$ .... $$
$$ \Delta(h) = h_{(1)} \otimes h_{(2)} $$
% or inside \[ ... \]
\[ \Delta(h) = h_{(1)} \otimes h_{(2)} \]
% or:
\begin{equation*}
\Delta(h) = h_{(1)} \otimes h_{(2)}
\end{equation*}

Here is how you declare a theorem:

\begin{thm}{A Big Fat Theorem.} \label{thm:BigFat}
We assert that the following is true:
\begin{equation} \label{eqn:Stupid}
x = 1, y = 1 \Rightarrow x + y = 2
\end{equation}
\end{thm}

Let us first consider:
%Here is how you declare a lemma:
\begin{lemma}{A Small but Important Lemma.} \label{lemma:LittleLem}
If $x = a$, and $y = b$, then $x + y = a + b$.
\end{lemma}

We can then see that Lemma \ref{lemma:LittleLem} implies Theorem \ref{thm:BigFat} by letting $a = 1$ and $b = 1$ in Equation (\ref{eqn:Stupid}). See how we refer to a previously labeled item in the text?

\section{A delightful new section}

Some text for the section should go here. And let us look at footnotes.\footnote{This is one way to use a footnote.} \footnotemark

\footnotetext{Here is a second way to introduce a footnote}


\begin{thm}
hmmm
\end{thm}

Here is how you call the proof environment:

\begin{proof}
hmmmm
\end{proof}

% you should close a chapter before beginning a new one!
\end{chapter}

\begin{chapter}{Cooler Title for the Second Chapter} As we saw in Chapter \ref{Intro}, everything can be
made to be complicated. (See, for example,  Figure \ref{fig1}.) This is usually not a good idea unless you want to lose your audience.

\begin{figure}
\centering{
\includegraphics[height=5cm, width=6cm]{templatefig1}
}
\caption{Graphics can really snaz it up!}
\label{fig1} % label of the figure has to go after the caption to be accurate!
\end{figure}

Most importantly, \textcolor{red}{\bf NEVER DIVIDE BY ZERO} unless, of course, you are wearing your protective
divide-by-zero suit (See \cite{Ab80} for the terrible consequences which might result. And this is how you cite multiple references: \cite{Ab80, BlTaWe, Bo1}. And if you wanted to, you could refer to specific pages: \cite[pages 567--569]{Bo2}).

\section{Another fascinating section}

Some text needs to go here.

\subsection{And sometimes you will need subsections...}

More text goes here.
\end{chapter}
\end{document}