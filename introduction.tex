\documentclass{article}
\input{leoheader}

\begin{document}
\begin{center}{\Large Thesis Topic}\\ \vspace{1 ex} \end{center}
A \textbf{metric space} is a set $X$ together with a function, $d: X \times X \rightarrow \Rr^+$, which must satisfy:
\begin{align*}
d(a,a) &= 0;\\
d(a,b) & > 0 \text{ if } a \neq bl\\
d(a,b) &= d(b,a);\\
d(a,b) &\leq d(a,c) + d(c,b)
\end{align*}
$ \forall a,b,c \in X$.

A simple example of a metric space is $\Rr^n$. In that case, the natural metric is the Euclidean metric: 
\[
d\left((a_1,..., a_n), (b_1,...,b_n)\right) = \sqrt{\sum_{i=1}^n (a_i-b_i)^2}.
\]
Multi-dimensional data can be represented as a finite subset of $\Rr^n$, with the Euclidean metric, where each point represents a data point, and each dimension represents a feature of the data. If two points are close together under this metric, they must vary a relatively small amount in every feature, so in some circumstances are likely to be similar in some way which we care about.

In that case, it can be useful to cluster the data, to see if any natural groups emerge. But it might not be immediately clear on what scale clustering makes sense. So we use \textbf{hierarchical clustering}: We consider the data on many scales, and cluster points which are close together under each scale. We take the information about which points are in which clusters for all scales at once and put it together to form a \textbf{persistent set}. This object holds information about the object at all scales.

We can also approach a finite metric space from an entirely different direction. Tom Leinster defines the Euler characteristic of what is called a \textbf{category}, which for the moment can be considered a generalization of metric spaces (not really a good way to look at it in general). Each finite metric $X$ space can be converted into a category, and we can then take that category's Euler characteristic, which gives us what we call the metric space's \textbf{magnitude} (denoted $|X|$). This is different from $\#(X)$, the cardinality or size of $X$. Let $X_k$ denote $X$ scaled by a non-negative real number $k$, meaning $X$ with all pairwise distances multiplied by $k$. Then as $k \rightarrow \infty, |X_k| \rightarrow \#(X_k) = \#(X)$. But if we scale $X$ down, the magnitude decreases (or may cease to be defined). In particular $|X_0| = 1$. 

We wonder to what extent it is interesting to consider $|X_k|$ to be related to the number of clusters in $X$ viewed at some scale inversely related to $k$. We also wonder to what extent these two notions of clustering are related.
\end{document}