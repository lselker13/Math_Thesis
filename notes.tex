\documentclass{article}
\usepackage{graphics}
\usepackage{fancyhdr}
\usepackage{amssymb}
\usepackage{amsmath}
\usepackage{amsthm}
\usepackage{graphicx}
\usepackage{enumerate}
\newcommand{\coursenumber}{176}
\newcommand{\Rr}{\mathbb{R}}
\newcommand{\Cc}{\mathbb{C}}
\newcommand{\Zz}{\mathbb{Z}}
\newcommand{\Nn}{\mathbb{N}}
\newcommand{\Ff}{\mathbb{F}}
\newcommand{\Qq}{\mathbb{Q}}
\newcommand{\Aa}{\mathbb{A}}
\newcommand{\fix}{\text{fix }}
\newcommand{\cl}{\text{cl }}
\DeclareMathOperator{\ob}{ob}

\topmargin  18pt  	% 27pt
\headheight 12pt  	% 12pt
\headsep    35pt 	% 45pt
\textheight 8.0in
\textwidth  6.2in

\oddsidemargin   22pt      % Left margin on odd-numbered pages.
\evensidemargin  22pt      % Left margin on even-numbered pages.
\marginparwidth  72pt       % Width of marginal notes.
\marginparsep    11pt      % Horizontal space between outer margin and 
                           % marginal note

\pagestyle{fancy}
\lhead{Leo Selker }
\chead{}
\rhead{2016-2017}
\lfoot{}
\cfoot{\thepage}
\rfoot{}
\renewcommand{\headrulewidth}{0pt}
\renewcommand{\footrulewidth}{0pt}
\newtheorem{cc}{cc} % basis for counter
\newtheorem{sumMobiusWeighting}[cc]{Lemma}
\theoremstyle{definition}
\newtheorem{defCatMobius}[cc]{Definition}
\newtheorem{defWeighting}[cc]{Definition}

    
\begin{document}

\begin{center}{\Large Thesis Notes}\\ \vspace{1 ex} \end{center}


\vspace{2 ex}
\noindent

Big idea: Magnitude of a metric space is large as $t$, some scale factor, gets large - $t$ scales the metric, and as points get farther apart, they are "counted separately." On the other hand, hierarchical clustering records fewer clusters as $r$, a scale factor, gets large. What is the relationship between these two "counts of number of clusters?" Is magnitude even such a count?

Mobius inversion:
\begin{align*}
g(n) &= \sum_{d | n} f(d) \\
\Rightarrow f(n) &= \sum_{d | n} \mu(\frac dn) g(n)
\end{align*}

\begin{defCatMobius}
A category $\Aa$ has \textbf{Mobius Inversion} $\mu$ if $\mu$ satisfies
\[
\sum_b \mu(a,b)\zeta(b,c) = \delta(a,c) = \sum_b \zeta(a,b)\mu(b,c),
\]
with each equality implying the other by finite dimensionality (TODO: show)
\end{defCatMobius}
\begin{defWeighting}
A \textbf{weighting} on a category $\Aa$ is a function $w_a$ on $\ob(\Aa)$ such that for all $a \in \Aa$,
\[
\sum_b \zeta(a,b)\omega_b = 1.
\]
\end{defWeighting}
\begin{sumMobiusWeighting}
If $\Aa$ has Mobius inversion, then $\omega_a = \sum_b \mu(a,b)$ is a weighting on $\Aa$.
\end{sumMobiusWeighting}
\begin{proof} For all $a \in \Aa$,
\begin{align*}
\sum_b \zeta(a,b)\omega_b &= \sum_b\left( \zeta(a,b)\sum_c \mu(b,c)\right)\\
&=\sum_b\sum_c\zeta(a,b)\mu(b,c)\\
&=\sum_c\left(\sum_b\zeta(a,b)\mu(b,c)\right)\\
&=\sum_c \delta(a,c)\\
&=1.
\end{align*}
\end{proof}



\end{document}
